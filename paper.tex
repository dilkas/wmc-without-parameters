\documentclass{article}
\usepackage[utf8]{inputenc}
\usepackage[UKenglish]{babel}
\usepackage[UKenglish]{isodate}
\usepackage{fullpage}
\usepackage{amsthm}
\usepackage{amsfonts}
\usepackage{amsmath}
\usepackage{mathtools}
\usepackage[capitalise]{cleveref}
\usepackage{bm}
\usepackage{booktabs}
\usepackage{tikz}
\usepackage{xcolor}
\usepackage[backgroundcolor=lightgray]{todonotes}

\newtheorem{theorem}{Theorem}
\newtheorem{lemma}{Lemma}
\newtheorem{proposition}{Proposition}
\theoremstyle{definition}
\newtheorem{definition}{Definition}
\theoremstyle{remark}
\newtheorem*{remark}{Remark}

\Crefname{property}{Property}{Properties}
\Crefname{condition}{Condition}{Conditions}
\creflabelformat{condition}{#2(#1)#3}

\DeclareMathOperator{\WMC}{WMC}
\DeclareMathOperator{\nWMC}{nWMC}
\DeclareMathOperator{\id}{id}
\DeclareMathOperator{\End}{End}

\usetikzlibrary{cd}

\tikzset{
  Subset/.style={
    draw=none,
    every to/.append style={
      edge node={node [sloped, allow upside down, auto=false]{$\subset$}}}
  }
}

\title{On the Limitations of Weighted Model Counting}
\author{Paulius Dilkas}

\begin{document}
\maketitle

%\section{Introduction}

%Contributions
%\begin{itemize}
%\item Equip polyadic algebras with a measure.
%\item Show that it is equivalent to WMC.
%\end{itemize}

% Big advantage: using algebras we can represent an incomplete state of
% knowledge, e.g., we know m(a \/ b), but not m(a) or m(b). In infinite BAs, one
% can have measure undefined for some elements.

\section{WMC as a Measure}

\subsection{Preliminaries}

\begin{definition} \label{def:ba}
  A \emph{Boolean algebra} is a tuple $(\mathbf{B}, \land, \lor, \neg, 0, 1)$ of
  a set $\mathbf{B}$ with operations $\land, \lor, \neg$ and elements $0, 1 \in
  \mathbf{B}$ such that the following axioms hold for all $a, b, \in
  \mathbf{B}$:
  \begin{itemize}
  \item both $\land$ and $\lor$ are associative and commutative;
  \item $a \lor (a \land b) = a$, and $a \land (a \lor b) = a$;
  \item $0$ is the identity of $\lor$, and $1$ is the identity of $\land$;
  \item $\lor$ distributes over $\land$ and vice versa;
  \item $a \lor \neg a = 1$, and $a \land \neg a = 0$.
  \end{itemize}

  Let $a, b \in \mathbf{B}$ be arbitrary. Let $\le$ be a partial order on
  $\mathbf{B}$ defined by $a \le b$ if $a = b \land a$ (or, equivalently, $a
  \lor b = b$), and let $a < b$ denote $a \le b$ and $a \ne b$.
\end{definition}

\todo[inline]{Which definition do I actually need?}
\begin{definition}[\cite{DBLP:books/daglib/0090259,levasseur2012applied}]
  An element $a \ne 0$ of a Boolean algebra $\mathbf{B}$ is an \emph{atom} if
  there is no $x \in \mathbf{B}$ such that $0 < x < a$. Equivalently, $a \ne 0$
  is an atom if, for all $x \in \mathbf{B}$, either $x \land a = a$ or $x \land
  a = 0$. A Boolean algebra is \emph{atomic} if for every $a \in \mathbf{B}
  \setminus \{0 \}$, there is an atom $x$ such that $x \le a$.
\end{definition}

\begin{lemma}[\cite{ganesh2006introduction}]
  For any two distinct atoms $a$, $b$ in a Boolean algebra, $a \land b = 0$.
\end{lemma}

\begin{lemma}[\cite{givant2008introduction}] \label{lemma:atomic}
  All finite Boolean algebras are atomic.
\end{lemma}

% TODO: we don't have to treat 0 as an exception we state that \/ of an empty set is 0 (and /\ of an empty set is 1).
% TODO: use bigvee and bigwedge directly with sets.
\begin{theorem} \label{thm:representation}
  Let $\mathbf{B}$ be a finite Boolean algebra. Then every $x \in \mathbf{B}
  \setminus \{ 0 \}$ can be uniquely expressed as
  \[
  x = \bigvee_{\text{atoms } a \le x} a.
  \]
\end{theorem}
\begin{proof}
  A simple consequence of the theorem that every finite Boolean algebra is
  isomorphic to a field of subsets of a set, where the cardinality of the set is
  equal to the number of atoms in the Boolean algebra.
\end{proof}

\todo[inline]{Remove the requirement for being strictly positive}
\begin{definition}[\cite{gaifman1964concerning}] \label{def:measure}
  A \emph{(strictly positive) measure} on a Boolean algebra $\mathbf{B}$ is a
  function $m : \mathbf{B} \to [0, 1]$ such that:
  \begin{enumerate}
  \item $m(1) = 1$, and $m(x) > 0$ for $x \ne
    0$; \label[property]{property:values}
  \item $m(x \lor y) = m(x) + m(y)$ for all $x, y \in \mathbf{B}$ whenever $x
    \land y = 0$. \label[property]{property:additive}
  \end{enumerate}
\end{definition}

\subsection{New Results}

\todo[inline]{Allow weight to be zero}
\begin{definition} \label{def:wmc}
  Let $\mathbf{B}$ be a finite Boolean algebra, and let $M \subseteq \mathbf{B}$
  be its set of atoms. Let $L \subseteq \mathbf{B}$ be such that every atom $m
  \in M$ can be uniquely expressed as $m = \bigwedge_{i \in I} l_i$ for some $\{
  l_i \}_{i \in I} \subseteq L$, and let $w : L \to \mathbb{R}_{>0}$ be
  arbitrary. The \emph{weighted model count} $\WMC : \mathbf{B} \to
  \mathbb{R}_{\ge 0}$ is defined as
  \[
    \WMC(a) = \begin{cases}
      0 & \text{if } a = 0 \\
      \prod_{i \in I} w(l_i) & \text{if } M \ni a = \bigwedge_{i \in I} l_i \\
      \sum_{i \in I} \WMC(m_i) & \text{if } \mathbf{B} \setminus (M \cup \{ 0
      \}) \ni a = \bigvee_{i \in I} m_i
    \end{cases}
  \]
  for any $a \in \mathbf{B}$. Furthermore, we define the \emph{normalised
    weighted model count} $\nWMC : \mathbf{B} \to [0, 1]$ as $\nWMC(a) =
  \frac{\WMC(a)}{\WMC(1)}$ for all $a \in \mathbf{B}$.
\end{definition}

\begin{proposition}
  $\nWMC$ is a measure for any finite Boolean algebra $\mathbf{B}$.
\end{proposition}
\begin{proof}
  First, note that \cref{property:values} of \cref{def:measure} is satisfied by
  the definition of $\nWMC$. Next, in order to prove \cref{property:additive},
  let $x, y \in \mathbf{B}$ be such that $x \land y = 0$. We want to show that
  \[
    \nWMC(x \lor y) = \nWMC(x) + \nWMC(y)
  \]
  which is equivalent to
  \begin{equation} \label{eq:additivity_proof}
    \WMC(x \lor y) = \WMC(x) + \WMC(y).
  \end{equation}
  If, say, $x = 0$, then \cref{eq:additivity_proof} becomes
  \[
    \WMC(y) = \WMC(0) + \WMC(y) = \WMC(y)
  \]
  (and likewise for $y = 0$). Thus we can assume that $x \ne 0 \ne y$ and use
  \cref{thm:representation} to write
  \[
    x = \bigvee_{i \in I} x_i \quad \text{and} \quad y = \bigvee_{j \in J} y_j
  \]
  for some sequences of atoms $(x_i)_{i \in I}$ and $(y_j)_{j \in J}$. If
  $x_{i'} = y_{j'}$ for some $i' \in I$ and $j' \in J$, then
  \[
    x \land y = \bigvee_{i \in I} \bigvee_{j \in J} x_i \land y_j = x_{i'} \land
    y_{j'} \ne 0,
  \]
  contradicting the assumption. This is enough to show that
  \begin{align*}
    \WMC(x \lor y) &= \WMC\left( \left( \bigvee_{i \in I} x_i \right) \lor \left(\bigvee_{j \in J} y_j \right) \right) = \sum_{i \in I} \WMC(x_i) + \sum_{j \in J} \WMC(y_j) \\
                   &= \WMC(x) + \WMC(y),
  \end{align*}
  finishing the proof.
\end{proof}

\section{An Extension to First-Order Logic}

\subsection{Preliminaries}

What follows is a summary of \cite{halmos2016algebraic}.

% Generic stuff
Let $\mathbf{B}$ be a Boolean algebra (of propositions). Let $X$ be the
(non-empty) domain of discourse. Let $I$ be an index set, elements of which can
be interpreted as variables. The elements of $X^I$ are functions from $I$ to
$X$. For any $x \in X^I$ and $i \in I$, we write $x_i$ to represent $x(i) \in
X$. Let $\mathbf{A^*}$ be the set of all functions $X^I \to \mathbf{B}$, and
note that it forms a Boolean algebra with operations defined pointwise.

% Defining S
Let $T$ be the semigroup of all $I \to I$ transformations. For any $\tau \in
T$, let $\tau_* : X^I \to X^I$ be defined by
\[
  (\tau_* x)_i = x_{\tau i}
\]
for all $x \in X^I$ and $i \in I$. For any (Boolean/polyadic) algebra
$\mathbf{C}$, let $\End(\mathbf{C})$ denote the set of all its endomorphisms. We
can then define $\mathbf{S}$ to be a map $\mathbf{S} : T \to \End(\mathbf{A^*})$
defined by
\[
  \mathbf{S}(\tau)p(x) = p(\tau_* x)
\]
for all $x \in X^I$ and $p \in \mathbf{A^*}$.

% Defining E
For any $J \subseteq I$, let $J_*$ be the relation on $X^I$ defined by
\[
  xJ_*y \quad \iff \quad x_i = y_i \quad \text{for all } i \in I \setminus J
\]
for all $x, y \in X^I$. For any $J \subseteq I$, we then define $\bm\exists(J)$
to be a transformation $\mathbf{A^*} \to \mathbf{A^*}$ defined by
\[
  \bm\exists(J)p(x) = \bigvee_{\substack{y \in X^I,\\ xJ_*y}} p(y)
\]
for all $p \in \mathbf{A^*}$, provided this supremum exists for all $x \in
X^I$\footnote{The universal quantifier $\bm\forall(J)$ is then defined as
  $\bm\forall(J)p = \neg(\bm\exists(J)\neg p)$ for all $p \in \mathbf{A^*}$.}.

Finally, a \emph{functional polyadic (Boolean) algebra}\footnote{To be more
  explicit, a $\mathbf{B}$-valued functional polyadic algebra with domain $X$
  and variables $I$.} is a subalgebra
$\mathbf{A}$ of $\mathbf{A^*}$ such that:
\begin{itemize}
\item $\mathbf{S}(\tau)p \in \mathbf{A}$ for all $p \in \mathbf{A}$ and $\tau
  \in T$;
\item $\bm\exists(J)p \in \mathbf{A}$ for all $p \in \mathbf{A}$ and $J
  \subseteq I$.
\end{itemize}

\begin{definition}
  Similarly to $\bm\exists$, a \emph{constant} $c$ is a map $c: \mathcal{P}(I)
  \to \End(\mathbf{A})$ (Boolean endomorphisms?) such that:
  \begin{itemize}
  \item $c(\emptyset) = \id_{\mathbf{A}}$;
  \item $c(J \cup K) = c(J)c(K)$;
  \item $c(J)\bm\exists(K) = \bm\exists(K)c(J \setminus K)$;
  \item $\bm\exists(J)c(K) = c(K)\bm\exists(J \setminus K)$;
  \item $c(J)\mathbf{S}(\tau) = \mathbf{S}(\tau)c(\tau^{-1}J)$
  \end{itemize}
  for all $J, K \in \mathcal{P}(I)$ and $\tau \in T$. If $J$ is a singleton
  set $\{ i \}$, we will simply write $c(i)$ instead of $c(J)$.
\end{definition}

\subsection{New Results}

\begin{proposition} \label{prop:polyadic_measure}
  Let $\mathbf{B}$ be a finite Boolean algebra with a measure $m :
  \mathbf{B} \to [0, 1]$. Let $\mathbf{A}$ be a $\mathbf{B}$-valued functional
  polyadic algebra with domain $X$ and variables $I$. Let $m^* : \mathbf{A} \to
  \mathbb{R}_{\ge 0}$ be defined by
  \[
    m^*(p) = \sum_{\substack{\text{atoms }y \in \mathbf{B} \text{ s.t.}\\ \exists x \in X^I:\, y \le p(x)}} m(y)
  \]
  for all $p \in \mathbf{A}$. Then $m^*$ is a measure on $\mathbf{A}$.
\end{proposition}

\begin{remark}
  While definitions of $m^*$ such as
  \[
    m^*(p) = m \left( \bigvee_{x \in X^I} p(x) \right)
  \]
  might look tempting, they are not additive.
\end{remark}

\begin{proof}\todo[inline]{Update the proof w.r.t. definitions}
  First, we can show that $m^*(1) = 1$ by observing that
  \[
    m^*(1) = \sum_{\text{atoms } y \in \mathbf{B}} m(y) = m \left(
      \bigvee_{\text{atoms } y \in \mathbf{B}} y \right) = m(1) = 1,
  \]
  where we use \cref{thm:representation} and express $1 \in \mathbf{B}$ as the
  supremum of all atoms in $\mathbf{B}$ \cite{ganesh2006introduction}. Clearly
  $m^*(p) \ge 0$ for all $p \in \mathbf{A}$, so we can restrict the codomain of
  $m^*$ to $[0, 1]$.

  Next, we want to show that $m^*(p) > 0$ for all $p \in \mathbf{A} \setminus \{
  0 \}$. If $p \ne 0$, then there must be some $x' \in X^I$ such that $p(x') \ne
  0$. But then, since finite Boolean algebras are atomic, there must also be an
  atom $y \in \mathbf{B}$ such that $y \le p(x')$. Therefore, $m^*(p) \ge m(y) >
  0$, finishing this part of the proof.

  Let $p, q \in \mathbf{A}$ be such that $p \land q = 0$. We want to show
  that $m^*(p \lor q) = m^*(p) \lor m^*(q)$. First, note that
  \[
    y \le (p \lor q)(x) = p(x) \lor q(x)
  \]
  if and only if
  \[
    y = (p(x) \lor q(x)) \land y = (p(x) \land y) \lor (q(x) \land y)
  \]
  by \cref{def:ba}. Also note that
  \[
    (p(x) \land y) \land (q(x) \land y) = p(x) \land q(x) \land y = (p \land
    q)(x) \land y = 0 \land y = 0,
  \]
  so
  \[
    m(y) = m((p(x) \land y) \lor (q(x) \land y)) = m(p(x) \land y) + m(q(x) \land y)
  \]
  by \cref{def:measure} which then leads to
  \begin{align*}
    m^*(p \lor q) &= \sum_{\substack{\text{atoms }y \in \mathbf{B} \text{ s.t.}\\ \exists x \in X^I:\, y \le (p \lor q)(x)}} m(y) = \sum_{\substack{\text{atoms }y \in \mathbf{B} \text{ s.t.}\\ \exists x \in X^I:\, y \le (p \lor q)(x)}} m(p(x) \land y) + m(q(x) \land y) \\
                  &= \sum_{\substack{\text{atoms }y \in \mathbf{B} \text{ s.t.}\\ \exists x \in X^I:\, y \le (p \lor q)(x)}} m(p(x) \land y) + \sum_{\substack{\text{atoms }y \in \mathbf{B} \text{ s.t.}\\ \exists x \in X^I:\, y \le (p \lor q)(x)}} m(q(x) \land y).
  \end{align*}
  Since $y$ is an atom,
  \[
    p(x) \land y  = \begin{cases}
      y & \text{if } y \le p(x) \\
      0 & \text{otherwise,}
    \end{cases}
  \]
  so
  \begin{align*}
    m^*(p \lor q) &= \sum_{\substack{\text{atoms }y \in \mathbf{B} \text{ s.t.}\\ \exists x \in X^I:\, y \le (p \lor q)(x) \text{ and } y \le p(x)}} m(p(x) \land y) + \sum_{\substack{\text{atoms }y \in \mathbf{B} \text{s.t.}\\ \exists x \in X^I:\, y \le (p \lor q)(x) \text{ and } y \le q(x)}} m(q(x) \land y) \\
                  &= \sum_{\substack{\text{atoms }y \in \mathbf{B} \text{ s.t.}\\ \exists x \in X^I:\, y \le p(x)}} m(p(x) \land y) + \sum_{\substack{\text{atoms }y \in \mathbf{B} \text{s.t.}\\ \exists x \in X^I:\, y \le q(x)}} m(q(x) \land y) \\
                  &= \sum_{\substack{\text{atoms }y \in \mathbf{B} \text{ s.t.}\\ \exists x \in X^I:\, y \le p(x)}} m(y) + \sum_{\substack{\text{atoms }y \in \mathbf{B} \text{s.t.}\\ \exists x \in X^I:\, y \le q(x)}} m(y) = m^*(p) + m^*(q),
  \end{align*}
  finishing the proof that $m^*$ is a measure.
\end{proof}

\begin{lemma} \label{lemma:simple_measure}
  Given the setup of \cref{prop:polyadic_measure} and $p \in \mathbf{A}$, if
  $p(x) = p(y)$ for all $x, y \in X^I$ (i.e., $p$ has no free variables), then
  \[
    m^*(p) = m(p(x))
  \]
  (for some $x \in X^I$) is an alternative (i.e., equivalent and simpler)
  definition of $m^*$.
\end{lemma}
\begin{proof}
  Fix some $x \in X^I$. Then
  \[
    m(p(x)) = m \left( \bigvee_{\substack{\text{atoms } y \in \mathbf{B} \text{
            s.t.}\\
          y \le p(x)}} y \right) = \sum_{\substack{\text{atoms } y \in \mathbf{B}
        \text{ s.t.}\\
        y \le p(x)}} m(y) =\sum_{\substack{\text{atoms } y \in \mathbf{B}
        \text{ s.t.}\\
        \exists x' \in X^I : y \le p(x')}} m(y) = m^*(p),
  \]
  where we use \cref{thm:representation} for the first step,
  \cref{def:measure} and \cref{lemma:atomic} for the second step, the
  assumptions of \cref{lemma:simple_measure} for the third step, and the
  definition of $m^*$ for the fourth one.
\end{proof}

\section{How Probabilities Are Computed}

\subsection{Preliminaries}

\begin{definition}[\cite{givant2008introduction}] \label{def:boolean_ideal}
  An \emph{ideal} in a Boolean algebra $\mathbf{B}$ is a subset $M \subseteq
  \mathbf{B}$ such that:
  \begin{itemize}
  \item $0 \in M$;
  \item $x \lor y \in M$ for all $x, y \in M$;
  \item $x \land y \in M$ for all $x \in M$ and $y \in \mathbf{B}$.
  \end{itemize}
  For any subset $S \subseteq \mathbf{B}$, the \emph{ideal generated by $S$} is
  the smallest ideal $M$ such that $S \subseteq M$.
\end{definition}

Note that \cref{def:boolean_ideal} gives us a simple characterisation of an
ideal generated by a set of atoms.

\begin{lemma}
  Let $\mathbf{B}$ be a Boolean algebra, and let $S \subseteq \mathbf{B}$ be a
  set of atoms. The ideal $I$ generated by $S$ is defined by the following:
  \begin{itemize}
  \item $0 \in I$,
  \item $S \subseteq I$,
  \item $x \lor y \in I$ for all $x, y \in I$.
  \end{itemize}
\end{lemma}

\begin{definition}[\cite{givant2008introduction}]
  Let $\mathbf{B}$ be a Boolean algebra, and let $I$ be an ideal in
  $\mathbf{B}$. The \emph{quotient algebra} $\mathbf{B}/I$ is a Boolean algebra
  on equivalence classes of elements of $\mathbf{B}$ (with operations defined
  pointwise) based on the equivalence relation
  \[
    x \sim y \quad \iff \quad x + y \in I
  \]
  where $x + y = (x \land \neg y) \lor (y \land \neg x)$ is the symmetric
  difference operation (written as a sum because it can interpreted as the
  `additive' part of the corresponding Boolean ring).
\end{definition}

\subsection{New Results (an Example)}

In order to make the example algebras easily describable, our example programs
will have to be tiny. Consider the following ProbLog
\cite{DBLP:conf/ijcai/RaedtKT07} program:
\begin{align*}
  1.0 &\dblcolon \mathsf{p}(a, b).\\
  0.5 &\dblcolon \mathsf{p}(X, X) \coloneq \mathsf{p}(X, Y);\, \mathsf{p}(Y, X).
\end{align*}
Let $L = \{ \mathsf{p}(a, a), \mathsf{p}(a, b), \mathsf{p}(b, a), \mathsf{p}(b,
b) \}$ be the set of all possible ground atoms. Let $\mathbf{B}$ be the
Boolean algebra freely generated by $L$ (see, e.g.,
\cite{givant2008introduction} for more on free Boolean algebras). Then
$\mathbf{B}$ will have sixteen atoms ranging from $\mathsf{p}(a, a) \land
\mathsf{p}(a, b) \land \mathsf{p}(b, a) \land \mathsf{p}(b, b)$ to
$\neg\mathsf{p}(a, a) \land \neg\mathsf{p}(a, b) \land \neg\mathsf{p}(b, a)
\land \neg\mathsf{p}(b, b)$. The weight function $w : L \to \mathbb{R}_{\ge 0}$
defined by
\[
  w(l) = \begin{cases}
    1 & \text{if } l = \mathsf{p}(a, b) \\
    0.5 & \text{if } l \in \{ \mathsf{p}(a, a), \mathsf{p}(b, b) \} \\
    0 & \text{if } l = \mathsf{p}(b, a) \\
    1-w(l') & \text{if } l = \neg l'
  \end{cases}
\]
for all $l \in L$ defines a WMC measure over $\mathbf{B}$. Note that while we
could define an ideal generated by $\{ \mathsf{p}(b, a), \neg\mathsf{p}(a, b)
\}$ and take the quotient of $\mathbf{B}$ by that ideal to get a Boolean algebra
with a strictly positive measure, this would put zero-probability queries
outside of our algebras, i.e., we would not be able to ask a question whose
answer is zero.

\begin{table}
  \centering
  \caption{Example elements of $\mathbf{A}$ as maps $X^I \to \mathbf{B}$, with
    $a : \mathcal{P}(I) \to \End(\mathbf{A})$ as one of two possible constants.}
  \label{tbl:examples}
  \begin{tabular}{ll}
    \toprule
    Element of $\mathbf{A}$ & Action on $X^I$ \\
    \midrule
    $p = \mathbf{S}(\id)p = \bm\exists(\emptyset)p = a(\emptyset)p = b(\emptyset)p$ & $(x_1, x_2) \mapsto \mathsf{p}(x_1, x_2)$ \\
    $\bm\exists(1)p$ & $(x_1, x_2) \mapsto \mathsf{p}(a, x_2) \lor \mathsf{p}(b, x_2)$ \\
    $\bm\exists(2)p$ & $(x_1, x_2) \mapsto \mathsf{p}(x_1, a) \lor \mathsf{p}(x_1, b)$ \\
    $\bm\exists(I)p$ & $(x_1, x_2) \mapsto \mathsf{p}(a, a) \lor \mathsf{p}(a, b) \lor \mathsf{p}(b, a) \lor \mathsf{p}(b, b)$ \\
    $\mathbf{S}(\{ 1 \mapsto 1, 2 \mapsto 1 \})p$ & $(x_1, x_2) \mapsto \mathsf{p}(x_1, x_1)$ \\
    $\mathbf{S}(\{ 1 \mapsto 2, 2 \mapsto 1 \})p$ & $(x_1, x_2) \mapsto \mathsf{p}(x_2, x_1)$ \\
    $\mathbf{S}(\{ 1 \mapsto 2, 2 \mapsto 2 \})p$ & $(x_1, x_2) \mapsto \mathsf{p}(x_2, x_2)$ \\
    $a(1)p$ & $(x_1, x_2) \mapsto \mathsf{p}(a, x_2)$ \\
    $a(2)p$ & $(x_1, x_2) \mapsto \mathsf{p}(x_1, a)$ \\
    $a(I)p$ & $(x_1, x_2) \mapsto \mathsf{p}(a, a)$ \\
    \bottomrule
  \end{tabular}
\end{table}

Finally, let $\mathbf{A}$ be the functional polyadic algebra $X^I \to
\mathbf{B}$ for $I = \{1, 2\}$ and $X = \{ a, b \}$\footnote{$X$ cannot (or
  should not) have constants that do not occur in $\mathbf{B}$.}. The elements
of $X^I$ can then be represented as tuples $(x_1, x_2)$ for some $x_1, x_2 \in
X$. See \cref{tbl:examples} for example elements of $\mathbf{A}$ which consists
of a single predicate function $p$ and operators $\bm\exists, \mathbf{S}, a, b,
\neg, \land, \lor$, the last three of which are defined pointwise.

\begin{table}
  \centering
  \caption{Step-by-step derivation of how a more complex element of
    $\mathbf{A}$ acts on elements of $X^I$}
  \label{tbl:derivation}
  \begin{tabular}{ll}
    \toprule
    Element of $\mathbf{A}$ & Action on $X^I$ \\
    \midrule
    $p$ & $(x_1, x_2) \mapsto \mathsf{p}(x_1, x_2)$ \\
    $b(2)p$ & $(x_1, x_2) \mapsto \mathsf{p}(x_1, b)$ \\
    $\neg b(2)p$ & $(x_1, x_2) \mapsto \neg\mathsf{p}(x_1, b)$ \\
    $\bm\exists(1)\neg b(2)p$ & $(x_1, x_2) \mapsto \neg\mathsf{p}(a, b) \lor \neg\mathsf{p}(b, b) = \neg(\mathsf{p}(a, b) \land \mathsf{p}(b, b))$ \\
    $\bm\forall(1)b(2)p = \neg\bm\exists(1)\neg b(2)p$ & $(x_1, x_2) \mapsto \neg\neg(\mathsf{p}(a, b) \land \mathsf{p}(b, b)) = \mathsf{p}(a, b) \land \mathsf{p}(b, b)$ \\
    \bottomrule
  \end{tabular}
\end{table}

\begin{table}
  \centering
  \caption{Atoms $y \in \mathbf{B}$ (and their measures) such that $y \le
    \mathsf{p}(a, b) \land \mathsf{p}(b, b)$}
  \label{tbl:atoms}
  \begin{tabular}{lc}
    \toprule
    Atom $y \in \mathbf{B}$ & $m(y)$ \\
    \midrule
    $\mathsf{p}(a, b) \land \mathsf{p}(b, b) \land \mathsf{p}(a, a) \land \mathsf{p}(b, a)$ & $1 \times 0.5 \times 0.5 \times 0 = 0$ \\
    $\mathsf{p}(a, b) \land \mathsf{p}(b, b) \land \neg\mathsf{p}(a, a) \land \mathsf{p}(b, a)$ & $1 \times 0.5 \times 0.5 \times 0 = 0$ \\
    $\mathsf{p}(a, b) \land \mathsf{p}(b, b) \land \mathsf{p}(a, a) \land \neg\mathsf{p}(b, a)$ & $1 \times 0.5 \times 0.5 \times 1 = 0.25$ \\
    $\mathsf{p}(a, b) \land \mathsf{p}(b, b) \land \neg\mathsf{p}(a, a) \land \neg\mathsf{p}(b, a)$ & $1 \times 0.5 \times 0.5 \times 1 = 0.25$ \\
    \bottomrule
  \end{tabular}
\end{table}

Let us calculate the probability $\Pr(\forall x_1 \in X, \mathsf{p}(x_1, b))$.
The same expression can be translated into the notation for our polyadic algebra
$\mathbf{A}$ as $m^*(\bm\forall(1)b(2)p)$. Recall that $\bm\forall(1)b(2)p =
\neg\bm\exists(1)\neg b(2)p$. The effect of this function on an arbitrary
element of $X^I$ is derived step-by-step in \cref{tbl:derivation}. Since the
resulting function is constant (i.e., the logical formula has no free
variables), \cref{lemma:simple_measure} tells us that
\[
  m^*(\bm\forall(1)b(2)p) = m(\mathsf{p}(a, b) \land \mathsf{p}(b, b)) = m
  \left( \bigvee_{\substack{\text{atoms } y \in \mathbf{B} \text{ s.t.}\\
        y \le \mathsf{p}(a, b) \land \mathsf{p}(b, b)}} y \right) = \sum
  _{\substack{\text{atoms } y \in \mathbf{B} \text{ s.t.}\\
      y \le \mathsf{p}(a, b) \land \mathsf{p}(b, b)}} m(y).
\]
The resulting sum is over four atoms; these atoms and their probabilities are
listed in \cref{tbl:atoms}. Thus, we get that
\[
  m^*(\bm\forall(1)b(2)p) = 0 + 0 + 0.25 + 0.25 = 0.5.
\]

\section{What Measures Are WMC-Computable?}

\subsection{WMC Requires Independent Literals}

% TODO: maybe I should gives this kind of a BA a name? A synonym of 'complete',
% perhaps.
% TODO: a special case for weight=0.

\begin{proposition}
  Let $\mathbf{B}$ be a finite measure algebra with measure $m : \mathbf{B} \to
  \mathbb{R}_{\ge 0}$. Let $L \subset \mathbf{B}$ be defined as
  \[
  L = \{ l_i \mid i \in [n] \} \cup \{ \neg l_i \mid i \in [n] \}
  \]
  for some $n \in \mathbb{N}$. Finally, assume that $\mathbf{B}$ has $2^n$
  atoms, where each atom $a \in \mathbf{B}$ is an infimum
  \[
    a = \bigwedge_{i=1}^n a_i
  \]
  such that $a_i \in \{ l_i, \neg l_i \}$ for $i \in [n]$. Then there exists a
  weight function $w : L \to \mathbb{R}_{>0}$ that makes $m$ a WMC measure if
  and only if
  \begin{equation} \label{eq:wmccondition}
  m(l \land l') = m(l)m(l')
  \end{equation}
  for all distinct $l, l' \in L$ such that $l \ne \neg l'$.
\end{proposition}

\begin{remark}
  Note that if $n = 1$, then \cref{eq:wmccondition} is vacuously satisfied and
  so any valid measure can be expressed as WMC.
\end{remark}

\begin{proof}
  Let us begin with the `if' part of the statement. Let $w : L \to
  \mathbb{R}_{>0}$ be defined by
  \begin{equation} \label{eq:assumption}
    w(l) = m(l)
  \end{equation}
  for all $l \in L$. We are going
  to show that $\nWMC = m$. First, note that $\nWMC(0) = 0 = m(0)$ by the
  definitions of both $\nWMC$ and $m$. Second, let
  \begin{equation} \label{eq:def_of_a}
    a = \bigwedge_{i=1}^n a_i
  \end{equation}
  be an atom in $\mathbf{B}$ such that $a_i \in \{ l_i, \neg l_i \}$ for all $i
  \in [n]$. Then
  \[
    \nWMC(a) = \frac{\WMC(a)}{\WMC(1)} = \frac{1}{\WMC(1)} \prod_{i=1}^n w(a_i)
    = \frac{1}{\WMC(1)} \prod_{i=1}^n m(a_i) = \frac{1}{\WMC(1)} m \left(
      \bigwedge_{i=1}^n a_i \right) = \frac{m(a)}{\WMC(1)}
  \]
  by \cref{def:wmc,eq:assumption,eq:wmccondition,eq:def_of_a}. Now we just need
  to show that $\WMC(1) = 1$. Indeed,
  \begin{align*}
    \WMC(1) &= \sum_{\text{atoms } a \in \mathbf{B}} \WMC(a) = \sum_{\text{atoms
      } a \in \mathbf{B}} \prod_{i=1}^n w(a_i) = \sum_{\text{atoms } a \in
      \mathbf{B}} \prod_{i=1}^n m(a_i) \\
    &= \sum_{\text{atoms } a \in
      \mathbf{B}} m \left( \bigwedge_{i=1}^n a_i \right) = \sum_{\text{atoms } a
      \in \mathbf{B}} m(a) = m \left( \bigvee_{\text{atoms } a \in \mathbf{B}}
    \right) = m(1) = 1.
  \end{align*}
  Finally, note that if $\nWMC$ and $m$ agree on all atoms, then they must also
  agree on all other non-zero elements of the Boolean algebra.

  For the other direction, we are given a weight function $w : L \to
  \mathbb{R}_{>0}$ that induces a measure $m = \nWMC : \mathbf{B} \to
  \mathbb{R}_{\ge 0}$, and we want to show that \cref{eq:wmccondition} is
  satisfied. Let $k_i, k_j \in L$ be such that $k_i \in \{ l_i, \neg l_i \}$,
  $k_j \in \{ l_j, \neg l_j \}$, and $i \ne j$. We will first prove an auxiliary
  result that
  \begin{equation} \label{eq:to_prove}
    m(k_i \land k_j) = m(k_i)m(k_j)
  \end{equation}
  is equivalent to
  \begin{equation} \label{eq:to_prove2}
    m(k_i \land k_j) \cdot m(\neg k_i \land \neg k_j) = m(k_i \land \neg k_j)
    \cdot m(\neg k_i \land k_j).
  \end{equation}
  First, note that $k_i$ can be expressed as
  \[
    k_i = (k_i \land k_j) \lor (k_i \land \neg k_j)
  \]
  with the condition that
  \[
    (k_i \land k_j) \land (k_i \land \neg k_j) = 0,
  \]
  so, by properties of a measure,
  \begin{equation} \label{eq:temp}
    m(k_i) = m(k_i \land k_j) + m(k_i \land \neg k_j).
  \end{equation}
  Applying \cref{eq:temp} and the equivalent expression for $m(k_j)$ allows us
  to rewrite \cref{eq:to_prove} as
  \begin{align*}
    m(k_i \land k_j) &= [m(k_i \land k_j) + m(k_i \land \neg k_j)] \cdot [m(k_i \land k_j) + m(\neg k_i \land k_j)] \\
                     &= m(k_i \land k_j)^2 + m(k_i \land k_j)[m(k_i \land \neg k_j) + m(\neg k_i \land k_j)] + m(k_i \land \neg k_j)m(\neg k_i \land k_j)
  \end{align*}
  Dividing both sides by $m(k_i \land k_j)$ gives
  \begin{equation} \label{eq:temp2}
    1 = m(k_i \land k_j) + m(k_i \land \neg k_j) + m(\neg k_i \land k_j) +
    \frac{m(k_i \land \neg k_j)m(\neg k_i \land k_j)}{m(k_i \land k_j)}.
  \end{equation}
  Since $k_i \land k_j \land k_i \land \neg k_j = 0$, and
  \[
    (k_i \land k_j) \lor (k_i \land \neg k_j) = k_i \land (k_j \lor \neg k_j) =
    k_i \land 1 = k_i,
  \]
  we have that
  \[
    m(k_i \land k_j) + m(k_i \land \neg k_j) = m(k_i).
  \]
  Similarly, $k_i \land \neg k_i \land k_j = 0$, and
  \[
    k_i \lor (\neg k_i \land k_j) = (k_i \lor \neg k_i) \land (k_i \lor k_j) =
    k_i \lor k_j,
  \]
  so
  \[
    m(k_i) + m(\neg k_i \land k_j) = m(k_i \lor k_j).
  \]
  Finally, note that
  \[
    (k_i \lor k_j) \land \neg(k_i \lor k_j) = 0,
  \]
  and
  \[
    (k_i \lor k_j) \lor \neg(k_i \lor k_j) = 1,
  \]
  so
  \[
    m(k_i \lor k_j) + m(\neg(k_i \lor k_j)) = m(1) = 1.
  \]
  This allows us to rewrite \cref{eq:temp2} as
  \[
    \frac{m(k_i \land \neg k_j)m(\neg k_i \land k_j)}{m(k_i \land k_j)} = 1 -
    m(k_i \lor k_j) = m(\neg(k_i \lor k_j)) = m(\neg k_i \land \neg k_j)
  \]
  which immediately gives us \cref{eq:to_prove2}.

  Now recall that $m = \nWMC$ and note that \cref{eq:to_prove2} can be
  multiplied by $\WMC(1)^2$ to turn the equation into one for $\WMC$ instead of
  $\nWMC$. Then
  \begin{align*}
    \WMC(k_i \land k_j) &= \sum_{\text{atoms } a \le k_i \land k_j} \WMC(a) = \sum_{\text{atoms } a \le k_i \land k_j} \prod_{m \in [n]} w(a_m) \\
                        &= \sum_{\text{atoms } a \le k_i \land k_j} w(a_i)w(a_j) \prod_{m \in [n] \setminus \{ i, j \}} w(a_m) = \sum_{\text{atoms } a \le k_i \land k_j} w(k_i)w(k_j) \prod_{m \in [n] \setminus \{ i, j \}} w(a_m) \\
    &= w(k_i)w(k_j) \sum_{\text{atoms } a \le k_i \land k_j} \prod_{m \in [n] \setminus \{ i, j \}} w(a_m) = w(k_i)w(k_j)C,
  \end{align*}
  where $C$ denotes the part of $\WMC(k_i \land k_j)$ that will be the same for
  $\WMC(\neg k_i \land k_j)$, $\WMC(k_i \land \neg k_j)$, and $\WMC(\neg k_i
  \land \neg k_j)$ as well. But then \cref{eq:to_prove2} becomes
  \[
    w(k_i)w(k_j)w(\neg k_i)w(\neg k_j)C^2 = w(k_i)w(\neg k_j)w(\neg k_i)w(k_j)C^2
  \]
  which is trivially true. By showing that WMC satisfies \cref{eq:to_prove2}, we
  also showed that it satisfies \cref{eq:to_prove}, finishing the second part of
  the proof.
\end{proof}
% TODO: the auxiliary result should be a 'claim' in-between the theorem and the proof.

\subsection{Extending the Algebra}

\subsubsection{Preliminaries}

\begin{definition}[\cite{givant2008introduction}]
  Let $\mathbf{A}$ and $\mathbf{B}$ be Boolean algebras. A \emph{Boolean
    homomorphism} from $\mathbf{A}$ to $\mathbf{B}$ is a map $f : \mathbf{A} \to
  \mathbf{B}$ such that:
  \begin{itemize}
  \item $f(x \land y) = f(x) \land f(y)$,
  \item $f(x \lor y) = f(x) \lor f(y)$,
  \item $f(\neg x) = \neg f(x)$
  \end{itemize}
  for all $x, y \in \mathbf{A}$.
\end{definition}

\begin{definition}[\cite{halmos2016algebraic}]
  Given two polyadic algebras $\mathbf{A}$ and $\mathbf{B}$, a \emph{polyadic
    homomorphism} from $\mathbf{A}$ to $\mathbf{B}$ is a Boolean homomorphism $f
  : \mathbf{A} \to \mathbf{B}$ such that
  \begin{itemize}
  \item $f\mathbf{S}(\tau)p = \mathbf{S}(\tau)fp$,
  \item $f\bm\exists(J)p = \bm\exists(J)fp$
  \end{itemize}
  for all $\tau \in T$, $p \in \mathbf{A}$, and $J \subseteq I$.
\end{definition}

\subsubsection{New Results}

A well-known way to overcome this limitation of independence is by adding more
literals \cite{DBLP:journals/ai/ChaviraD08}, i.e., extending the set $L$ covered
by the WMC weight function $w : L \to \mathbb{R}_{>0}$. Let us translate this
idea to the language of Boolean algebras.

\begin{theorem} \label{thm:extension} % TODO: cite the fact about atoms
  Let $\mathbf{B}$ be a finite Boolean algebra freely generated by some set of
  `literals' $L$, and let $m : \mathbf{B} \to \mathbb{R}_{\ge 0}$ be an
  arbitrary measure. We know that $\mathbf{B}$ has $n = 2^{|L|}$ atoms. Let
  $(a_i)_{i=1}^n$ be those atoms. Let $L' = L \cup \{ \phi_i \mid i \in [n] \}$
  be the set $L$ extended with $n$ new literals. Let $\mathbf{B'} =
  \mathbf{C}/I$ be the quotient of the Boolean algebra $\mathbf{C}$ freely
  generated by $L'$ and the ideal $I$ generated by
  \[
    \{\phi_i \land \phi_j \mid i, j \in [n], i \ne j \} \cup \{ a_i \land \phi_j
    \mid i, j \in [n], i \ne j \}.
  \]
  Let $h : \mathbf{B} \to \mathbf{B'}$ be the Boolean homomorphism defined by
  $h(l) = l$ for all $l \in L$ (note that this indeed defines a homomorphism).
  Let $w : L' \to \mathbb{R}_{> 0}$ be defined by
  \[
    w(l) = \begin{cases}
      m(a_i) & \text{if } l = \phi_i \text{ for some } i \in [n] \\
      1 & \text{otherwise}
    \end{cases}
  \]
  for all $l \in L'$. This defines a WMC measure $m' : \mathbf{B'} \to
  \mathbb{R}_{\ge 0}$. Furthermore,
  \[
    m(a) = m'(h(a)).
  \]
\end{theorem}

In simpler terms, any measure can be computed using WMC by extending the Boolean
algebra with more literals and connecting the two via a homomorphism. More
precisely, we are given the red part in
\[
  \begin{tikzcd}
    & \textcolor{red}{\mathbb{R}_{\ge 0}} & & \\
    \textcolor{red}{\mathbf{B}} \arrow[red]{ur}{m} \arrow{rr}{h} & & \mathbf{B'}
    \arrow{ul}[swap]{m'} & \\
    \textcolor{red}{L} \arrow[Subset,red]{u}{} \arrow[Subset]{rr}{} & & L'
    \arrow[Subset]{u}{} \arrow{r}{w} & \mathbb{R}_{>0}
  \end{tikzcd}
\]
and construct the black part in such a way that the triangle commutes.

\begin{proof}
  % TODO
\end{proof}

% TODO: proof that n is the smallest number (in the worst case, without
% independence assumptions)
%\begin{proposition}
%  In \cref{thm:extension}, if $L'$ has less than $n$ additional elements, then
%  there exists a measure $$
%\end{proposition}

% TODO: describe how my approach is better in the worst case than that by
% Darwiche

\todo[inline,caption={}]{The big TODO list
  \begin{itemize}
  \item Extension to infinite (atomic?) Boolean algebras.
  \item Compare my polyadic measures with first-order WMC.
  \item Abstraction refinements as homomorphisms.
  \item Definition of a measure-preserving homomorphism from Jech's set theory
    book.
  \item A Boolean algebra is approximable if its Stone space is approximable.
  \end{itemize}
}

\bibliographystyle{plain}
\bibliography{paper}

\end{document}
