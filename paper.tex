\documentclass{article}
\usepackage[utf8]{inputenc}
\usepackage[UKenglish]{babel}
\usepackage[UKenglish]{isodate}
\usepackage{fullpage}
\usepackage{amsthm}
\usepackage{amsfonts}
\usepackage{amsmath}
\usepackage{mathtools}
\usepackage[capitalise]{cleveref}
\usepackage{bm}
\usepackage{booktabs}
\usepackage{tikz}
\usepackage{xcolor}
\usepackage[backgroundcolor=lightgray]{todonotes}

\newtheorem{theorem}{Theorem}
\newtheorem{lemma}{Lemma}
\newtheorem{proposition}{Proposition}
\newtheorem{corollary}{Corollary}
\newtheorem{conjecture}{Conjecture}
\theoremstyle{definition}
\newtheorem{definition}{Definition}
\newtheorem{example}{Example}
\theoremstyle{remark}
\newtheorem*{remark}{Remark}

\definecolor{color1}{HTML}{fbb4ae}
\definecolor{color2}{HTML}{b3cde3}
\definecolor{color3}{HTML}{ccebc5}
\definecolor{color4}{HTML}{decbe4}

\Crefname{property}{Property}{Properties}
\Crefname{condition}{Condition}{Conditions}
\creflabelformat{condition}{#2(#1)#3}

\DeclareMathOperator{\WMC}{WMC}
\DeclareMathOperator{\nWMC}{NWMC}
\DeclareMathOperator{\id}{id}
\DeclareMathOperator{\End}{End}

\usetikzlibrary{cd}

\tikzset{
  Subset/.style={
    draw=none,
    every to/.append style={
      edge node={node [sloped, allow upside down, auto=false]{$\subset$}}}
  }
}

\title{On the Limitations of Weighted Model Counting}
%\title{Weighted Model Counting/Integration from the Perspective of Boolean Algebras}
%\title{What Boolean Algebras Can Teach Us About Weighted Model Counting/Integration?}
\author{Paulius Dilkas}

\begin{document}
\maketitle

\section{Introduction}
% Feedback: What are the main claims, what are the main takeaways, intuitive
% [???] of theorems to follow. To do this, we appeal to algebraic constructions
% to define the main concepts for introducing measures on Boolean algebras.

% Contributions:
% 1. A generalisation of WMC adapted to Boolean algebras.
% 2. We show that it is a valid measure.
% 3. We show that WMC assumes independence, i.e., cannot represent all measures.
% 4. We show how the BA can be extended with new 'literals' to represent any measure.
% 5. (Maybe) we establish a lower bound on the number of new literals, if we
% make no assumptions about independence.
% 6. show how evidence = quotiening by a filter, and how quotiening by an ideal works
% 7. (WMI) If all weights are 1, we extend Boolean variables to discrete
% variables, and continuous variables are arbitrary subsets (in the
% sigma-algebra), then the WMI process calculates the probability of any event
% and doesn't require going down to the level of models (assuming a suitable
% probability measure).
% 8. If the weight of a literal (and its negation) is 1, then we don't need to
% look at its models--the probability can be calculated at any level. This
% extends to any number of variables.
% 9. How WMI transforms to our algebras. The boring version and what's wrong
% with it.
% 10. Show how to transfer the measure to a quotient algebra and that the answer
% match with each other and with the original definition of WMC.

Potential directions to explore (no good ones, really):
\begin{itemize}
\item Infinite BAs.
  \begin{itemize}
  \item For example, the BA of finite and cofinite sets could be interesting.
  \item OUWMC requires infinite logics to be compact (whatever that means). My
    algebraic angle suggests that completeness should be enough. Topologically,
    compactness implies completeness, but I have no idea how this translates to
    logics and algebras.
  \end{itemize}
\item WMI
\item Measures with something other than $\mathbb{R}_{\ge 0}$ as the codomain
  (doesn't look promising).
\end{itemize}

Previous/related work:
\begin{itemize}
\item Hailperin's approach to probability logic
  \cite{DBLP:journals/ndjfl/Hailperin84}
\item Nilsson's (somewhat successful) probabilistic logic
  \cite{DBLP:journals/ai/Nilsson86}
\item Logical induction: a big paper with a good overview of previous attempts
  to assign probabilities to logical sentences in a sensible way
  \cite{DBLP:journals/eccc/GarrabrantBCST16}
\item Semiring programming \cite{DBLP:journals/corr/BelleR16}
\item WMI \cite{DBLP:conf/ijcai/BellePB15}
\item Measures on Boolean algebras
  \begin{itemize}
  \item On possibility and probability measures in finite Boolean algebras
    \cite{DBLP:journals/soco/CastineiraCT02}
  \item Representation of conditional probability measures
    \cite{krauss1968representation}
  \end{itemize}
\end{itemize}

\section{Preliminaries}

\begin{definition} \label{def:ba}
  A \emph{Boolean algebra} (BA) is a tuple $(\mathbf{B}, \land, \lor, \neg, 0,
  1)$ consisting of a set $\mathbf{B}$ with binary operations \emph{meet}
  $\land$ and \emph{join} $\lor$, unary operation $\neg$ and elements $0, 1 \in
  \mathbf{B}$ such that the following axioms hold for all $a, b, \in
  \mathbf{B}$:
  \begin{itemize}
  \item both $\land$ and $\lor$ are associative and commutative;
  \item $a \lor (a \land b) = a$, and $a \land (a \lor b) = a$;
  \item $0$ is the identity of $\lor$, and $1$ is the identity of $\land$;
  \item $\lor$ distributes over $\land$ and vice versa;
  \item $a \lor \neg a = 1$, and $a \land \neg a = 0$.
  \end{itemize}
\end{definition}

For clarity and succinctness, we will occasionally use three other operations
that can be defined using the original three\footnote{We use $+$ to denote
  symmetric difference because it is the additive operation of a Boolean ring.}:
\begin{align*}
  a \to b &= \neg a \lor b, \\
  a \leftrightarrow b &= (a \land b) \lor (\neg a \land \neg b), \\
  a + b &= (a \land \neg b) \lor (\neg a \land b).
\end{align*}
We can also define a partial order $\le$ on $\mathbf{B}$ as $a \le b$ if $a = b
\land a$ (or, equivalently, $a \lor b = b$) for $a, b \in \mathbf{B}$.
Furthermore, let $a < b$ denote $a \le b$ and $a \ne b$. For the rest of this
paper, let $\mathbf{B}$ refer to the BA $(\mathbf{B}, \land, \lor, \neg, 0, 1)$.
For any $S \subseteq \mathbf{B}$, we write $\bigvee S$ for $\bigvee_{x \in S} x$
and call it the \emph{supremum} of $S$. Similarly, $\bigwedge S = \bigwedge_{x
  \in S} x$ is the \emph{infimum}. By convention, $\bigwedge \emptyset = 1$ and
$\bigvee \emptyset = 0$. For any $a, b \in \mathbf{B}$, we say that $a$ and $b$
are \emph{disjoint} if $a \land b = 0$.

\begin{definition}[\cite{DBLP:books/daglib/0090259,levasseur2012applied}]
  An element $a \ne 0$ of $\mathbf{B}$ is an \emph{atom} if, for all $x \in
  \mathbf{B}$, either $x \land a = a$ or $x \land a = 0$. Equivalently, $a \ne
  0$ is an atom if there is no $x \in \mathbf{B}$ such that $0 < x < a$. We say
  that $\mathbf{B}$ is \emph{atomic} if for every $a \in \mathbf{B} \setminus \{0
  \}$, there is an atom $x$ such that $x \le a$.
\end{definition}

\begin{lemma}[\cite{ganesh2006introduction}]
  For any two distinct atoms $a$, $b \in \mathbf{B}$, $a \land b = 0$.
\end{lemma}

\begin{lemma}[\cite{givant2008introduction}] \label{thm:representation}
  The following are equivalent:
  \begin{itemize}
  \item $\mathbf{B}$ is atomic.
  \item For any $x \in \mathbf{B}$,
    \[
      x = \bigvee_{\text{atoms } a \le x} a.
    \]
  \item $1$ is the supremum of all atoms.
  \end{itemize}
\end{lemma}

\begin{lemma}[\cite{givant2008introduction}] \label{lemma:atomic}
  All finite BAs are atomic.
\end{lemma}

\begin{definition}[\cite{gaifman1964concerning,DBLP:books/daglib/0090259}] \label{def:measure}
  A \emph{measure} on $\mathbf{B}$ is a function $m\colon
  \mathbf{B} \to \mathbb{R}_{\ge 0}$ such that:
  \begin{itemize}
  \item $m(0) = 0$;
  \item $m(a \lor b) = m(a) + m(b)$ for all $a, b \in \mathbf{B}$ whenever $a
    \land b = 0$.
  \end{itemize}
  If $m(1) = 1$, we call $m$ a \emph{probability measure}. Also, if $m(x) > 0$
  for all $x \ne 0$, then $m$ is \emph{strictly positive}.
\end{definition}

\begin{lemma}[\cite{horn1948measures}] \label{lemma:m_and_order}
  Let $m: \mathbf{B} \to \mathbb{R}_{\ge 0}$ be a measure. For any $a, b \in
  \mathbf{B}$, if $a \le b$, then $m(a) \le m(b)$.
\end{lemma}

%\begin{definition}[\cite{givant2008introduction}] \label{def:filter}
%  A \emph{(Boolean) filter} is a non-empty subset $F \subseteq \mathbf{B}$ such
%  that
%  \begin{itemize}
%  \item $f \land g \in F$ for all $f, g \in F$,
%  \item $f \lor a \in F$ for all $f \in F$ and $a \in \mathbf{B}$.
%  \end{itemize}
%  For any $p \in \mathbf{B}$, the \emph{principal filter of $p$} is the
%  intersection $I$ of all filters that contain $p$. It also has the property
%  that $p \le a$ for all $a \in I$.
%\end{definition}
%\begin{definition}[\cite{sikorski1969boolean}] \label{def:quotient}
%  Let $F \subseteq \mathbf{B}$ be a filter. The \emph{quotient} $\mathbf{B}/F$
%  is a BA of equivalence classes of elements of $\mathbf{B}$ with respect to the
%  equivalence relation
%  \[
%    a \sim b \qquad \iff \qquad a \to b \in F \quad \text{and} \quad b \to a \in F
%  \]
%  for all $a, b \in \mathbf{B}$. Elements of $\mathbf{B}/F$ are usually denoted
%  by $a/F$ (for some $a \in \mathbf{B}$) with the understanding that if $b \sim
%  a$ (for some $b \in \mathbf{B}$), then $b/F = a/F$.
%\end{definition}

\begin{definition}[\cite{givant2008introduction}] \label{def:ideal}
  An \emph{ideal} is a non-empty subset $I \subseteq \mathbf{B}$ such that
  \begin{itemize}
  \item $i \lor j \in I$ for all $i, j \in I$;
  \item $i \land a \in I$ for all $i \in I$ and $a \in \mathbf{B}$.
  \end{itemize}
  For any $p \in \mathbf{B}$, the \emph{principal ideal of $p$}---denoted by
  $(p)$---is the smallest ideal that contains $p$. It can also be expressed as
  $(p) = \{ a \in \mathbf{B} \mid a \le p \}$.
\end{definition}

\begin{definition}[\cite{givant2008introduction}] \label{def:quotient}
  Let $I$ be an ideal in $\mathbf{B}$. The \emph{quotient algebra of
    $\mathbf{B}$ modulo the ideal $I$} $\mathbf{B}/I$ is a BA
  of equivalence classes of elements of $\mathbf{B}$ with respect to the
  equivalence relation
  \[
    a \sim b \quad \iff \quad a + b \in I
  \]
  for all $a, b \in \mathbf{B}$. Elements of $\mathbf{B}/I$ are usually denoted
  by $a/I$ (for some $a \in \mathbf{B}$) with the understanding that if $b \sim
  a$ (for some $b \in \mathbf{B}$), then $b/I = a/I$. The three algebraic
  operations on $\mathbf{B}/I$ are defined as
  \begin{align*}
    a/I \land b/I &= a \land b / I, \\
    a/I \lor b/I &= a \lor b / I, \\
    \neg(a/I) &= (\neg a)/I.
  \end{align*}
\end{definition}

\begin{definition}[\cite{givant2008introduction}]
  Let $\mathbf{A}$ and $\mathbf{B}$ be BAs. A \emph{(Boolean) homomorphism} from
  $\mathbf{A}$ to $\mathbf{B}$ is a map $f\colon \mathbf{A} \to \mathbf{B}$ such
  that:
  \begin{itemize}
  \item $f(x \land y) = f(x) \land f(y)$,
  \item $f(x \lor y) = f(x) \lor f(y)$,
  \item $f(\neg x) = \neg f(x)$
  \end{itemize}
  for all $x, y \in \mathbf{A}$.
\end{definition}

\begin{lemma}[\cite{givant2008introduction}] \label{lemma:canonical_homomorphism}
  Let $I \subseteq \mathbf{B}$ be an ideal. The map $f\colon \mathbf{B} \to
  \mathbf{B}/I$ defined by $f(x) = x/I$ is a homomorphism.
\end{lemma}

\begin{lemma}[Homomorphisms preserve order
  \cite{givant2008introduction}] \label{lemma:homomorphisms_and_order}
  Let $f\colon \mathbf{A} \to \mathbf{B}$ be a homomorphism between two BAs
  $\mathbf{A}$ and $\mathbf{B}$. Then, for any $x, y \in \mathbf{A}$, if $x \le
  y$, then $f(x) \le f(y)$.
\end{lemma}

\begin{lemma}[\cite{sikorski1969boolean}] \label{lemma:order}
  For any $a, b \in \mathbf{B}$, $a \le b$ if and only if $a \land \neg b = 0$.
\end{lemma}

\begin{lemma}[\cite{givant2008introduction}] \label{lemma:measure_and_order}
  Let $m\colon \mathbf{B} \to \mathbb{R}_{\ge 0}$ be a measure. Then for all $a,
  b \in \mathbf{B}$, if $a \le b$, then $m(a) \le m(b)$.
\end{lemma}

% TODO: make up my mind about a,b vs x,y and stick to it (maybe x,y?)
% TODO: perhaps reorder things into paragraphs, i.e., a paragraph for order, for
% homomorphisms, for ideals and quotients. This would take up less space.

\section{WMC as a Measure}

\begin{definition} \label{def:algebra_from_logic}
  Let $\mathcal{L}$ be a propositional (or first-order) logic, and let
  $\Delta$ be a theory in $\mathcal{L}$. We can define an equivalence
  relation on formulas in $\mathcal{L}$ as
  \[
    \alpha \sim \beta \quad \text{if and only if} \quad \Delta \vdash \alpha
    \leftrightarrow \beta
  \]
  for all $\alpha, \beta \in \mathcal{L}$. Let $[\alpha]$ denote the equivalence
  class of $\alpha \in \mathcal{L}$ with respect to $\sim$. We can then let
  $B(\Delta) = \{ [\alpha] \mid \alpha \in \mathcal{L} \}$ and define the
  structure of a BA on $B(\Delta)$ as
  \begin{align*}
    [\alpha] \lor [\beta] &= [\alpha \lor \beta], \\
    [\alpha] \land [\beta] &= [\alpha \land \beta], \\
    \neg[\alpha] &= [\neg\alpha], \\
    1 &= [\alpha \to \alpha], \\
    0 &= [\alpha \land \neg\alpha]
  \end{align*}
  for all $\alpha, \beta \in \mathcal{L}$. Then $B(\Delta)$ is the
  \emph{Lindenbaum-Tarski algebra} of $\Delta$
  \cite{koppelberg1989handbook,tarski1983logic}.
\end{definition}

\begin{figure}
  \[
    \begin{tikzcd}
      & & & & & \colorbox{color3}{$1$} \ar[dlll,dash,gray] \ar[dl,dash,gray]
      \ar[dr,dash,gray] \ar[drrr,dash,gray] & & & \\
      & & \colorbox{color3}{$p \lor q$} & & \colorbox{color3}{$q \to p$} & &
      \colorbox{color2}{$p \to q$} & & \colorbox{color4}{$\neg p \lor \neg q$}
      \\
      \colorbox{color3}{$p$} \ar[urr,dash,gray] \ar[urrrr,dash,gray] & &
      \colorbox{color2}{$q$} \ar[u,dash,gray] \ar[urrrr,dash,gray] & &
      \colorbox{color2}{$p \leftrightarrow q$} \ar[u,dash,gray]
      \ar[urr,dash,gray] & & \colorbox{color4}{$p + q$} \ar[ullll,dash,gray]
      \ar[urr,dash,gray] & & \colorbox{color4}{$\neg q$} \ar[ullll,dash,gray]
      \ar[u,dash,gray] & & \colorbox{color1}{$\neg p$} \ar[ullll,dash,gray]
      \ar[ull,dash,gray] \\
      & & \fcolorbox{black}{color2}{$p \land q$} \ar[ull,dash,gray]
      \ar[u,dash,gray] \ar[urr,dash,gray] & & \fcolorbox{black}{color4}{$p \land
        \neg q$} \ar[ullll,dash,gray] \ar[urr,dash,gray] \ar[urrrr,dash,gray] &
      & \fcolorbox{black}{color1}{$\neg p \land q$} \ar[ullll,dash,gray]
      \ar[u,dash,gray] \ar[urrrr,dash,gray] & & \fcolorbox{black}{color1}{$\neg
        p \land \neg q$} \ar[ullll,dash,gray] \ar[u,dash,gray]
      \ar[urr,dash,gray] \\
      & & & & & \colorbox{color1}{$0$} \ar[ulll,dash,gray] \ar[ul,dash,gray]
      \ar[ur,dash,gray] \ar[urrr,dash,gray] & & &
    \end{tikzcd}
  \]
  \[
    \begin{tikzcd}
      & \colorbox{color3}{$\left[1\right]$} \ar[dl,dash,gray] \ar[dr,dash,gray]
      & \\
      \fcolorbox{black}{color2}{$\left[q\right]$} & &
      \fcolorbox{black}{color4}{$\left[\neg q\right]$} \\
      & \colorbox{color1}{$\left[0\right]$} \ar[ul,dash,gray] \ar[ur,dash,gray]
      &
    \end{tikzcd}
  \]
  \caption{Two BAs from \cref{example:construction}: $B(\mathcal{L})$ at the top
    and $B(\Delta)$ at the bottom. An edge between elements $a$ and $b$ (with
    $a$ positioned lower than $b$) means that $a < b$. Each element of
    $B(\Delta)$ is an equivalence class of elements of $B(\mathcal{L})$, and the
    colours show which elements of $B(\mathcal{L})$ belong to which class. In
    both algebras, atoms have borders around them.}
  \label{fig:example}
\end{figure}

\begin{example} \label{example:construction}
  Let $\mathcal{L}$ be a propositional logic with $p$ and $q$ as its only atoms.
  Then $L = \{ p, q, \neg p, \neg q \}$ is its set of literals. Let $w : L \to
  \mathbb{R}_{\ge 0}$ be the \emph{weight function} defined by
  \begin{align*}
    w(p) = 0.3, \\
    w(\neg p) = 0.7, \\
    w(q) = 0.2, \\
    w(\neg q) = 0.8.
  \end{align*}
  Let $\Delta$ be a theory in $\mathcal{L}$ with a sole axiom $p$. Then
  $\Delta$ has two models, i.e., $\{ p, q \}$ and $\{ p, \neg q \}$. The
  \emph{weighted model count} (WMC) \cite{DBLP:journals/ai/ChaviraD08} of $\Delta$ is
  then
  \[
    \sum_{\omega \models \Delta} \prod_{\omega \models l} w(l) =
    w(p)w(q) + w(p)w(\neg q) = 0.3.
  \]

  The corresponding BA $B(\Delta)$ can then be constructed using
  \cref{def:algebra_from_logic}. Alternatively, one can first construct the free
  BA generated by the set $\{ p, q \}$---this corresponds to $B(\mathcal{L})$ in
  \cref{fig:example}---and then take a quotient with respect to either the
  filter generated by $p$ or the ideal\footnote{More details on these concepts
    can be found in many books on BAs
    \cite{givant2008introduction,koppelberg1989handbook}.} generated by $\neg
  p$. In any case, the resulting BA is pictured at the bottom of
  \cref{fig:example}.

  Each element of $B(\mathcal{L})$ can also be seen as a subset of the set of
  all models of $\mathcal{L}$, with $0$ representing $\emptyset$, $1$
  representing the set of all (four) models, each atom representing a single
  model, and each edge going upward representing a subset relation. Thus,
  the Boolean-algebraic way of calculating the WMC of $\Delta$ consists of:
  \begin{enumerate} % TODO: why is step 1 always possible?
  \item Identifying an element $a \in B(\mathcal{L})$ that corresponds to
    $\Delta$.
  \item Finding all atoms of $B(\mathcal{L})$ that are `dominated' by $a$
    according to the partial order.
  \item Using $w$ to calculate the weight of each such atom.
  \item Adding the weights of these atoms.
  \end{enumerate}
  This motivates the following definition of WMC generalised to BAs.
\end{example}
% TODO: how to compute the number of elements in the algebra.
% TODO: clarify what B(L) means. And whether B(Delta) is even necessary.
% TODO: reference for the set/subset thing.

\todo[inline]{This should be replaced with inner sums (a.k.a. free products)}
\begin{definition} \label{def:wmc}
  Let $\mathbf{B}$ be an atomic BA, and let $M \subset \mathbf{B}$ be its set of
  atoms. Let $L \subset \mathbf{B}$ be such that every atom $m \in M$ can be
  uniquely expressed as $m = \bigwedge L'$ for some $L' \subseteq L$, and let
  $w\colon L \to \mathbb{R}_{\ge 0}$ be arbitrary. The \emph{weighted model
    count} $\WMC_w\colon \mathbf{B} \to \mathbb{R}_{\ge 0}$ is defined as
  \[
    \WMC_w(x) = \begin{cases}
      0 & \text{if } x = 0 \\
      \prod_{l \in L'} w(l) & \text{if } M \ni x = \bigwedge L' \\
      \sum_{\text{atoms } a \le x} \WMC_w(a) & \text{otherwise}
    \end{cases}
  \]
  for any $x \in \mathbf{B}$. Furthermore, we define the \emph{normalised
    weighted model count} $\nWMC_w\colon \mathbf{B} \to [0, 1]$ as $\nWMC_w(x) =
  \frac{\WMC_w(x)}{\WMC_w(1)}$ for all $x \in \mathbf{B}$. For both $\WMC_w$ and
  $\nWMC_w$, we will drop the subscript when doing so results in no potential
  confusion.
%  Finally, we say that a measure $m\colon \mathbf{B} \to
%  \mathbb{R}_{\ge 0}$ is a \emph{WMC measure} (or is \emph{WMC-computable}) if
%  there exists a subset $L \subset \mathbf{B}$ and a weight function $w\colon L
%  \to \mathbb{R}_{\ge 0}$ such that $m = \WMC_w$.
\end{definition}
% TODO: mention that the definition can be reduced to a single formula (i.e.,
% without cases)
% TODO: any measure is a WMC measure if all atoms are in L

\begin{proposition}
  $\WMC$ is a measure, and $\nWMC$ is a probability measure.
\end{proposition}
\begin{proof}
  First, note that $\WMC$ is non-negative and $\WMC(0) = 0$ by definition. Next,
  let $x, y \in \mathbf{B}$ be such that $x \land y = 0$. We want to show that
  \begin{equation} \label{eq:additivity_proof}
    \WMC(x \lor y) = \WMC(x) + \WMC(y).
  \end{equation}
  If, say, $x = 0$, then \cref{eq:additivity_proof} becomes
  \[
    \WMC(y) = \WMC(0) + \WMC(y) = \WMC(y)
  \]
  (and likewise for $y = 0$). Thus we can assume that $x \ne 0 \ne y$ and use
  \cref{thm:representation} to write
  \[
    x = \bigvee_{i \in I} x_i \quad \text{and} \quad y = \bigvee_{j \in J} y_j
  \]
  for some sequences of atoms $(x_i)_{i \in I}$ and $(y_j)_{j \in J}$. If
  $x_{i'} = y_{j'}$ for some $i' \in I$ and $j' \in J$, then
  \[
    x \land y = \bigvee_{i \in I} \bigvee_{j \in J} x_i \land y_j = x_{i'} \land
    y_{j'} \ne 0,
  \]
  contradicting the assumption. This is enough to show that
  \begin{align*}
    \WMC(x \lor y) &= \WMC\left( \left( \bigvee_{i \in I} x_i \right) \lor \left(\bigvee_{j \in J} y_j \right) \right) = \sum_{i \in I} \WMC(x_i) + \sum_{j \in J} \WMC(y_j) \\
                   &= \WMC(x) + \WMC(y),
  \end{align*}
  finishing the proof that $\WMC$ is a measure. This immediately shows that
  $\nWMC$ is a probability measure since, by definition, $\nWMC(1) = 1$.
\end{proof}

Given a theory $\Delta$ in a logic $\mathcal{L}$, the usual way of using WMC to
compute the probability of a query $q$ is
\cite{DBLP:conf/uai/Belle17,DBLP:conf/aaai/SangBK05}
\[
  \Pr_{\Delta, w}(q) = \frac{\WMC_w(\Delta \land q)}{\WMC_w(\Delta)}.
\]
In our algebraic formulation, this can be computed in two different ways:
\begin{itemize}
\item as $\frac{\WMC_w(\Delta \land q)}{\WMC_w(\Delta)}$ in $B(\mathcal{L})$,
\item and as $\nWMC_w([q])$ in $B(\Delta)$.
\end{itemize}
But how does the measure defined on $B(\mathcal{L})$ transfer to $B(\Delta)$?

\begin{lemma} \label{lemma:sum}
  For any measure $m\colon \mathbf{B} \to \mathbb{R}_{\ge 0}$ and elements $a, b
  \in \mathbf{B}$,
  \[
    m(a \lor b) = m(a) + m(b) - m(a \land b).
  \]
\end{lemma}
\begin{proof}
  By \cref{def:measure},
  \begin{align*}
    m(a) &= m(a \land b) + m(a \land \neg b), \\
    m(b) &= m(a \land b) + m(\neg a \land b), \\
    m(a \lor b) &= m(a \land b) + m(a \land \neg b) + m(\neg a \land b),
  \end{align*}
  so
  \[
    m(a) + m(b) - m(a \land b) = m(a \land b) + m(a \land \neg b) + m(\neg a
    \land b) = m(a \lor b)
  \]
  as required.
\end{proof}

\begin{lemma} \label{lemma:well-defined}
  For any $a, b \in \mathbf{B}$ and any principal ideal $(p)$, if $a/(p) =
  b/(p)$, then $a \lor p = b \lor p$.
\end{lemma}
\begin{proof}
  Note that
  \[
    a/(p) = b/(p) \quad \iff \quad a+b \in (p) \quad \iff \quad a+b \le p \quad
    \iff \quad (a+b) \lor p = p
  \]
  by \cref{def:quotient,def:ideal}, and the definition of $\le$. So $p = (a
  \land \neg b) \lor (\neg a \land b) \lor p$, and thus
  \[
    0 = p \land \neg p = (a \land \neg b \land \neg p) \lor (\neg a \land b
    \land \neg p) \lor (p \land \neg p) = (a \land \neg (b \lor p)) \lor (b
    \land \neg (a \lor p)).
  \]
  It follows that
  \[
    a \land \neg (b \lor p) = 0 \quad \text{and} \quad b \land \neg (a \lor p) =
    0.
  \]
  Focusing on the first equation,
  \[
    \neg a = (\neg a \lor a) \land [\neg a \lor \neg (b \lor p)] = \neg [a \land
    (b \lor p)],
  \]
  and so $a = a \land (b \lor p)$, and
  \[
    a \lor p = (a \lor p) \land (b \lor p) = (a \land b) \lor p.
  \]
  By similar arguments, $b \lor p = (a \land b) \lor p$ as well which shows that
  $a \lor p = b \lor p$ as required.
\end{proof}

\todo[inline]{Outdated. $m(a \land \neg p)$ is better than $m (a \lor p)$.}
\begin{proposition}[Measures on quotients] \label{def:measures_on_quotients}
  Let $m\colon \mathbf{B} \to \mathbb{R}_{\ge 0}$ be a measure, and let $(p)$ be
  a principal ideal. Let $m^*\colon \mathbf{B}/(p) \to \mathbb{R}_{\ge 0}$ be
  defined as
  \[
    m^*(a/(p)) = m(a \lor p)
  \]
  for any $a \in \mathbf{B}$. The function $m^*$ is well-defined. Furthermore,
  it is a measure on $\mathbf{B}/(p)$ if and only if $m(p) = 0$. Moreover, if it
  is a measure, then the following properties transfer from $m$ to $m^*$:
  \begin{itemize}
  \item if $m$ is a \emph{probability measure}, then so is $m^*$;
  \item if $m$ is \emph{strictly positive}, then so is $m^*$.
  \end{itemize}
\end{proposition}
\begin{proof}
  \Cref{lemma:well-defined} proves that the function is well-defined. Next,
  note that
  \[
    m^*(0/(p)) = m(0 \lor p) = m(p),
  \]
  so $m^*(0/(p)) = 0$ if and only if $m(p) = 0$. For the second part of
  \cref{def:measure}, let $a/(p), b/(p) \in \mathbf{B}/(p)$ be such that
  \[
    a/(p) \land b/(p) = a \land b / (p) = 0 / (p).
  \]
  This condition is equivalent to $a \land b \in (p)$ and $(a \land b) \lor p =
  p$ by well-known properties of quotients and ideals
  \cite{givant2008introduction}, \cref{def:ideal}, and the definition of $\le$,
  respectively. Now
  \begin{align*}
    m^*(a/(p) \lor b/(p)) &= m^*(a \lor b / (p)) = m(a \lor b \lor p) = m((a \lor p) \lor (b \lor p)) \\
                          &= m(a \lor p) + m(b \lor p) - m((a \lor p) \land (b \lor p)) \\
                          &= m^*(a/(p)) + m^*(b/(p)) - m((a \lor p) \land (b \lor p))
  \end{align*}
  by \cref{lemma:sum}. However
  \[
    (a \lor p) \land (b \lor p) = (a \land b) \lor p = p,
  \]
  so $m^*(a/(p) \lor b/(p)) = m^*(a/(p)) + m^*(b/(p))$ if and only if $m(p) = 0$.

  The two remaining properties are easy to prove:
  \begin{itemize}
  \item If $m(1) = 1$, then $m^*(1/(p)) = m(1 \lor p) = m (1) = 1$.
  \item Suppose that $m$ is strictly positive, and let $a/(p) \in
    \mathbf{B}/(p)$ be such that $a/(p) \ne 0/(p)$. Then
    \[
      m^*(a/(p)) = m(a \lor p) \ge m(a) > 0,
    \]
    where the first inequality comes from an elementary property of $\le$ that
    $x \le x \lor y$ for any $x, y \in \mathbf{B}$ \cite{sikorski1969boolean}
    and \cref{lemma:m_and_order}; and the second inequality follows because
    $a/(p) \ne 0/(p)$ implies that $a \ne 0$, and $m$ is assumed to be strictly
    positive.
  \end{itemize}
\end{proof}
% TODO: add parentheses around quotients (e.g., (a+b)/I).
% TODO: I think I need to cite something for: a/I = 0/I <=> a \in I

\subsection{Lemma Galore}

\todo[inline]{This section made me realise that I was using the wrong definition}

% TODO: maybe 'unite' this with the result about a \/ p, i.e., a ~ a \/ p ~ a /\
% ~p.
% TODO: maybe I should state that A,B are assumed to be BAs with no additional
% properties unless stated otherwise.
\begin{lemma} \label{lemma:elements_in_quotients}
  Let $(p)$ be a principal ideal. Then for any $a \in \mathbf{B}$, $(a \land
  \neg p)/(p) = a/(p)$.
\end{lemma}
\begin{proof}
  Note that
  \[
    (a \land \neg p)/(p) = a/(p) \quad \iff \quad (a \land \neg p) + a \in (p)
    \quad \iff \quad (a \land \neg p) + a \le p.
  \]
  We also have that
  \[
    (a \land \neg p) + a = (a \land \neg p \land \neg a) \lor (\neg(a \land \neg
    p) \land a) = (\neg a \lor p) \land a = (\neg a \land a) \lor (p \land a) =
    p \land a.
  \]
  And, since $p \land a \le p$, we have that $(a \land \neg p) + a \le p$ as
  required.
\end{proof} % TODO: why? (the last bit) It's elementary
% TODO: remove all those 'let (p) be a principal ideal'.

\begin{lemma} \label{lemma:min_element_and_order}
  Let $(p)$ be a principal ideal. For any $a, b \in \mathbf{B}$, $a/(p) \le
  b/(p)$ if and only if $a \land \neg p \le b \land \neg p$.
\end{lemma}
\begin{proof}
  Let us begin with the `only of' direction. \Cref{lemma:elements_in_quotients}
  tells us that $(a \land \neg p)/(p) = a/(p)$. Combining this with
  \cref{lemma:canonical_homomorphism,lemma:homomorphisms_and_order} shows that
  \[
    a \land \neg p \le b \land \neg p \quad \implies \quad (a \land \neg p)/(p)
    \le (b \land \neg p)/(p) \quad \iff \quad a/(p) \le b/(p)
  \]
  as required.

  For the other direction, let $a, b \in \mathbf{B}$ be such that $a/(p) \le
  b/(p)$. Then, by \cref{lemma:order},
  \[
    [a/(p)] \land \neg [b/(p)] = (a \land \neg b)/(p) = 0/(p),
  \]
  i.e.,
  \[
    a \land \neg b \in (p) \quad \iff \quad a \land \neg b \le p \quad \iff
    \quad a \land \neg b \land \neg p = 0
  \]
  by \cref{def:ideal,lemma:order}. We need to show that $a \land \neg p \le b
  \land \neg p$. By \cref{lemma:order}, this is equivalent to $a \land \neg p
  \land \neg(b \land \neg p) = 0$. But
  \[
    a \land \neg p \land \neg(b \land \neg p) = a \land \neg p \land (\neg b
    \lor p) = (a \land \neg p \land \neg b) \lor (a \land \neg p \land p) = a
    \land \neg p \land \neg b,
  \]
  and we already have that $a \land \neg p \land \neg b = 0$ by assumption.
\end{proof}

\begin{lemma} \label{lemma:measure_of_atom}
  Let $m\colon \mathbf{B} \to \mathbb{R}_{\ge 0}$ be a measure, let $p \in
  \mathbf{B}$ be such that $m(p) = 0$, and let $m^*\colon \mathbf{B}/(p) \to
  \mathbb{R}_{\ge 0}$ be a measure defined by $m^*(a/(p)) = m(a \lor p)$. Then
  for any $a \in \mathbf{B}$, if $a/(p)$ is an atom in $\mathbf{B}/(p)$, then $a
  \land \neg p$ is an atom in $\mathbf{B}$ such that $m^*(a/(p)) = m(a \land
  \neg p)$.
\end{lemma}
\begin{proof} % TODO: This proof is a mess and needs more citations.
  First, we want to show that if $a/(p)$ is an atom, then $a \land \neg p$ is an
  atom. We can instead prove the contrapositive statement, i.e., if there exists
  a $b \in \mathbf{B}$ such that $0 < b < a \land \neg p$, then there exists a
  $b' \in \mathbf{B}$ such that $0/(p) < b'/(p) < a/(p)$. We will show that, in
  fact, we set $b' = b$.
  \Cref{lemma:canonical_homomorphism,lemma:homomorphisms_and_order} already tell
  us that $b/(p) \le a/(p)$, so we only need to show that $0/(p) < b/(p) \ne
  a/(p)$. For the first part, note that
  \[
    0/(p) < b/(p) \quad \iff \quad b/(p) \ne 0/(p) \quad \iff \quad b \not\in
    (p) \quad \iff \quad b \not\le p \quad \iff \quad b \land \neg p \ne 0
  \]
  by \cref{lemma:order}. But if $b \land \neg p = 0$, then $b \land a \land \neg
  p = 0$. This contradicts either that $b \le a \land \neg p$ (i.e., $b \land a
  \land \neg p = b$) or that $b \ne 0$. For the second part, i.e., $b/(p) \ne
  a/(p)$, we will show that if $b/(p) = a/(p)$, and $b \le a \land \neg p$, then
  $b = a \land \neg p$. Indeed,
  \[
    b/(p) = a/(p) \quad \iff \quad a + b \in (p) \quad \iff \quad a+b \le p
    \quad \iff \quad (a+b) \land \neg p = 0,
  \]
  and
  \[
    (a+b) \land \neg p = [(a \land \neg b) \lor (\neg a \land b)] \land \neg p =
    (a \land \neg b \land \neg p) \lor (\neg a \land b \land \neg p),
  \]
  so $(a+b) \land \neg p = 0$ implies that $a \land \neg b \land \neg p = 0$
  which is equivalent to $a \land \neg p \le b$. Therefore we have that $a \land
  \neg p \le b \le a \land \neg p$, so $b = a \land \neg p$ which, by
  contradiction, shows that $b/(p) \ne a/(p)$ and finishes the proof that $0/(p)
  < b/(p) < a/(p)$.

  In order to show that $m^*(a/(p)) = m(a \land \neg p)$, note that $a \land p$,
  $a \land \neg p$, and $\neg a \land p$ are pairwise disjoint and their
  supremum is $a \lor p$, so we have that
  \[
    m^*(a/(p)) = m(a \lor p) =  m(a \land p) + m(a \land \neg p) + m(\neg a
    \land p).
  \]
  Furthermore, since $a \land p \le p$, $m(a \land p) \le m(p) = 0$. Similarly,
  $m(\neg a \land p) = 0$, so $m^*(a/(p)) = m(a \land \neg p)$ as required.
\end{proof}
% TODO: the undergraduate book vaguely mentions that, when taking a quotient
% w.r.t. an ideal of elements with measure zero, all elements in an equivalence
% class will have the same measure.

\begin{lemma} \label{lemma:minimal_in_quotient} % TODO: define completeness
  Let $\mathbf{B}$ be a complete BA. For any $a, b \in \mathbf{B}$, if $a/(p) =
  b/(p)$, then $a \land \neg p = b \land \neg p$. As a consequence, $a \land
  \neg p \le b$.
\end{lemma}
\begin{proof} % TODO: rename b to something else
  As in the proof of \cref{lemma:measure_of_atom}, $a/(p) = b/(p)$ implies that
  $a \land \neg p \le b$. Since $\mathbf{B}$ is complete, let $b = \bigwedge \{
  c \in \mathbf{B} \mid c/(p) = a/(p) \}$; then we still have that $b/(p) =
  a/(p)$. But then $b \le a \land \neg p \le b$, so $a \land \neg p = b$. This
  defines $a \land \neg p$ independently of $a$ as the least element in $\{c \in
  \mathbf{B} \mid c/(p) = a/(p) \}$.
\end{proof}

\begin{corollary}
  For any complete BA $\mathbf{B}$, if $a \in \mathbf{B}$ is an atom, then
  $a/(p)$ is either an atom or $0/(p)$. In the former case, $a = a \land \neg
  p$.
\end{corollary}
\begin{proof}
  Since \cref{lemma:minimal_in_quotient} tells us that for all $b \in
  \mathbf{B}$, if $b/(p) = a/(p)$, then $b \ge a \land \neg p$, if there is an
  atom $b \in \mathbf{B}$ such that $b/(p) = a/(p)$, then it must be $a \land
  \neg p$. If $a$ is an atom, then $a \land \neg p \le a$ implies that either $a
  = a \land \neg p$ or $a \land \neg p = 0$. The latter is equivalent to $a/(p)
  = 0/(p)$ by \cref{lemma:order}. The former, combined with the assumption that
  $a$ is an atom and \cref{lemma:min_element_and_order}, implies that $a/(p)$ is
  an atom.
\end{proof}

% TODO: start here
%\[
%  \begin{tikzcd}
%    \mathbf{B}/(p) \arrow[rr, shift left, "x/(p) \mapsto x \lor p"] \arrow[rr,
%    shift right, swap, "x/(p) \mapsto x \land \neg p"] & & \mathbf{B} \arrow{r}{\WMC} & \mathbb{R}_{\ge 0}
%  \end{tikzcd}
%\]

%\begin{theorem}[Using the previous results]
%  Let $L \subset \mathbf{B}$ and $w\colon L \to \mathbb{R}_{\ge 0}$ be such that
%  $\WMC_w\colon \mathbf{B} \to \mathbb{R}_{\ge0}$ is a WMC measure. Let $p \in
%  \mathbf{B}$ be such that $\WMC(p) = 0$, and let $g\colon \mathbf{B}/(p) \to
%  \mathbf{B}$ be defined as $g(a/(p)) = a \land \neg p$. Then $\WMC \circ g$
%  is a WMC measure on $\mathbf{B}/(p)$ with respect to the same $L$ and $w$
%  such that
%  \[
%    \begin{tikzcd}
%      \mathbf{B} \arrow{d}[swap]{\WMC} & \mathbf{B}/(p) \arrow{l}[swap]{g}
%      \arrow{dl}{\WMC \circ g} \\
%      \mathbb{R}_{\ge 0}
%    \end{tikzcd}
%  \]
%  commutes.
%\end{theorem}

% TODO: how can WMC compute something in the quotient algebra? Add an example.
% TODO: we can then move through the quotient BA using representatives we get
% from mapping true things to 1 and false things to 0.

% TODO: Feedback: Can you say something here about factorized vs non-factorized
% weight function definitions? That is, factorized is when w maps literals to
% R_>=0, non-factorized is when w maps models to R_>=0 and show:
% a) come up with nice example when non-factorized weights are intuitive
% b) what if weight functions are negative/complex?
% c) clarify that the factorized definition have is w.r.t. models, in case some
% one gets confused [It doesn't have to be, if the BA is not free -- P.]
% d) can you say something about WMI

\section{What Measures Are WMC-Computable?}

\todo[inline]{Proofs need to be updated and propositions could be phrased in a
  better way, but the gist should be the same.}

\subsection{WMC Requires Independent Literals}

% TODO: maybe I should gives this kind of a BA a name? A synonym of 'complete',
% perhaps.
% TODO: a special case for weight=0.

\begin{proposition}
  Let $\mathbf{B}$ be a finite measure algebra with measure $m\colon \mathbf{B} \to
  \mathbb{R}_{\ge 0}$. Let $L \subset \mathbf{B}$ be defined as
  \[
  L = \{ l_i \mid i \in [n] \} \cup \{ \neg l_i \mid i \in [n] \}
  \]
  for some $n \in \mathbb{N}$. Finally, assume that $\mathbf{B}$ has $2^n$
  atoms, where each atom $a \in \mathbf{B}$ is an infimum
  \[
    a = \bigwedge_{i=1}^n a_i
  \]
  such that $a_i \in \{ l_i, \neg l_i \}$ for $i \in [n]$. Then there exists a
  weight function $w\colon L \to \mathbb{R}_{\ge 0}$ that makes $m$ a WMC measure if
  and only if
  \begin{equation} \label{eq:wmccondition}
  m(l \land l') = m(l)m(l')
  \end{equation}
  for all distinct $l, l' \in L$ such that $l \ne \neg l'$.
\end{proposition}

\begin{remark}
  Note that if $n = 1$, then \cref{eq:wmccondition} is vacuously satisfied and
  so any valid measure can be expressed as WMC.
\end{remark}

\begin{proof}
  Let us begin with the `if' part of the statement. Let $w\colon L \to
  \mathbb{R}_{\ge 0}$ be defined by
  \begin{equation} \label{eq:assumption}
    w(l) = m(l)
  \end{equation}
  for all $l \in L$. We are going
  to show that $\nWMC = m$. First, note that $\nWMC(0) = 0 = m(0)$ by the
  definitions of both $\nWMC$ and $m$. Second, let
  \begin{equation} \label{eq:def_of_a}
    a = \bigwedge_{i=1}^n a_i
  \end{equation}
  be an atom in $\mathbf{B}$ such that $a_i \in \{ l_i, \neg l_i \}$ for all $i
  \in [n]$. Then
  \[
    \nWMC(a) = \frac{\WMC(a)}{\WMC(1)} = \frac{1}{\WMC(1)} \prod_{i=1}^n w(a_i)
    = \frac{1}{\WMC(1)} \prod_{i=1}^n m(a_i) = \frac{1}{\WMC(1)} m \left(
      \bigwedge_{i=1}^n a_i \right) = \frac{m(a)}{\WMC(1)}
  \]
  by \cref{def:wmc,eq:assumption,eq:wmccondition,eq:def_of_a}. Now we just need
  to show that $\WMC(1) = 1$. Indeed,
  \begin{align*}
    \WMC(1) &= \sum_{\text{atoms } a \in \mathbf{B}} \WMC(a) = \sum_{\text{atoms
      } a \in \mathbf{B}} \prod_{i=1}^n w(a_i) = \sum_{\text{atoms } a \in
      \mathbf{B}} \prod_{i=1}^n m(a_i) \\
    &= \sum_{\text{atoms } a \in
      \mathbf{B}} m \left( \bigwedge_{i=1}^n a_i \right) = \sum_{\text{atoms } a
      \in \mathbf{B}} m(a) = m \left( \bigvee_{\text{atoms } a \in \mathbf{B}}
    \right) = m(1) = 1.
  \end{align*}
  Finally, note that if $\nWMC$ and $m$ agree on all atoms, then they must also
  agree on all other non-zero elements of the Boolean algebra.

  For the other direction, we are given a weight function $w\colon L \to
  \mathbb{R}_{\ge 0}$ that induces a measure $m = \nWMC\colon \mathbf{B} \to
  \mathbb{R}_{\ge 0}$, and we want to show that \cref{eq:wmccondition} is
  satisfied. Let $k_i, k_j \in L$ be such that $k_i \in \{ l_i, \neg l_i \}$,
  $k_j \in \{ l_j, \neg l_j \}$, and $i \ne j$. We will first prove an auxiliary
  result that
  \begin{equation} \label{eq:to_prove}
    m(k_i \land k_j) = m(k_i)m(k_j)
  \end{equation}
  is equivalent to
  \begin{equation} \label{eq:to_prove2}
    m(k_i \land k_j) \cdot m(\neg k_i \land \neg k_j) = m(k_i \land \neg k_j)
    \cdot m(\neg k_i \land k_j).
  \end{equation}
  First, note that $k_i$ can be expressed as
  \[
    k_i = (k_i \land k_j) \lor (k_i \land \neg k_j)
  \]
  with the condition that
  \[
    (k_i \land k_j) \land (k_i \land \neg k_j) = 0,
  \]
  so, by properties of a measure,
  \begin{equation} \label{eq:temp}
    m(k_i) = m(k_i \land k_j) + m(k_i \land \neg k_j).
  \end{equation}
  Applying \cref{eq:temp} and the equivalent expression for $m(k_j)$ allows us
  to rewrite \cref{eq:to_prove} as
  \begin{align*}
    m(k_i \land k_j) &= [m(k_i \land k_j) + m(k_i \land \neg k_j)] \cdot [m(k_i \land k_j) + m(\neg k_i \land k_j)] \\
                     &= m(k_i \land k_j)^2 + m(k_i \land k_j)[m(k_i \land \neg k_j) + m(\neg k_i \land k_j)] + m(k_i \land \neg k_j)m(\neg k_i \land k_j)
  \end{align*}
  Dividing both sides by $m(k_i \land k_j)$ gives
  \begin{equation} \label{eq:temp2}
    1 = m(k_i \land k_j) + m(k_i \land \neg k_j) + m(\neg k_i \land k_j) +
    \frac{m(k_i \land \neg k_j)m(\neg k_i \land k_j)}{m(k_i \land k_j)}.
  \end{equation}
  Since $k_i \land k_j \land k_i \land \neg k_j = 0$, and
  \[
    (k_i \land k_j) \lor (k_i \land \neg k_j) = k_i \land (k_j \lor \neg k_j) =
    k_i \land 1 = k_i,
  \]
  we have that
  \[
    m(k_i \land k_j) + m(k_i \land \neg k_j) = m(k_i).
  \]
  Similarly, $k_i \land \neg k_i \land k_j = 0$, and
  \[
    k_i \lor (\neg k_i \land k_j) = (k_i \lor \neg k_i) \land (k_i \lor k_j) =
    k_i \lor k_j,
  \]
  so
  \[
    m(k_i) + m(\neg k_i \land k_j) = m(k_i \lor k_j).
  \]
  Finally, note that
  \[
    (k_i \lor k_j) \land \neg(k_i \lor k_j) = 0,
  \]
  and
  \[
    (k_i \lor k_j) \lor \neg(k_i \lor k_j) = 1,
  \]
  so
  \[
    m(k_i \lor k_j) + m(\neg(k_i \lor k_j)) = m(1) = 1.
  \]
  This allows us to rewrite \cref{eq:temp2} as
  \[
    \frac{m(k_i \land \neg k_j)m(\neg k_i \land k_j)}{m(k_i \land k_j)} = 1 -
    m(k_i \lor k_j) = m(\neg(k_i \lor k_j)) = m(\neg k_i \land \neg k_j)
  \]
  which immediately gives us \cref{eq:to_prove2}.

  Now recall that $m = \nWMC$ and note that \cref{eq:to_prove2} can be
  multiplied by $\WMC(1)^2$ to turn the equation into one for $\WMC$ instead of
  $\nWMC$. Then
  \begin{align*}
    \WMC(k_i \land k_j) &= \sum_{\text{atoms } a \le k_i \land k_j} \WMC(a) = \sum_{\text{atoms } a \le k_i \land k_j} \prod_{m \in [n]} w(a_m) \\
                        &= \sum_{\text{atoms } a \le k_i \land k_j} w(a_i)w(a_j) \prod_{m \in [n] \setminus \{ i, j \}} w(a_m) = \sum_{\text{atoms } a \le k_i \land k_j} w(k_i)w(k_j) \prod_{m \in [n] \setminus \{ i, j \}} w(a_m) \\
    &= w(k_i)w(k_j) \sum_{\text{atoms } a \le k_i \land k_j} \prod_{m \in [n] \setminus \{ i, j \}} w(a_m) = w(k_i)w(k_j)C,
  \end{align*}
  where $C$ denotes the part of $\WMC(k_i \land k_j)$ that will be the same for
  $\WMC(\neg k_i \land k_j)$, $\WMC(k_i \land \neg k_j)$, and $\WMC(\neg k_i
  \land \neg k_j)$ as well. But then \cref{eq:to_prove2} becomes
  \[
    w(k_i)w(k_j)w(\neg k_i)w(\neg k_j)C^2 = w(k_i)w(\neg k_j)w(\neg k_i)w(k_j)C^2
  \]
  which is trivially true. By showing that WMC satisfies \cref{eq:to_prove2}, we
  also showed that it satisfies \cref{eq:to_prove}, finishing the second part of
  the proof.
\end{proof}
% TODO: the auxiliary result should be a 'claim' in-between the theorem and the proof.

\subsection{Extending the Algebra}

A well-known way to overcome this limitation of independence is by adding more
literals \cite{DBLP:journals/ai/ChaviraD08}, i.e., extending the set $L$ covered
by the WMC weight function $w\colon L \to \mathbb{R}_{\ge 0}$. Let us translate this
idea to the language of Boolean algebras.

\begin{theorem} \label{thm:extension} % TODO: cite the fact about atoms
  Let $\mathbf{B}$ be a finite Boolean algebra freely generated by some set of
  `literals' $L$, and let $m\colon \mathbf{B} \to \mathbb{R}_{\ge 0}$ be an
  arbitrary measure. We know that $\mathbf{B}$ has $n = 2^{|L|}$ atoms. Let
  $(a_i)_{i=1}^n$ denote those atoms in some arbitrary order. Let $L' = L \cup
  \{ \phi_i \mid i \in [n] \} \cup \{ \neg \phi_i \mid i \in [n] \}$ be the set
  $L$ extended with $2n$ new literals. Let $\mathbf{B'}$ be the unique Boolean
  algebra with
  \[
    \{ \phi_i \land a_i \mid i \in [n] \} \cup \{ \neg \phi_i \land a_i \mid i
    \in [n] \}
  \]
  as its set of atoms. Let $\iota\colon \mathbf{B} \to \mathbf{B'}$ be the inclusion
  homomorphism (i.e., $\iota(a) = a$ for all $a \in \mathbf{B}$). Let $w\colon L'
  \to \mathbb{R}_{\ge 0}$ be defined by
  \[
    w(l) = \begin{cases}
      \frac{m(a_i)}{2} & \text{if } l = \phi_i \text{ or } l = \neg\phi_i \text{
        for some } i \in [n] \\
      1 & \text{otherwise}
    \end{cases}
  \]
  for all $l \in L'$, and note that this defines a WMC measure $m'\colon \mathbf{B'}
  \to \mathbb{R}_{\ge 0}$. Then
  \[
    m(a) = (m' \circ \iota)(a)
  \]
  for all $a \in \mathbf{B}$.
\end{theorem}

In simpler terms, any measure can be computed using WMC by extending the Boolean
algebra with more literals. More precisely, we are given the red part in
\[
  \begin{tikzcd}
    \textcolor{red}{\mathbb{R}_{\ge 0}} & & \\
    \textcolor{red}{\mathbf{B}} \arrow[red]{u}{m} \arrow{r}{\iota} &
    \mathbf{B'} \arrow{lu}[swap]{m'} & \\
    \textcolor{red}{L} \arrow[Subset,red]{u}{} \arrow[Subset]{r}{} & L'
    \arrow[Subset]{u}{} \arrow{r}{w} & \mathbb{R}_{\ge 0}
  \end{tikzcd}
\]
and construct the black part in such a way that the triangle commutes.

% TODO: make J depend on i
\begin{proof} % TODO: find a reference for this first claim
  Since $\mathbf{B}$ is freely generated by $L$, each atom $a_i \in \mathbf{B}$
  is an infimum of elements in $L$, i.e.,
  \[
    a_i = \bigwedge_{j \in J} a_{i,j}
  \]
  for some $\{ a_{i,j} \}_{j \in J} \subset L$. Moreover, each atom $b \in
  \mathbf{B'}$ can be represented as either
  \[
    b = \phi_i \land a_i \quad \text{or} \quad b = \neg\phi_i \land a_i
  \]
  for some atom $a_i \in \mathbf{B}$, also making it an infimum over a subset of
  $L'$. Then, for any $b \in \mathbf{B}$,
  \[
    (m' \circ \iota)(b) = \sum_{\substack{\text{atoms } a_i \in \mathbf{B}:\\
        \phi_i \land a_i \le \iota(b)}} (w(\phi_i) + w(\neg\phi_i)) \prod_{j \in
    J} w(a_{i,j}),
  \]
  recognising that, for any $\iota(b)$, any atom $a_i \in \mathbf{B}$ satisfies
  \[
    \phi_i \land a_i \le \iota(b)
  \]
  if and only if it satisfies
  \[
    \neg\phi_i \land a_i \le \iota(b).
  \]
  Then, according to the definition of $w$,
  \[
    (m' \circ \iota)(b) = \sum_{\substack{\text{atoms } a_i \in \mathbf{B}:\\
        \phi_i \land a_i \le \iota(b)}} (w(\phi_i) + w(\neg\phi_i)) =
    \sum_{\substack{\text{atoms } a_i \in \mathbf{B}:\\ \phi_i \land a_i \le
        \iota(b)}} m(a_i) = m(b),
  \]
  provided that
  \[
    \phi_i \land a_i \le \iota(b) \quad \text{if and only if} \quad a_i \le b,
  \]
  but this is equivalent to
  \[
    \phi_i \land a_i = \phi_i \land a_i \land b \quad \text{if and only if}
    \quad a_i = a_i \land b
  \]
  which is true because $\phi_i \not\in L$.
\end{proof}

Now we can show that the construction in \cref{thm:extension} is smallest
possible.

\begin{conjecture}
  Let $\mathbf{B}$ and $\mathbf{B'}$ be Boolean algebras, and $\iota\colon
  \mathbf{B} \to \mathbf{B'}$ be the inclusion map such that $\mathbf{B}$ is
  freely generated by $L$, all atoms of $\mathbf{B'}$ can be expressed as
  meets of elements of $L'$, and the following subset relations are satisfied:
  \[
    \begin{tikzcd}
      \mathbf{B} \arrow{r}{\iota} & \mathbf{B'} \\
      L \arrow[Subset]{u}{} \arrow[Subset]{r}{} & L' \arrow[Subset]{u}{}
    \end{tikzcd}
  \]
  If, for any measure $m\colon \mathbf{B} \to \mathbb{R}_{\ge 0}$, one can
  construct a weight function $w\colon L' \to \mathbb{R}_{\ge 0}$ such that the WMC
  measure $\WMC\colon \mathbf{B'} \to \mathbb{R}_{\ge 0}$ with respect to $w$
  satisfies
  \[
    m = \WMC \circ \iota,
  \]
  then $|L' \setminus L| \ge 2^{|L|+1}$.
\end{conjecture}
% \begin{proof}
%   % 1. An atom in B' must have more than just elements of L.
%   Let $a$ be an atom in $\mathbf{B}$, and let $b$ be an atom in $\mathbf{B'}$
%   such that $b \le a$. First, let us notice that as long as $|L| \ge
%   4$\footnote{Note that $|L|$ has to be an even number.}, $b \ne a$. Indeed, let
%   $p, r, \neg p, \neg r \in L$. Then
%   \begin{align*}
%     (\WMC \circ \iota)(p \land r) &= w(p)w(r), \\
%     (\WMC \circ \iota)(p \land \neg r) &= w(p)w(\neg r), \\
%     (\WMC \circ \iota)(\neg p \land r) &= w(\neg p)w(r), \\
%     (\WMC \circ \iota)(\neg p \land \neg r) &= w(\neg p)w(\neg r), \\
%   \end{align*}
%   But then we have that
%   \[
%     \frac{m(p \land r)}{m(\neg p \land r)} = \frac{w(p)}{w(\neg p)} =
%     \frac{m(p \land \neg r)}{m(\neg p \land \neg r)}.
%   \]
%   This places a condition on $m$, contradicting the assumption that the
%   construction works for an arbitrary $m$. Hence $b < a$.

%   Second, we can show that if $b = a \land \bigwedge_{i = 1}^k \phi_i$ for some
%   positive integer $k$, then there must also be $2^k - 1$ other atoms in
%   $\mathbf{B'}$ that correspond to every possible way to negate a subset of
%   $\phi_i$'s, i.e., ranging from

%   % 2. If we add phi, then we must also add -phi.
%   % 3. Extension to multiple literals: we must have all (2^n) combinations of
%   % added literals).
%   % 4. Profit
% \end{proof}

Let us note how our lower bound on the number of added literals compares to two
methods of translating a discrete probability distribution into a WMC problem
over a propositional knowledge base proposed by Darwiche
\cite{DBLP:conf/kr/Darwiche02} and Sang et al. \cite{DBLP:conf/aaai/SangBK05}.
Suppose we have a discrete probability distribution with  $n$ variables, and the
$i$th variable has $v_i$ values, for each $i \in [n]$. Interpreted as a logical
system, it has $\prod_{i=1}^n v_i$ models. My expansion would then use
\[
  \sum_{i=1}^n v_i + 2\prod_{i=1}^n v_i
\]
variables, i.e., a variable for each possible variable-value assignment, and two
additional variables for each model. Without making any independence
assumptions, the encoding by Darwiche \cite{DBLP:conf/kr/Darwiche02} would use
\[
  \sum_{i=1}^n v_i + \sum_{i=1}^n \prod_{j=1}^i v_j
\]
variables, while for the encoding by Sang et al. \cite{DBLP:conf/aaai/SangBK05},
\[
  \sum_{i=1}^n v_i + \sum_{i=1}^n (v_i - 1) \prod_{j=1}^{i-1} v_j
\]
variables would suffice.

\section{Implications for Lifted Inference}

\begin{definition}
  Given a BA $\mathbf{A}$, a \emph{subalgebra} is a subset $\mathbf{B} \subseteq
  \mathbf{A}$ that, together with the operations, zero, and one of $\mathbf{A}$,
  is a BA.
\end{definition}

\begin{definition}
  Let $\mathbf{A},$ $\mathbf{B}$, and $\mathbf{C}$ be BAs such that $\mathbf{B}$
  is a subalgebra of $\mathbf{A}$. Let $f\colon \mathbf{A} \to \mathbf{C}$ and
  $g\colon \mathbf{B} \to \mathbf{C}$ be homomorphisms. Then $f$ is an
  \emph{extension} of $g$ if $f(x) = g(x)$ for all $x \in \mathbf{B}$. If $f$ is
  an extension of each member of a family $\{ g_i \}_{i \in I}$ of
  homomorphisms, then $f$ is called a \emph{common extension} of $\{ g_i \}_{i
    \in I}$.
\end{definition}

\begin{definition}
  Let $\{ \mathbf{A}_i \}_{i \in I}$ be a family of subalgebras of a BA
  $\mathbf{A}$ with a family of inclusion maps $\{ \iota_i\colon \mathbf{A}_i
  \to \mathbf{A} \}_{i \in I}$. If for any BA $\mathbf{B}$ with a family of
  homomorphisms $\{ f_i\colon \mathbf{A} \to \mathbf{B} \}_{i \in I}$ there
  exists a unique common extension of $\{ f_i\colon \mathbf{A} \to \mathbf{B}
  \}_{i \in I}$ ($f\colon \mathbf{A} \to \mathbf{B}$ in the diagram),
  \[
    \begin{tikzcd}
      \mathbf{A}_i \arrow{r}{\iota_i} \arrow{rd}[swap]{f_i} & \mathbf{A} \arrow[d,dashed,"f"] \\
      & \mathbf{B}
    \end{tikzcd}
  \]
  then $\mathbf{A}$ is the \emph{internal sum}\footnote{It is also known as the
    \emph{free product} and as the coproduct in the category of BAs.} of $\{
  \mathbf{A}_i \}_{i \in I}$. We will denote it as $\bigoplus_{i \in I}
  \mathbf{A}_i$.
\end{definition}

\section{Polyadic Measure Algebras}

Potential directions to explore:
\begin{itemize}
\item Representing independence and exchangeability. This seems important.
\item (More detail below.) Inequalities as bounds for probabilities. This seems
  to be somewhat explored with other setups.
\item Implementation in SageMath. I would need to define prenex normal forms,
  equality, and lots of other things.
\item Alternative compact ways to define a probability distribution over $N$
  models without assuming that everything is independent. Declaring a measure on
  a FO formula defines a linear equation over the probabilities of models, so
  using this method by itself would require $N-1$ equations, but maybe combining
  this with information about independence and exchangeability can help.
\end{itemize}

\todo[inline]{I'm not sure if it's wise to develop this idea further}

We show how the measure on the models of a FO theory can be extended to a
measure over FO formulas. Two outcomes:
\begin{itemize}
\item For any $p, q \in \mathbf{A}$, if $p \land q = 0$, then $m^*(p \lor q) =
  m^*(p) + m^*(q)$.
\item If $p \le q$, then $m^*(p) \le m^*(q)$. This can be useful in two
  situations:
  \begin{itemize}
  \item If $m^*(q) = 0$ or $m^*(p) = 1$, then this immediately tells us the
    measure on the other sentence.
  \item If we want to find $m^*(p)$, finding $q, r \in \mathbf{A}$ such that $r
    \le p \le q$ bounds the answer.
  \end{itemize}
\end{itemize}

\begin{definition}[\cite{halmos2016algebraic}]
  Given two polyadic algebras $\mathbf{A}$ and $\mathbf{B}$, a \emph{polyadic
    homomorphism} from $\mathbf{A}$ to $\mathbf{B}$ is a Boolean homomorphism
  $f\colon \mathbf{A} \to \mathbf{B}$ such that
  \begin{itemize}
  \item $f\mathbf{S}(\tau)p = \mathbf{S}(\tau)fp$,
  \item $f\bm\exists(J)p = \bm\exists(J)fp$
  \end{itemize}
  for all $\tau \in T$, $p \in \mathbf{A}$, and $J \subseteq I$.
\end{definition}

% by considering several X's, we can have several domains
% I needs to be infinite in order to support arbitrarily complex sentences

\subsection{The Set-Up}

% TODO: Feedback: My general remark here is that this section looks very [???]
% for SRL/WMC/PRM readers. The first-order [???] may not be justified if we are
% only doing essentially propositional models. If we also consider full FoL,
% i.e., infinite domains, and make this a separate paper for KR-type audience.
% That is, let us do the WMC/WMI bit separately and do FO-version in another
% paper for KR track. Title for second paper: "Measures on first-order
% structures via WMC" (or similar...).

\subsubsection{Preliminaries}

What follows is a summary of \cite{halmos2016algebraic}.

% Generic stuff
Let $\mathbf{B}$ be a Boolean algebra (of propositions). Let $X$ be the
(non-empty) domain of discourse. Let $I$ be an index set, elements of which can
be interpreted as variables. The elements of $X^I$ are functions from $I$ to
$X$. For any $x \in X^I$ and $i \in I$, we write $x_i$ to represent $x(i) \in
X$. Let $\mathbf{A^*}$ be the set of all functions $X^I \to \mathbf{B}$, and
note that it forms a Boolean algebra with operations defined pointwise.

% Defining S
Let $T$ be the semigroup of all $I \to I$ transformations. For any $\tau \in
T$, let $\tau_* : X^I \to X^I$ be defined by
\[
  (\tau_* x)_i = x_{\tau i}
\]
for all $x \in X^I$ and $i \in I$. For any (Boolean/polyadic) algebra
$\mathbf{C}$, let $\End(\mathbf{C})$ denote the set of all its endomorphisms. We
can then define $\mathbf{S}$ to be a map $\mathbf{S} : T \to \End(\mathbf{A^*})$
defined by
\[
  \mathbf{S}(\tau)p(x) = p(\tau_* x)
\]
for all $x \in X^I$ and $p \in \mathbf{A^*}$.

% Defining E
For any $J \subseteq I$, let $J_*$ be the relation on $X^I$ defined by
\[
  xJ_*y \quad \iff \quad x_i = y_i \quad \text{for all } i \in I \setminus J
\]
for all $x, y \in X^I$. For any $J \subseteq I$, we then define $\bm\exists(J)$
to be a transformation $\mathbf{A^*} \to \mathbf{A^*}$ defined by
\[
  \bm\exists(J)p(x) = \bigvee_{\substack{y \in X^I,\\ xJ_*y}} p(y)
\]
for all $p \in \mathbf{A^*}$, provided this supremum exists for all $x \in
X^I$\footnote{The universal quantifier $\bm\forall(J)$ is then defined as
  $\bm\forall(J)p = \neg(\bm\exists(J)\neg p)$ for all $p \in \mathbf{A^*}$.}.

Finally, a \emph{functional polyadic (Boolean) algebra}\footnote{To be more
  explicit, a $\mathbf{B}$-valued functional polyadic algebra with domain $X$
  and variables $I$.} is a subalgebra
$\mathbf{A}$ of $\mathbf{A^*}$ such that:
\begin{itemize}
\item $\mathbf{S}(\tau)p \in \mathbf{A}$ for all $p \in \mathbf{A}$ and $\tau
  \in T$;
\item $\bm\exists(J)p \in \mathbf{A}$ for all $p \in \mathbf{A}$ and $J
  \subseteq I$.
\end{itemize}

\begin{proposition}
  For all finite $J, K \subseteq I$, finite $\sigma, \tau \in T$, and all $p \in
  \mathbf{A}$,
  \begin{itemize}
  \item $\bm\exists(\emptyset)p = p$;
  \item $\bm\exists(J)\bm\exists(K) = \bm\exists(J \cup K)$;
  \item $\mathbf{S}(\id)p = p$;
  \item $(\sigma\tau)p = \sigma(\tau p)$;
  \item if $\sigma_{|I \setminus J} = \tau_{|I \setminus J}$, then $\sigma
    \bm\exists(J) = \tau \bm\exists(J)$;
  \item if $\tau$ is injective on $\tau^{-1}J$, then $\bm\exists(J)\tau = \tau
    \bm\exists(\tau^{-1}J)$;
  \item for every $p \in \mathbf{A}$, there exists a finite $J \subseteq I$ such
    that $\bm\exists(i)p = p$ whenever $i \not\in J$.
  \end{itemize}
\end{proposition}

\begin{definition}
  Similarly to $\bm\exists$, a \emph{constant} $c$ is a map $c: \mathcal{P}(I)
  \to \End(\mathbf{A})$ such that:
  \begin{itemize}
  \item $c(\emptyset) = \id_{\mathbf{A}}$;
  \item $c(J \cup K) = c(J)c(K)$;
  \item $c(J)\bm\exists(K) = \bm\exists(K)c(J \setminus K)$;
  \item $\bm\exists(J)c(K) = c(K)\bm\exists(J \setminus K)$;
  \item $c(J)\mathbf{S}(\tau) = \mathbf{S}(\tau)c(\tau^{-1}J)$
  \end{itemize}
  for all $J, K \in \mathcal{P}(I)$ and $\tau \in T$. If $J$ is a singleton
  set $\{ i \}$, we will simply write $c(i)$ instead of $c(J)$.
\end{definition}

\subsubsection{New Results}

\begin{proposition} \label{prop:polyadic_measure}
  Let $\mathbf{B}$ be a finite Boolean algebra with a measure $m :
  \mathbf{B} \to [0, 1]$. Let $\mathbf{A}$ be a $\mathbf{B}$-valued functional
  polyadic algebra with domain $X$ and variables $I$. Let $m^* : \mathbf{A} \to
  \mathbb{R}_{\ge 0}$ be defined by
  \[
    m^*(p) = \sum_{\substack{\text{atoms }y \in \mathbf{B} \text{ s.t.}\\ \exists x \in X^I:\, y \le p(x)}} m(y)
  \]
  for all $p \in \mathbf{A}$. Then $m^*$ is a measure on $\mathbf{A}$.
\end{proposition}

\begin{remark}
  While definitions of $m^*$ such as
  \[
    m^*(p) = m \left( \bigvee_{x \in X^I} p(x) \right)
  \]
  might look tempting, they are not additive.
\end{remark}

\begin{proof} % TODO: Update the proof w.r.t. definitions
  First, we can show that $m^*(1) = 1$ by observing that
  \[
    m^*(1) = \sum_{\text{atoms } y \in \mathbf{B}} m(y) = m \left(
      \bigvee_{\text{atoms } y \in \mathbf{B}} y \right) = m(1) = 1,
  \]
  where we use \cref{thm:representation} and express $1 \in \mathbf{B}$ as the
  supremum of all atoms in $\mathbf{B}$ \cite{ganesh2006introduction}. Clearly
  $m^*(p) \ge 0$ for all $p \in \mathbf{A}$, so we can restrict the codomain of
  $m^*$ to $[0, 1]$.

  Next, we want to show that $m^*(p) > 0$ for all $p \in \mathbf{A} \setminus \{
  0 \}$. If $p \ne 0$, then there must be some $x' \in X^I$ such that $p(x') \ne
  0$. But then, since finite Boolean algebras are atomic, there must also be an
  atom $y \in \mathbf{B}$ such that $y \le p(x')$. Therefore, $m^*(p) \ge m(y) >
  0$, finishing this part of the proof.

  Let $p, q \in \mathbf{A}$ be such that $p \land q = 0$. We want to show
  that $m^*(p \lor q) = m^*(p) \lor m^*(q)$. First, note that
  \[
    y \le (p \lor q)(x) = p(x) \lor q(x)
  \]
  if and only if
  \[
    y = (p(x) \lor q(x)) \land y = (p(x) \land y) \lor (q(x) \land y)
  \]
  by \cref{def:ba}. Also note that
  \[
    (p(x) \land y) \land (q(x) \land y) = p(x) \land q(x) \land y = (p \land
    q)(x) \land y = 0 \land y = 0,
  \]
  so
  \[
    m(y) = m((p(x) \land y) \lor (q(x) \land y)) = m(p(x) \land y) + m(q(x) \land y)
  \]
  by \cref{def:measure} which then leads to
  \begin{align*}
    m^*(p \lor q) &= \sum_{\substack{\text{atoms }y \in \mathbf{B} \text{ s.t.}\\ \exists x \in X^I:\, y \le (p \lor q)(x)}} m(y) = \sum_{\substack{\text{atoms }y \in \mathbf{B} \text{ s.t.}\\ \exists x \in X^I:\, y \le (p \lor q)(x)}} m(p(x) \land y) + m(q(x) \land y) \\
                  &= \sum_{\substack{\text{atoms }y \in \mathbf{B} \text{ s.t.}\\ \exists x \in X^I:\, y \le (p \lor q)(x)}} m(p(x) \land y) + \sum_{\substack{\text{atoms }y \in \mathbf{B} \text{ s.t.}\\ \exists x \in X^I:\, y \le (p \lor q)(x)}} m(q(x) \land y).
  \end{align*}
  Since $y$ is an atom,
  \[
    p(x) \land y  = \begin{cases}
      y & \text{if } y \le p(x) \\
      0 & \text{otherwise,}
    \end{cases}
  \]
  so
  \begin{align*}
    m^*(p \lor q) &= \sum_{\substack{\text{atoms }y \in \mathbf{B} \text{ s.t.}\\ \exists x \in X^I:\, y \le (p \lor q)(x) \text{ and } y \le p(x)}} m(p(x) \land y) + \sum_{\substack{\text{atoms }y \in \mathbf{B} \text{s.t.}\\ \exists x \in X^I:\, y \le (p \lor q)(x) \text{ and } y \le q(x)}} m(q(x) \land y) \\
                  &= \sum_{\substack{\text{atoms }y \in \mathbf{B} \text{ s.t.}\\ \exists x \in X^I:\, y \le p(x)}} m(p(x) \land y) + \sum_{\substack{\text{atoms }y \in \mathbf{B} \text{s.t.}\\ \exists x \in X^I:\, y \le q(x)}} m(q(x) \land y) \\
                  &= \sum_{\substack{\text{atoms }y \in \mathbf{B} \text{ s.t.}\\ \exists x \in X^I:\, y \le p(x)}} m(y) + \sum_{\substack{\text{atoms }y \in \mathbf{B} \text{s.t.}\\ \exists x \in X^I:\, y \le q(x)}} m(y) = m^*(p) + m^*(q),
  \end{align*}
  finishing the proof that $m^*$ is a measure.
\end{proof}

\begin{lemma} \label{lemma:simple_measure}
  Given the setup of \cref{prop:polyadic_measure} and $p \in \mathbf{A}$, if
  $p(x) = p(y)$ for all $x, y \in X^I$ (i.e., $p$ has no free variables), then
  \[
    m^*(p) = m(p(x))
  \]
  (for some $x \in X^I$) is an alternative (i.e., equivalent and simpler)
  definition of $m^*$.
\end{lemma}
\begin{proof}
  Fix some $x \in X^I$. Then
  \[
    m(p(x)) = m \left( \bigvee_{\substack{\text{atoms } y \in \mathbf{B} \text{
            s.t.}\\
          y \le p(x)}} y \right) = \sum_{\substack{\text{atoms } y \in \mathbf{B}
        \text{ s.t.}\\
        y \le p(x)}} m(y) =\sum_{\substack{\text{atoms } y \in \mathbf{B}
        \text{ s.t.}\\
        \exists x' \in X^I : y \le p(x')}} m(y) = m^*(p),
  \]
  where we use \cref{thm:representation} for the first step,
  \cref{def:measure} and \cref{lemma:atomic} for the second step, the
  assumptions of \cref{lemma:simple_measure} for the third step, and the
  definition of $m^*$ for the fourth one.
\end{proof}

\begin{proposition}
  The following identities are true for any $p \in \mathbf{A}$, $J \subseteq I$,
  $\tau \in T$, and constant $c : \mathcal{P}(I) \to \End(A)$:
  \begin{itemize}
  \item $c \neg = \neg c$ (i.e., constants commute with negation);
  \item $\mathbf{S}(\tau) \neg = \neg \mathbf{S}(\tau)$;
  \item $\bm\forall(J)p \le c(J)p \le \bm\exists(J)p$;
  \end{itemize}
\end{proposition}
\begin{proof}
  \begin{itemize}
  \item Obvious.
  \item Ditto.
  \item The first part can be derived from $p \le q \implies \neg q \le \neg p$.
    The second part comes from the definition of $\bm\exists$.
  \end{itemize}
\end{proof}

\subsection{How Probabilities Are Computed}

In order to make the example algebras easily describable, our example programs
will have to be tiny. Consider the following ProbLog
\cite{DBLP:conf/ijcai/RaedtKT07} program:
\begin{align*}
  1.0 &\dblcolon \mathsf{p}(a, b).\\
  0.5 &\dblcolon \mathsf{p}(X, X) \coloneq \mathsf{p}(X, Y);\, \mathsf{p}(Y, X).
\end{align*}
Let $L = \{ \mathsf{p}(a, a), \mathsf{p}(a, b), \mathsf{p}(b, a), \mathsf{p}(b,
b) \}$ be the set of all possible ground atoms. Let $\mathbf{B}$ be the
Boolean algebra freely generated by $L$ (see, e.g.,
\cite{givant2008introduction} for more on free Boolean algebras). Then
$\mathbf{B}$ will have sixteen atoms ranging from $\mathsf{p}(a, a) \land
\mathsf{p}(a, b) \land \mathsf{p}(b, a) \land \mathsf{p}(b, b)$ to
$\neg\mathsf{p}(a, a) \land \neg\mathsf{p}(a, b) \land \neg\mathsf{p}(b, a)
\land \neg\mathsf{p}(b, b)$. The weight function $w : L \to \mathbb{R}_{\ge 0}$
defined by
\[
  w(l) = \begin{cases}
    1 & \text{if } l = \mathsf{p}(a, b) \\
    0.5 & \text{if } l \in \{ \mathsf{p}(a, a), \mathsf{p}(b, b) \} \\
    0 & \text{if } l = \mathsf{p}(b, a) \\
    1-w(l') & \text{if } l = \neg l'
  \end{cases}
\]
for all $l \in L$ defines a WMC measure over $\mathbf{B}$. Note that while we
could define an ideal generated by $\{ \mathsf{p}(b, a), \neg\mathsf{p}(a, b)
\}$ and take the quotient of $\mathbf{B}$ by that ideal to get a Boolean algebra
with a strictly positive measure, this would put zero-probability queries
outside of our algebras, i.e., we would not be able to ask a question whose
answer is zero.

\begin{table}
  \centering
  \caption{Example elements of $\mathbf{A}$ as maps $X^I \to \mathbf{B}$, with
    $a : \mathcal{P}(I) \to \End(\mathbf{A})$ as one of two possible constants.}
  \label{tbl:examples}
  \begin{tabular}{ll}
    \toprule
    Element of $\mathbf{A}$ & Action on $X^I$ \\
    \midrule
    $p = \mathbf{S}(\id)p = \bm\exists(\emptyset)p = a(\emptyset)p = b(\emptyset)p$ & $(x_1, x_2) \mapsto \mathsf{p}(x_1, x_2)$ \\
    $\bm\exists(1)p$ & $(x_1, x_2) \mapsto \mathsf{p}(a, x_2) \lor \mathsf{p}(b, x_2)$ \\
    $\bm\exists(2)p$ & $(x_1, x_2) \mapsto \mathsf{p}(x_1, a) \lor \mathsf{p}(x_1, b)$ \\
    $\bm\exists(I)p$ & $(x_1, x_2) \mapsto \mathsf{p}(a, a) \lor \mathsf{p}(a, b) \lor \mathsf{p}(b, a) \lor \mathsf{p}(b, b)$ \\
    $\mathbf{S}(\{ 1 \mapsto 1, 2 \mapsto 1 \})p$ & $(x_1, x_2) \mapsto \mathsf{p}(x_1, x_1)$ \\
    $\mathbf{S}(\{ 1 \mapsto 2, 2 \mapsto 1 \})p$ & $(x_1, x_2) \mapsto \mathsf{p}(x_2, x_1)$ \\
    $\mathbf{S}(\{ 1 \mapsto 2, 2 \mapsto 2 \})p$ & $(x_1, x_2) \mapsto \mathsf{p}(x_2, x_2)$ \\
    $a(1)p$ & $(x_1, x_2) \mapsto \mathsf{p}(a, x_2)$ \\
    $a(2)p$ & $(x_1, x_2) \mapsto \mathsf{p}(x_1, a)$ \\
    $a(I)p$ & $(x_1, x_2) \mapsto \mathsf{p}(a, a)$ \\
    \bottomrule
  \end{tabular}
\end{table}

Finally, let $\mathbf{A}$ be the functional polyadic algebra $X^I \to
\mathbf{B}$ for $I = \{1, 2\}$ and $X = \{ a, b \}$\footnote{$X$ cannot (or
  should not) have constants that do not occur in $\mathbf{B}$.}. The elements
of $X^I$ can then be represented as tuples $(x_1, x_2)$ for some $x_1, x_2 \in
X$. See \cref{tbl:examples} for example elements of $\mathbf{A}$ which consists
of a single predicate function $p$ and operators $\bm\exists, \mathbf{S}, a, b,
\neg, \land, \lor$, the last three of which are defined pointwise.

\begin{table}
  \centering
  \caption{Step-by-step derivation of how a more complex element of
    $\mathbf{A}$ acts on elements of $X^I$}
  \label{tbl:derivation}
  \begin{tabular}{ll}
    \toprule
    Element of $\mathbf{A}$ & Action on $X^I$ \\
    \midrule
    $p$ & $(x_1, x_2) \mapsto \mathsf{p}(x_1, x_2)$ \\
    $b(2)p$ & $(x_1, x_2) \mapsto \mathsf{p}(x_1, b)$ \\
    $\neg b(2)p$ & $(x_1, x_2) \mapsto \neg\mathsf{p}(x_1, b)$ \\
    $\bm\exists(1)\neg b(2)p$ & $(x_1, x_2) \mapsto \neg\mathsf{p}(a, b) \lor \neg\mathsf{p}(b, b) = \neg(\mathsf{p}(a, b) \land \mathsf{p}(b, b))$ \\
    $\bm\forall(1)b(2)p = \neg\bm\exists(1)\neg b(2)p$ & $(x_1, x_2) \mapsto \neg\neg(\mathsf{p}(a, b) \land \mathsf{p}(b, b)) = \mathsf{p}(a, b) \land \mathsf{p}(b, b)$ \\
    \bottomrule
  \end{tabular}
\end{table}

\begin{table}
  \centering
  \caption{Atoms $y \in \mathbf{B}$ (and their measures) such that $y \le
    \mathsf{p}(a, b) \land \mathsf{p}(b, b)$}
  \label{tbl:atoms}
  \begin{tabular}{lc}
    \toprule
    Atom $y \in \mathbf{B}$ & $m(y)$ \\
    \midrule
    $\mathsf{p}(a, b) \land \mathsf{p}(b, b) \land \mathsf{p}(a, a) \land \mathsf{p}(b, a)$ & $1 \times 0.5 \times 0.5 \times 0 = 0$ \\
    $\mathsf{p}(a, b) \land \mathsf{p}(b, b) \land \neg\mathsf{p}(a, a) \land \mathsf{p}(b, a)$ & $1 \times 0.5 \times 0.5 \times 0 = 0$ \\
    $\mathsf{p}(a, b) \land \mathsf{p}(b, b) \land \mathsf{p}(a, a) \land \neg\mathsf{p}(b, a)$ & $1 \times 0.5 \times 0.5 \times 1 = 0.25$ \\
    $\mathsf{p}(a, b) \land \mathsf{p}(b, b) \land \neg\mathsf{p}(a, a) \land \neg\mathsf{p}(b, a)$ & $1 \times 0.5 \times 0.5 \times 1 = 0.25$ \\
    \bottomrule
  \end{tabular}
\end{table}

Let us calculate the probability $\Pr(\forall x_1 \in X, \mathsf{p}(x_1, b))$.
The same expression can be translated into the notation for our polyadic algebra
$\mathbf{A}$ as $m^*(\bm\forall(1)b(2)p)$. Recall that $\bm\forall(1)b(2)p =
\neg\bm\exists(1)\neg b(2)p$. The effect of this function on an arbitrary
element of $X^I$ is derived step-by-step in \cref{tbl:derivation}. Since the
resulting function is constant (i.e., the logical formula has no free
variables), \cref{lemma:simple_measure} tells us that
\[
  m^*(\bm\forall(1)b(2)p) = m(\mathsf{p}(a, b) \land \mathsf{p}(b, b)) = m
  \left( \bigvee_{\substack{\text{atoms } y \in \mathbf{B} \text{ s.t.}\\
        y \le \mathsf{p}(a, b) \land \mathsf{p}(b, b)}} y \right) = \sum
  _{\substack{\text{atoms } y \in \mathbf{B} \text{ s.t.}\\
      y \le \mathsf{p}(a, b) \land \mathsf{p}(b, b)}} m(y).
\]
The resulting sum is over four atoms; these atoms and their probabilities are
listed in \cref{tbl:atoms}. Thus, we get that
\[
  m^*(\bm\forall(1)b(2)p) = 0 + 0 + 0.25 + 0.25 = 0.5.
\]

% TODO: use I^I for transformations
% TODO: extra notation for transformations: (i/j) sends j to i, (i, j)
% interchanges i and j.

\bibliographystyle{plain}
\bibliography{paper}

\end{document}
