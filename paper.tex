\documentclass{article}
\usepackage[utf8]{inputenc}
\usepackage[UKenglish]{babel}
\usepackage[UKenglish]{isodate}
\usepackage{fullpage}
\usepackage{amsthm}
\usepackage{amsfonts}
\usepackage{amsmath}
\usepackage{mathtools}
\usepackage[capitalise]{cleveref}
\usepackage{bm}
\usepackage{booktabs}
\usepackage[backgroundcolor=lightgray]{todonotes}

\newtheorem{theorem}{Theorem}
\newtheorem{lemma}{Lemma}
\newtheorem{proposition}{Proposition}
\theoremstyle{definition}
\newtheorem{definition}{Definition}
\theoremstyle{remark}
\newtheorem*{remark}{Remark}

\Crefname{property}{Property}{Properties}
\DeclareMathOperator{\WMC}{WMC}
\DeclareMathOperator{\nWMC}{nWMC}
\DeclareMathOperator{\id}{id}
\DeclareMathOperator{\End}{End}

\title{Weighted Model Counting as a Special Case of Polyadic Measure Algebras}
\author{Paulius Dilkas}

\begin{document}
\maketitle

%\section{Introduction}

%Contributions
%\begin{itemize}
%\item Equip polyadic algebras with a measure.
%\item Show that it is equivalent to WMC.
%\end{itemize}

\section{Propositional Logic and Boolean Algebras}

\subsection{Preliminaries}

\begin{definition} \label{def:ba}
  A \emph{Boolean algebra} is a tuple $(\mathbf{B}, \land, \lor, \neg, 0, 1)$ of
  a set $\mathbf{B}$ with operations $\land, \lor, \neg$ and elements $0, 1 \in
  \mathbf{B}$ such that the following axioms hold for all $a, b, \in
  \mathbf{B}$:
  \begin{itemize}
  \item both $\land$ and $\lor$ are associative and commutative;
  \item $a \lor (a \land b) = a$, and $a \land (a \lor b) = a$;
  \item $0$ is the identity of $\lor$, and $1$ is the identity of $\land$;
  \item $\lor$ distributes over $\land$ and vice versa;
  \item $a \lor \neg a = 1$, and $a \land \neg a = 0$.
  \end{itemize}

  Let $a, b \in \mathbf{B}$ be arbitrary. Let $\le$ be a partial order on
  $\mathbf{B}$ defined by $a \le b$ if $a = b \land a$ (or, equivalently, $a
  \lor b = b$), and let $a < b$ denote $a \le b$ and $a \ne b$.
\end{definition}

\todo[inline]{Which definition do I actually need?}
\begin{definition}[\cite{DBLP:books/daglib/0090259,levasseur2012applied}]
  An element $a \ne 0$ of a Boolean algebra $\mathbf{B}$ is an \emph{atom} if
  there is no $x \in \mathbf{B}$ such that $0 < x < a$. Equivalently, $a \ne 0$
  is an atom if, for all $x \in \mathbf{B}$, either $x \land a = a$ or $x \land
  a = 0$. A Boolean algebra is \emph{atomic} if for every $a \in \mathbf{B}
  \setminus \{0 \}$, there is an atom $x$ such that $x \le a$.
\end{definition}

\begin{lemma}[\cite{ganesh2006introduction}]
  For any two distinct atoms $a$, $b$ in a Boolean algebra, $a \land b = 0$.
\end{lemma}

\begin{lemma}[\cite{givant2008introduction}] \label{lemma:atomic}
  All finite Boolean algebras are atomic.
\end{lemma}

\begin{theorem} \label{thm:representation}
  Let $\mathbf{B}$ be a finite Boolean algebra. Then every $a \in \mathbf{B}
  \setminus \{ 0 \}$ can be uniquely expressed as $a = \bigvee_{i \in I} a_i$,
  where $\{ a_i \}_{i \in I}$ is the set of atoms in $\mathbf{B}$ such that $a_i
  \le a$ for all $i \in I$.
\end{theorem}
\begin{proof}
  A simple consequence of the theorem that every finite Boolean algebra is
  isomorphic to a field of subsets of a set, where the cardinality of the set is
  equal to the number of atoms in the Boolean algebra.
\end{proof}

\todo[inline]{Remove the requirement for being strictly positive}
\begin{definition}[\cite{gaifman1964concerning}] \label{def:measure}
  A \emph{(strictly positive) measure} on a Boolean algebra $\mathbf{B}$ is a
  function $m : \mathbf{B} \to [0, 1]$ such that:
  \begin{enumerate}
  \item $m(1) = 1$, and $m(x) > 0$ for $x \ne
    0$; \label[property]{property:values}
  \item $m(x \lor y) = m(x) + m(y)$ for all $x, y \in \mathbf{B}$ whenever $x
    \land y = 0$. \label[property]{property:additive}
  \end{enumerate}
\end{definition}

\subsection{New Results}

\todo[inline]{Allow weight to be zero}
\begin{definition}
  Let $\mathbf{B}$ be a finite Boolean algebra, and let $M \subseteq \mathbf{B}$
  be its set of atoms. Let $L \subseteq \mathbf{B}$ be such that every atom $m
  \in M$ can be uniquely expressed as $m = \bigwedge_{i \in I} l_i$ for some $\{
  l_i \}_{i \in I} \subseteq L$, and let $w : L \to \mathbb{R}_{>0}$ be
  arbitrary. The \emph{weighted model count} $\WMC : \mathbf{B} \to
  \mathbb{R}_{\ge 0}$ is defined as
  \[
    \WMC(a) = \begin{cases}
      0 & \text{if } a = 0 \\
      \prod_{i \in I} w(l_i) & \text{if } M \ni a = \bigwedge_{i \in I} l_i \\
      \sum_{i \in I} \WMC(m_i) & \text{if } \mathbf{B} \setminus (M \cup \{ 0
      \}) \ni a = \bigvee_{i \in I} m_i
    \end{cases}
  \]
  for any $a \in \mathbf{B}$. Furthermore, we define the \emph{normalised
    weighted model count} $\nWMC : \mathbf{B} \to [0, 1]$ as $\nWMC(a) =
  \frac{\WMC(a)}{\WMC(1)}$ for all $a \in \mathbf{B}$.
\end{definition}

\begin{proposition}
  $\nWMC$ is a measure for any finite Boolean algebra $\mathbf{B}$.
\end{proposition}
\begin{proof}
  First, note that \cref{property:values} of \cref{def:measure} is satisfied by
  the definition of $\nWMC$. Next, in order to prove \cref{property:additive},
  let $x, y \in \mathbf{B}$ be such that $x \land y = 0$. We want to show that
  \[
    \nWMC(x \lor y) = \nWMC(x) + \nWMC(y)
  \]
  which is equivalent to
  \begin{equation} \label{eq:additivity_proof}
    \WMC(x \lor y) = \WMC(x) + \WMC(y).
  \end{equation}
  If, say, $x = 0$, then \cref{eq:additivity_proof} becomes
  \[
    \WMC(y) = \WMC(0) + \WMC(y) = \WMC(y)
  \]
  (and likewise for $y = 0$). Thus we can assume that $x \ne 0 \ne y$ and use
  \cref{thm:representation} to write
  \[
    x = \bigvee_{i \in I} x_i \quad \text{and} \quad y = \bigvee_{j \in J} y_j
  \]
  for some sequences of atoms $(x_i)_{i \in I}$ and $(y_j)_{j \in J}$. If
  $x_{i'} = y_{j'}$ for some $i' \in I$ and $j' \in J$, then
  \[
    x \land y = \bigvee_{i \in I} \bigvee_{j \in J} x_i \land y_j = x_{i'} \land
    y_{j'} \ne 0,
  \]
  contradicting the assumption. This is enough to show that
  \begin{align*}
    \WMC(x \lor y) &= \WMC\left( \left( \bigvee_{i \in I} x_i \right) \lor \left(\bigvee_{j \in J} y_j \right) \right) = \sum_{i \in I} \WMC(x_i) + \sum_{j \in J} \WMC(y_j) \\
                   &= \WMC(x) + \WMC(y),
  \end{align*}
  finishing the proof.
\end{proof}

\section{First-Order Logic and Polyadic Algebras}

\subsection{Preliminaries}

What follows is a summary of \cite{halmos2016algebraic}.

% Generic stuff
Let $\mathbf{B}$ be a Boolean algebra (of propositions). Let $X$ be the
(non-empty) domain of discourse. Let $I$ be an index set, elements of which can
be interpreted as variables. The elements of $X^I$ are functions from $I$ to
$X$. For any $x \in X^I$ and $i \in I$, we write $x_i$ to represent $x(i) \in
X$. Let $\mathbf{A^*}$ be the set of all functions $X^I \to \mathbf{B}$, and
note that it forms a Boolean algebra with operations defined pointwise.

% Defining S
Let $T$ be the semigroup of all $I \to I$ transformations. For any $\tau \in
T$, let $\tau_* : X^I \to X^I$ be defined by
\[
  (\tau_* x)_i = x_{\tau i}
\]
for all $x \in X^I$ and $i \in I$. For any (Boolean/polyadic) algebra
$\mathbf{C}$, let $\End(\mathbf{C})$ denote the set of all its endomorphisms. We
can then define $\mathbf{S}$ to be a map $\mathbf{S} : T \to \End(\mathbf{A^*})$
defined by
\[
  \mathbf{S}(\tau)p(x) = p(\tau_* x)
\]
for all $x \in X^I$ and $p \in \mathbf{A^*}$.

% Defining E
For any $J \subseteq I$, let $J_*$ be the relation on $X^I$ defined by
\[
  xJ_*y \quad \iff \quad x_i = y_i \quad \text{for all } i \in I \setminus J
\]
for all $x, y \in X^I$. For any $J \subseteq I$, we then define $\bm\exists(J)$
to be a transformation $\mathbf{A^*} \to \mathbf{A^*}$ defined by
\[
  \bm\exists(J)p(x) = \bigvee_{\substack{y \in X^I,\\ xJ_*y}} p(y)
\]
for all $p \in \mathbf{A^*}$, provided this supremum exists for all $x \in
X^I$\footnote{The universal quantifier $\bm\forall(J)$ is then defined as
  $\bm\forall(J)p = \neg(\bm\exists(J)\neg p)$ for all $p \in \mathbf{A^*}$.}.

Finally, a \emph{functional polyadic (Boolean) algebra}\footnote{To be more
  explicit, a $\mathbf{B}$-valued functional polyadic algebra with domain $X$
  and variables $I$.} is a subalgebra
$\mathbf{A}$ of $\mathbf{A^*}$ such that:
\begin{itemize}
\item $\mathbf{S}(\tau)p \in \mathbf{A}$ for all $p \in \mathbf{A}$ and $\tau
  \in T$;
\item $\bm\exists(J)p \in \mathbf{A}$ for all $p \in \mathbf{A}$ and $J
  \subseteq I$.
\end{itemize}

\begin{definition}
  Similarly to $\bm\exists$, a \emph{constant} $c$ is a map $c: \mathcal{P}(I)
  \to \End(\mathbf{A})$ (Boolean endomorphisms?) such that:
  \begin{itemize}
  \item $c(\emptyset) = \id_{\mathbf{A}}$;
  \item $c(J \cup K) = c(J)c(K)$;
  \item $c(J)\bm\exists(K) = \bm\exists(K)c(J \setminus K)$;
  \item $\bm\exists(J)c(K) = c(K)\bm\exists(J \setminus K)$;
  \item $c(J)\mathbf{S}(\tau) = \mathbf{S}(\tau)c(\tau^{-1}J)$
  \end{itemize}
  for all $J, K \in \mathcal{P}(I)$ and $\tau \in T$. If $J$ is a singleton
  set $\{ i \}$, we will simply write $c(i)$ instead of $c(J)$.
\end{definition}

\subsection{New Results}

\begin{proposition} \label{prop:polyadic_measure}
  Let $\mathbf{B}$ be a finite Boolean algebra with a measure $m :
  \mathbf{B} \to [0, 1]$. Let $\mathbf{A}$ be a $\mathbf{B}$-valued functional
  polyadic algebra with domain $X$ and variables $I$. Let $m^* : \mathbf{A} \to
  \mathbb{R}_{\ge 0}$ be defined by
  \[
    m^*(p) = \sum_{\substack{\text{atoms }y \in \mathbf{B} \text{ s.t.}\\ \exists x \in X^I:\, y \le p(x)}} m(y)
  \]
  for all $p \in \mathbf{A}$. Then $m^*$ is a measure on $\mathbf{A}$.
\end{proposition}

\begin{remark}
  While definitions of $m^*$ such as
  \[
    m^*(p) = m \left( \bigvee_{x \in X^I} p(x) \right)
  \]
  might look tempting, they are not additive.
\end{remark}

\begin{proof}\todo[inline]{Update the proof w.r.t. definitions}
  First, we can show that $m^*(1) = 1$ by observing that
  \[
    m^*(1) = \sum_{\text{atoms } y \in \mathbf{B}} m(y) = m \left(
      \bigvee_{\text{atoms } y \in \mathbf{B}} y \right) = m(1) = 1,
  \]
  where we use \cref{thm:representation} and express $1 \in \mathbf{B}$ as the
  supremum of all atoms in $\mathbf{B}$ \cite{ganesh2006introduction}. Clearly
  $m^*(p) \ge 0$ for all $p \in \mathbf{A}$, so we can restrict the codomain of
  $m^*$ to $[0, 1]$.

  Next, we want to show that $m^*(p) > 0$ for all $p \in \mathbf{A} \setminus \{
  0 \}$. If $p \ne 0$, then there must be some $x' \in X^I$ such that $p(x') \ne
  0$. But then, since finite Boolean algebras are atomic, there must also be an
  atom $y \in \mathbf{B}$ such that $y \le p(x')$. Therefore, $m^*(p) \ge m(y) >
  0$, finishing this part of the proof.

  Let $p, q \in \mathbf{A}$ be such that $p \land q = 0$. We want to show
  that $m^*(p \lor q) = m^*(p) \lor m^*(q)$. First, note that
  \[
    y \le (p \lor q)(x) = p(x) \lor q(x)
  \]
  if and only if
  \[
    y = (p(x) \lor q(x)) \land y = (p(x) \land y) \lor (q(x) \land y)
  \]
  by \cref{def:ba}. Also note that
  \[
    (p(x) \land y) \land (q(x) \land y) = p(x) \land q(x) \land y = (p \land
    q)(x) \land y = 0 \land y = 0,
  \]
  so
  \[
    m(y) = m((p(x) \land y) \lor (q(x) \land y)) = m(p(x) \land y) + m(q(x) \land y)
  \]
  by \cref{def:measure} which then leads to
  \begin{align*}
    m^*(p \lor q) &= \sum_{\substack{\text{atoms }y \in \mathbf{B} \text{ s.t.}\\ \exists x \in X^I:\, y \le (p \lor q)(x)}} m(y) = \sum_{\substack{\text{atoms }y \in \mathbf{B} \text{ s.t.}\\ \exists x \in X^I:\, y \le (p \lor q)(x)}} m(p(x) \land y) + m(q(x) \land y) \\
                  &= \sum_{\substack{\text{atoms }y \in \mathbf{B} \text{ s.t.}\\ \exists x \in X^I:\, y \le (p \lor q)(x)}} m(p(x) \land y) + \sum_{\substack{\text{atoms }y \in \mathbf{B} \text{ s.t.}\\ \exists x \in X^I:\, y \le (p \lor q)(x)}} m(q(x) \land y).
  \end{align*}
  Since $y$ is an atom,
  \[
    p(x) \land y  = \begin{cases}
      y & \text{if } y \le p(x) \\
      0 & \text{otherwise,}
    \end{cases}
  \]
  so
  \begin{align*}
    m^*(p \lor q) &= \sum_{\substack{\text{atoms }y \in \mathbf{B} \text{ s.t.}\\ \exists x \in X^I:\, y \le (p \lor q)(x) \text{ and } y \le p(x)}} m(p(x) \land y) + \sum_{\substack{\text{atoms }y \in \mathbf{B} \text{s.t.}\\ \exists x \in X^I:\, y \le (p \lor q)(x) \text{ and } y \le q(x)}} m(q(x) \land y) \\
                  &= \sum_{\substack{\text{atoms }y \in \mathbf{B} \text{ s.t.}\\ \exists x \in X^I:\, y \le p(x)}} m(p(x) \land y) + \sum_{\substack{\text{atoms }y \in \mathbf{B} \text{s.t.}\\ \exists x \in X^I:\, y \le q(x)}} m(q(x) \land y) \\
                  &= \sum_{\substack{\text{atoms }y \in \mathbf{B} \text{ s.t.}\\ \exists x \in X^I:\, y \le p(x)}} m(y) + \sum_{\substack{\text{atoms }y \in \mathbf{B} \text{s.t.}\\ \exists x \in X^I:\, y \le q(x)}} m(y) = m^*(p) + m^*(q),
  \end{align*}
  finishing the proof that $m^*$ is a measure.
\end{proof}

\begin{lemma} \label{lemma:simple_measure}
  Given the setup of \cref{prop:polyadic_measure} and $p \in \mathbf{A}$, if
  $p(x) = p(y)$ for all $x, y \in X^I$ (i.e., $p$ has no free variables), then
  \[
    m^*(p) = m(p(x))
  \]
  (for some $x \in X^I$) is an alternative (i.e., equivalent and simpler)
  definition of $m^*$.
\end{lemma}
\begin{proof}
  Fix some $x \in X^I$. Then
  \[
    m(p(x)) = m \left( \bigvee_{\substack{\text{atoms } y \in \mathbf{B} \text{
            s.t.}\\
          y \le p(x)}} y \right) = \sum_{\substack{\text{atoms } y \in \mathbf{B}
        \text{ s.t.}\\
        y \le p(x)}} m(y) =\sum_{\substack{\text{atoms } y \in \mathbf{B}
        \text{ s.t.}\\
        \exists x' \in X^I : y \le p(x')}} m(y) = m^*(p),
  \]
  where we use \cref{thm:representation} for the first step,
  \cref{def:measure} and \cref{lemma:atomic} for the second step, the
  assumptions of \cref{lemma:simple_measure} for the third step, and the
  definition of $m^*$ for the fourth one.
\end{proof}

\section{From First-Order Logic to Polyadic Algebras}

\subsection{Preliminaries}

\begin{definition}[\cite{givant2008introduction}] \label{def:boolean_ideal}
  An \emph{ideal} in a Boolean algebra $\mathbf{B}$ is a subset $M \subseteq
  \mathbf{B}$ such that:
  \begin{itemize}
  \item $0 \in M$;
  \item $x \lor y \in M$ for all $x, y \in M$;
  \item $x \land y \in M$ for all $x \in M$ and $y \in \mathbf{B}$.
  \end{itemize}
  For any subset $S \subseteq \mathbf{B}$, the \emph{ideal generated by $S$} is
  the smallest ideal $M$ such that $S \subseteq M$.
\end{definition}

Note that \cref{def:boolean_ideal} gives us a simple characterisation of an
ideal generated by a set of atoms.

\begin{lemma}
  Let $\mathbf{B}$ be a Boolean algebra, and let $S \subseteq \mathbf{B}$ be a
  set of atoms. The ideal $I$ generated by $S$ is defined by the following:
  \begin{itemize}
  \item $0 \in I$,
  \item $S \subseteq I$,
  \item $x \lor y \in I$ for all $x, y \in I$.
  \end{itemize}
\end{lemma}

\begin{definition}[\cite{givant2008introduction}]
  Let $\mathbf{B}$ be a Boolean algebra, and let $I$ be an ideal in
  $\mathbf{B}$. The \emph{quotient algebra} $\mathbf{B}/I$ is a Boolean algebra
  on equivalence classes of elements of $\mathbf{B}$ (with operations defined
  pointwise) based on the equivalence relation
  \[
    x \sim y \quad \iff \quad x + y \in I
  \]
  where $x + y = (x \land \neg y) \lor (y \land \neg x)$ is the symmetric
  difference operation (written as a sum because it can interpreted as the
  `additive' part of the corresponding Boolean ring).
\end{definition}

\subsection{New Results (an Example)}

In order to make the example algebras easily describable, our example programs
will have to be tiny. Consider the following ProbLog
\cite{DBLP:conf/ijcai/RaedtKT07} program:
\begin{align*}
  1.0 &\dblcolon \mathsf{p}(a, b).\\
  0.5 &\dblcolon \mathsf{p}(X, X) \coloneq \mathsf{p}(X, Y);\, \mathsf{p}(Y, X).
\end{align*}
Let $L = \{ \mathsf{p}(a, a), \mathsf{p}(a, b), \mathsf{p}(b, a), \mathsf{p}(b,
b) \}$ be the set of all possible ground atoms. Let $\mathbf{B}$ be the
Boolean algebra freely generated by $L$ (see, e.g.,
\cite{givant2008introduction} for more on free Boolean algebras). Then
$\mathbf{B}$ will have sixteen atoms ranging from $\mathsf{p}(a, a) \land
\mathsf{p}(a, b) \land \mathsf{p}(b, a) \land \mathsf{p}(b, b)$ to
$\neg\mathsf{p}(a, a) \land \neg\mathsf{p}(a, b) \land \neg\mathsf{p}(b, a)
\land \neg\mathsf{p}(b, b)$. The weight function $w : L \to \mathbb{R}_{\ge 0}$
defined by
\[
  w(l) = \begin{cases}
    1 & \text{if } l = \mathsf{p}(a, b) \\
    0.5 & \text{if } l \in \{ \mathsf{p}(a, a), \mathsf{p}(b, b) \} \\
    0 & \text{if } l = \mathsf{p}(b, a) \\
    1-w(l') & \text{if } l = \neg l'
  \end{cases}
\]
for all $l \in L$ defines a WMC measure over $\mathbf{B}$. Note that while we
could define an ideal generated by $\{ \mathsf{p}(b, a), \neg\mathsf{p}(a, b)
\}$ and take the quotient of $\mathbf{B}$ by that ideal to get a Boolean algebra
with a strictly positive measure, this would put zero-probability queries
outside of our algebras, i.e., we would not be able to ask a question whose
answer is zero.

\begin{table}
  \centering
  \caption{Example elements of $\mathbf{A}$ as maps $X^I \to \mathbf{B}$, with
    $a : \mathcal{P}(I) \to \End(\mathbf{A})$ as one of two possible constants.}
  \label{tbl:examples}
  \begin{tabular}{ll}
    \toprule
    Element of $\mathbf{A}$ & Action on $X^I$ \\
    \midrule
    $p = \mathbf{S}(\id)p = \bm\exists(\emptyset)p = a(\emptyset)p = b(\emptyset)p$ & $(x_1, x_2) \mapsto \mathsf{p}(x_1, x_2)$ \\
    $\bm\exists(1)p$ & $(x_1, x_2) \mapsto \mathsf{p}(a, x_2) \lor \mathsf{p}(b, x_2)$ \\
    $\bm\exists(2)p$ & $(x_1, x_2) \mapsto \mathsf{p}(x_1, a) \lor \mathsf{p}(x_1, b)$ \\
    $\bm\exists(I)p$ & $(x_1, x_2) \mapsto \mathsf{p}(a, a) \lor \mathsf{p}(a, b) \lor \mathsf{p}(b, a) \lor \mathsf{p}(b, b)$ \\
    $\mathbf{S}(\{ 1 \mapsto 1, 2 \mapsto 1 \})p$ & $(x_1, x_2) \mapsto \mathsf{p}(x_1, x_1)$ \\
    $\mathbf{S}(\{ 1 \mapsto 2, 2 \mapsto 1 \})p$ & $(x_1, x_2) \mapsto \mathsf{p}(x_2, x_1)$ \\
    $\mathbf{S}(\{ 1 \mapsto 2, 2 \mapsto 2 \})p$ & $(x_1, x_2) \mapsto \mathsf{p}(x_2, x_2)$ \\
    $a(1)p$ & $(x_1, x_2) \mapsto \mathsf{p}(a, x_2)$ \\
    $a(2)p$ & $(x_1, x_2) \mapsto \mathsf{p}(x_1, a)$ \\
    $a(I)p$ & $(x_1, x_2) \mapsto \mathsf{p}(a, a)$ \\
    \bottomrule
  \end{tabular}
\end{table}

Finally, let $\mathbf{A}$ be the functional polyadic algebra $X^I \to
\mathbf{B}$ for $I = \{1, 2\}$ and $X = \{ a, b \}$\footnote{$X$ cannot (or
  should not) have constants that do not occur in $\mathbf{B}$.}. The elements
of $X^I$ can then be represented as tuples $(x_1, x_2)$ for some $x_1, x_2 \in
X$. See \cref{tbl:examples} for example elements of $\mathbf{A}$ which consists
of a single predicate function $p$ and operators $\bm\exists, \mathbf{S}, a, b,
\neg, \land, \lor$, the last three of which are defined pointwise.

\begin{table}
  \centering
  \caption{Step-by-step derivation of how a more complex element of
    $\mathbf{A}$ acts on elements of $X^I$}
  \label{tbl:derivation}
  \begin{tabular}{ll}
    \toprule
    Element of $\mathbf{A}$ & Action on $X^I$ \\
    \midrule
    $p$ & $(x_1, x_2) \mapsto \mathsf{p}(x_1, x_2)$ \\
    $b(2)p$ & $(x_1, x_2) \mapsto \mathsf{p}(x_1, b)$ \\
    $\neg b(2)p$ & $(x_1, x_2) \mapsto \neg\mathsf{p}(x_1, b)$ \\
    $\bm\exists(1)\neg b(2)p$ & $(x_1, x_2) \mapsto \neg\mathsf{p}(a, b) \lor \neg\mathsf{p}(b, b) = \neg(\mathsf{p}(a, b) \land \mathsf{p}(b, b))$ \\
    $\bm\forall(1)b(2)p = \neg\bm\exists(1)\neg b(2)p$ & $(x_1, x_2) \mapsto \neg\neg(\mathsf{p}(a, b) \land \mathsf{p}(b, b)) = \mathsf{p}(a, b) \land \mathsf{p}(b, b)$ \\
    \bottomrule
  \end{tabular}
\end{table}

\begin{table}
  \centering
  \caption{Atoms $y \in \mathbf{B}$ (and their measures) such that $y \le
    \mathsf{p}(a, b) \land \mathsf{p}(b, b)$}
  \label{tbl:atoms}
  \begin{tabular}{lc}
    \toprule
    Atom $y \in \mathbf{B}$ & $m(y)$ \\
    \midrule
    $\mathsf{p}(a, b) \land \mathsf{p}(b, b) \land \mathsf{p}(a, a) \land \mathsf{p}(b, a)$ & $1 \times 0.5 \times 0.5 \times 0 = 0$ \\
    $\mathsf{p}(a, b) \land \mathsf{p}(b, b) \land \neg\mathsf{p}(a, a) \land \mathsf{p}(b, a)$ & $1 \times 0.5 \times 0.5 \times 0 = 0$ \\
    $\mathsf{p}(a, b) \land \mathsf{p}(b, b) \land \mathsf{p}(a, a) \land \neg\mathsf{p}(b, a)$ & $1 \times 0.5 \times 0.5 \times 1 = 0.25$ \\
    $\mathsf{p}(a, b) \land \mathsf{p}(b, b) \land \neg\mathsf{p}(a, a) \land \neg\mathsf{p}(b, a)$ & $1 \times 0.5 \times 0.5 \times 1 = 0.25$ \\
    \bottomrule
  \end{tabular}
\end{table}

Let us calculate the probability $\Pr(\forall x_1 \in X, \mathsf{p}(x_1, b))$.
The same expression can be translated into the notation for our polyadic algebra
$\mathbf{A}$ as $m^*(\bm\forall(1)b(2)p)$. Recall that $\bm\forall(1)b(2)p =
\neg\bm\exists(1)\neg b(2)p$. The effect of this function on an arbitrary
element of $X^I$ is derived step-by-step in \cref{tbl:derivation}. Since the
resulting function is constant (i.e., the logical formula has no free
variables), \cref{lemma:simple_measure} tells us that
\[
  m^*(\bm\forall(1)b(2)p) = m(\mathsf{p}(a, b) \land \mathsf{p}(b, b)) = m
  \left( \bigvee_{\substack{\text{atoms } y \in \mathbf{B} \text{ s.t.}\\
        y \le \mathsf{p}(a, b) \land \mathsf{p}(b, b)}} y \right) = \sum
  _{\substack{\text{atoms } y \in \mathbf{B} \text{ s.t.}\\
      y \le \mathsf{p}(a, b) \land \mathsf{p}(b, b)}} m(y).
\]
The resulting sum is over four atoms; these atoms and their probabilities are
listed in \cref{tbl:atoms}. Thus, we get that
\[
  m^*(\bm\forall(1)b(2)p) = 0 + 0 + 0.25 + 0.25 = 0.5.
\]

\todo[inline,caption={}]{The big TODO list
  \begin{itemize}
  \item Condition for a measurable Boolean algebra to be representable as WMC.
    Note: if all atoms are in $L$, then WMC can define any distribution.
  \item Extension to infinite (atomic?) Boolean algebras.
  \item How many extra variable do you need to add to make any probability
    distribution representable using WMC?
  \item Abstraction refinements as homomorphisms.
  \item Definition of a measure-preserving homomorphism from Jech's set theory
    book.
  \item A Boolean algebra is approximable if its Stone space is approximable.
  \end{itemize}
}

\section{Homomorphisms}

\begin{definition}[\cite{givant2008introduction}]
  Let $\mathbf{A}$ and $\mathbf{B}$ be Boolean algebras. A \emph{Boolean
    homomorphism} from $\mathbf{A}$ to $\mathbf{B}$ is a map $f : \mathbf{A} \to
  \mathbf{B}$ such that:
  \begin{itemize}
  \item $f(x \land y) = f(x) \land f(y)$,
  \item $f(x \lor y) = f(x) \lor f(y)$,
  \item $f(\neg x) = \neg f(x)$
  \end{itemize}
  for all $x, y \in \mathbf{A}$.
\end{definition}

\begin{definition}[\cite{halmos2016algebraic}]
  Given two polyadic algebras $\mathbf{A}$ and $\mathbf{B}$, a \emph{polyadic
    homomorphism} from $\mathbf{A}$ to $\mathbf{B}$ is a Boolean homomorphism $f
  : \mathbf{A} \to \mathbf{B}$ such that
  \begin{itemize}
  \item $f\mathbf{S}(\tau)p = \mathbf{S}(\tau)fp$,
  \item $f\bm\exists(J)p = \bm\exists(J)fp$
  \end{itemize}
  for all $\tau \in T$, $p \in \mathbf{A}$, and $J \subseteq I$.
\end{definition}

\bibliographystyle{plain}
\bibliography{paper}

\end{document}