\documentclass{article}
\usepackage[utf8]{inputenc}
\usepackage[UKenglish]{babel}
\usepackage[UKenglish]{isodate}
\usepackage{fullpage}
\usepackage{amsthm}
\usepackage{amsfonts}
\usepackage{amsmath}
\usepackage{mathtools}
\usepackage[capitalise]{cleveref}
\usepackage{bm}

\newtheorem{theorem}{Theorem}
\newtheorem{lemma}{Lemma}
\newtheorem{proposition}{Proposition}
\theoremstyle{definition}
\newtheorem{definition}{Definition}
\theoremstyle{remark}
\newtheorem*{remark}{Remark}

\Crefname{property}{Property}{Properties}
\DeclareMathOperator{\WMC}{WMC}
\DeclareMathOperator{\nWMC}{nWMC}
\DeclareMathOperator{\id}{id}

\title{Statistical Relational Models as Polyadic Measure Algebras}
\author{Paulius Dilkas}

\begin{document}
\maketitle

%\section{Introduction}

%Contributions
%\begin{itemize}
%\item Equip polyadic algebras with a measure.
%\item Show that it is equivalent to WMC.
%\end{itemize}

\section{Propositional Logic and Boolean Algebras}

\subsection{Preliminaries}

\begin{definition}
  A \emph{Boolean algebra} is a tuple $(\mathbf{B}, \land, \lor, \neg, 0, 1)$ of
  a set $\mathbf{B}$ with operations $\land, \lor, \neg$ and elements $0, 1 \in
  \mathbf{B}$ such that the following axioms hold for all $a, b, \in
  \mathbf{B}$:
  \begin{itemize}
  \item both $\land$ and $\lor$ are associative and commutative;
  \item $a \lor (a \land b) = a$, and $a \land (a \lor b) = a$;
  \item $0$ is the identity of $\lor$, and $1$ is the identity of $\land$;
  \item $\lor$ distributes over $\land$ and vice versa;
  \item $a \lor \neg a = 1$, and $a \land \neg a = 0$.
  \end{itemize}

  Let $a, b \in \mathbf{B}$ be arbitrary. Let $\le$ be a partial order on
  $\mathbf{B}$ defined by $a \le b$ if $a = b \land a$ (or, equivalently, $a
  \lor b = b$), and let $a < b$ denote $a \le b$ and $a \ne b$.
\end{definition}

\begin{definition}[\cite{DBLP:books/daglib/0090259}]
  An element $a \ne 0$ of a Boolean algebra $\mathbf{B}$ is an \emph{atom} if
  there is no $x \in \mathbf{B}$ such that $0 < x < a$. A Boolean algebra is
  \emph{atomic} if for every $a \in \mathbf{B} \setminus \{ 0 \}$, there is an
  atom $x$ such that $x \le a$.
\end{definition}

\begin{lemma}[\cite{ganesh2006introduction}]
  For any two distinct atoms $a$, $b$ in a Boolean algebra, $a \land b = 0$.
\end{lemma}

\begin{lemma}[\cite{givant2008introduction}]
  All finite Boolean algebras are atomic.
\end{lemma}

\begin{theorem}[\cite{ganesh2006introduction}] \label{thm:representation}
  Let $\mathbf{B}$ be a finite Boolean algebra. Then every $a \in \mathbf{B}
  \setminus \{ 0 \}$ can be uniquely expressed as $a = \bigvee_{i \in I} m_i$
  for some set of atoms $\{m_i\}_{i \in I}$.
\end{theorem}

\begin{definition}[\cite{gaifman1964concerning}] \label{def:measure}
  A \emph{(strictly positive) measure} on a Boolean algebra $\mathbf{B}$ is a
  function $m : \mathbf{B} \to [0, 1]$ such that:
  \begin{enumerate}
  \item $m(1) = 1$, and $m(x) > 0$ for $x \ne
    0$; \label[property]{property:values}
  \item $m(x \lor y) = m(x) + m(y)$ for all $x, y \in \mathbf{B}$ whenever $x
    \land y = 0$. \label[property]{property:additive}
  \end{enumerate}
\end{definition}

\subsection{New Results}

\begin{definition}
  Let $\mathbf{B}$ be a finite Boolean algebra, let $L$ be a subset of
  $\mathbf{B}$ such that every atom $m$ can be uniquely expressed as $m =
  \bigwedge_{i \in I} l_i$ for some $\{ l_i \}_{i \in I} \subseteq L$, and let
  $w : L \to \mathbb{R}_{>0}$ be arbitrary. The \emph{weighted model count}
  $\WMC : \mathbf{B} \to \mathbb{R}_{\ge 0}$ is defined as
  \[
    \WMC(a) = \begin{cases}
      0 & \text{if } a = 0 \\
      \prod_{i \in I} w(l_i) & \text{if } M \ni a = \bigwedge_{i \in I} l_i \\
      \sum_{i \in I} \WMC(m_i) & \text{if } \mathbf{B} \setminus (M \cup \{ 0
      \}) \ni a = \bigvee_{i \in I} m_i
    \end{cases}
  \]
  for any $a \in \mathbf{B}$. Furthermore, we define the \emph{normalised
    weighted model count} $\nWMC : \mathbf{B} \to [0, 1]$ as $\nWMC(a) =
  \frac{\WMC(a)}{\WMC(1)}$ for all $a \in \mathbf{B}$.
\end{definition}

\begin{proposition}
  $\nWMC$ is a measure for any finite Boolean algebra $\mathbf{B}$.
\end{proposition}
\begin{proof}
  First, note that \cref{property:values} of \cref{def:measure} is satisfied by
  the definition of $\nWMC$. Next, in order to prove \cref{property:additive},
  let $x, y \in \mathbf{B}$ be such that $x \land y = 0$. We want to show that
  \[
    \nWMC(x \lor y) = \nWMC(x) + \nWMC(y)
  \]
  which is equivalent to
  \begin{equation} \label{eq:additivity_proof}
    \WMC(x \lor y) = \WMC(x) + \WMC(y).
  \end{equation}
  If, say, $x = 0$, then \cref{eq:additivity_proof} becomes
  \[
    \WMC(y) = \WMC(0) + \WMC(y) = \WMC(y)
  \]
  (and likewise for $y = 0$). Thus we can assume that $x \ne 0 \ne y$ and use
  \cref{thm:representation} to write
  \[
    x = \bigvee_{i \in I} x_i \quad \text{and} \quad y = \bigvee_{j \in J} y_j
  \]
  for some sequences of atoms $(x_i)_{i \in I}$ and $(y_j)_{j \in J}$. If
  $x_{i'} = y_{j'}$ for some $i' \in I$ and $j' \in J$, then
  \[
    x \land y = \bigvee_{i \in I} \bigvee_{j \in J} x_i \land y_j = x_{i'} \land
    y_{j'} \ne 0,
  \]
  contradicting the assumption. This is enough to show that
  \begin{align*}
    \WMC(x \lor y) &= \WMC\left( \left( \bigvee_{i \in I} x_i \right) \lor \left(\bigvee_{j \in J} y_j \right) \right) = \sum_{i \in I} \WMC(x_i) + \sum_{j \in J} \WMC(y_j) \\
                   &= \WMC(x) + \WMC(y),
  \end{align*}
  finishing the proof.
\end{proof}

\section{First-Order Logic and Polyadic Algebras}

\subsection{Preliminaries}

What follows is a summary of \cite{halmos2016algebraic}.

% Generic stuff
Let $\mathbf{B}$ be a Boolean algebra (of propositions). Let $X$ be the
(non-empty) domain of discourse. Let $I$ be an index set, elements of which can
be interpreted as variables. The elements of $X^I$ are functions from $I$ to
$X$. For any $x \in X^I$ and $i \in I$, we write $x_i$ to represent $x(i) \in
X$. Let $\mathbf{A^*}$ be the set of all functions $X^I \to \mathbf{B}$, and
note that it forms a Boolean algebra with operations defined pointwise.

% Defining S
Let $T$ be the semigroup of all $I \to I$ transformations. For any $\tau \in
T$, let $\tau_* : X^I \to X^I$ be defined by
\[
  (\tau_* x)_i = x_{\tau i}
\]
for all $x \in X^I$ and $i \in I$. We can then define $\mathbf{S}$ to be a map
from $T$ to Boolean endomorphisms of $\mathbf{A^*}$ defined by
\[
  \mathbf{S}(\tau)p(x) = p(\tau_* x)
\]
for all $x \in X^I$ and $p \in \mathbf{A^*}$.

% Defining E
For any $J \subseteq I$, let $J_*$ be the relation on $X_I$ defined by
\[
  xJ_*y \quad \iff \quad x_i = y_i \quad \text{for all } i \in I \setminus J
\]
for all $x, y \in X^I$. For any $J \subseteq I$, we then define $\bm\exists(J)$
to be a transformation $\mathbf{A^*} \to \mathbf{A^*}$ defined by
\[
  \bm\exists(J)p(x) = \bigvee_{\substack{y \in X^I,\\ xJ_*y}} p(y)
\]
for all $p \in \mathbf{A^*}$, provided this supremum exists for all $x \in
X^I$\footnote{The universal quantifier $\bm\forall(J)$ is then defined as
  $\bm\forall(J)p = \neg(\bm\exists(J)\neg p)$ for all $p \in \mathbf{A^*}$.}.

Finally, a \emph{functional polyadic (Boolean) algebra}\footnote{To be more
  explicit, a $\mathbf{B}$-valued functional polyadic algebra with domain $X$
  and variables $I$.} is a subalgebra
$\mathbf{A}$ of $\mathbf{A^*}$ such that:
\begin{itemize}
\item $\mathbf{S}(\tau)p \in \mathbf{A}$ for all $p \in \mathbf{A}$ and $\tau
  \in T$;
\item $\bm\exists(J)p \in \mathbf{A}$ for all $p \in \mathbf{A}$ and $J
  \subseteq I$.
\end{itemize}

\begin{definition}
  Similarly to $\bm\exists$, a \emph{constant} $c$ is a mapping from
  $\mathcal{P}(I)$ to Boolean endomorphisms of $\mathbf{A}$ such that:
  \begin{itemize}
  \item $c(\emptyset) = \id_{\mathbf{A}}$;
  \item $c(J \cup K) = c(J)c(K)$;
  \item $c(J)\bm\exists(K) = \bm\exists(K)c(J \setminus K)$;
  \item $\bm\exists(J)c(K) = c(K)\bm\exists(J \setminus K)$;
  \item $c(J)\mathbf{S}(\tau) = \mathbf{S}(\tau)c(\tau^{-1}J)$
  \end{itemize}
  for all $J, K \in \mathcal{P}(I)$ and $\tau \in T$. If $J$ is a singleton
  set $\{ i \}$, we will simply write $c(i)$ instead of $c(J)$.
\end{definition}

\subsection{New Results}

\begin{proposition}
  Let $\mathbf{B}$ be a finite Boolean algebra with a measure $m :
  \mathbf{B} \to [0, 1]$. Let $\mathbf{A}$ be a $\mathbf{B}$-valued functional
  polyadic algebra with domain $X$ and variables $I$. For any $p \in
  \mathbf{A}$, let $\sup(p) = \bigvee_{x \in X^I} p(x)$. Let $m^* : \mathbf{A}
  \to [0, 1]$ be defined by
  \[
    m^*(p) = \sum_{\substack{\text{atom } y \in \mathbf{B}:\\ y \le \sup(p)}}
    m(y)
  \]
  for all $p \in \mathbf{A}$. Then $m^*$ is a measure on $\mathbf{A}$.
\end{proposition}

\begin{remark}
  While defining $m^*$ as $m^*(p) = m(\sup(p))$ might look tempting, this
  definition is not additive.
\end{remark}

\begin{proof} % TODO: update the proof
  First, since $1 \in \mathbf{A}$ is a function $1 : X^I \to \mathbf{B}$ defined
  as $1(x) = 1$, we have that $m^*(1) = m(1) = 1$. If $m^*(p) = 0$ for some $p
  \in \mathbf{A}$, then it must be the case that
  \[
    \bigvee_{x \in X^I} p(x) = 0
  \]
  which means that for all $x \in X^I$, $p(x) = 0$, which is equivalent to
  saying that $p = 0$. Thus, $m^*(p) > 0$ for $p \in \mathbf{A} \setminus \{ 0
  \}$.
\end{proof}

% TODO: how to interpret first-order sentences in this algebra
% TODO: define WMC: connect my definition of WMC with the original one
% TODO: condition for a measurable BA to be representable as WMC
% TODO: can I show that WMC is the only possible probability measure?
% TODO: abstraction refinements as homomorphisms.
% TODO: definition of a measure-preserving homomorphism from Thomas Jech
% TODO: a Boolean algebra is approximable if its Stone space is approximable

\bibliographystyle{plain}
\bibliography{paper}

\end{document}