\documentclass{article}
\usepackage[utf8]{inputenc}
\usepackage[UKenglish]{babel}
\usepackage[UKenglish]{isodate}
\usepackage{fullpage}
\usepackage{amsthm}
\usepackage{amsfonts}
\usepackage{amsmath}
\usepackage{mathtools}
\usepackage[capitalise]{cleveref}
\usepackage{bm}
\usepackage{booktabs}
\usepackage{tikz}
\usepackage{xcolor}
\usepackage[backgroundcolor=lightgray]{todonotes}

\newtheorem{theorem}{Theorem}
\newtheorem{lemma}{Lemma}
\newtheorem{proposition}{Proposition}
\newtheorem{conjecture}{Conjecture}
\theoremstyle{definition}
\newtheorem{definition}{Definition}
\newtheorem{example}{Example}
\theoremstyle{remark}
\newtheorem*{remark}{Remark}

\definecolor{color1}{HTML}{fbb4ae}
\definecolor{color2}{HTML}{b3cde3}
\definecolor{color3}{HTML}{ccebc5}
\definecolor{color4}{HTML}{decbe4}

\Crefname{property}{Property}{Properties}
\Crefname{condition}{Condition}{Conditions}
\creflabelformat{condition}{#2(#1)#3}

\DeclareMathOperator{\WMC}{WMC}
\DeclareMathOperator{\nWMC}{NWMC}
\DeclareMathOperator{\id}{id}
\DeclareMathOperator{\End}{End}

\usetikzlibrary{cd}

\tikzset{
  Subset/.style={
    draw=none,
    every to/.append style={
      edge node={node [sloped, allow upside down, auto=false]{$\subset$}}}
  }
}

%\title{On the Limitations of Weighted Model Counting}
\title{Weighted Model Counting/Integration from the Perspective of Boolean Algebras}
%\title{What Boolean Algebras Can Teach Us About Weighted Model Counting/Integration?}
\author{Paulius Dilkas}

\begin{document}
\maketitle

\section{Introduction}
% Feedback: What are the main claims, what are the main takeaways, intuitive
% [???] of theorems to follow. To do this, we appeal to algebraic constructions
% to define the main concepts for introducing measures on Boolean algebras.

% Contributions:
% 1. A generalisation of WMC adapted to Boolean algebras.
% 2. We show that it is a valid measure.
% 3. We show that WMC assumes independence, i.e., cannot represent all measures.
% 4. We show how the BA can be extended with new 'literals' to represent any measure.
% 5. (Maybe) we establish a lower bound on the number of new literals, if we
% make no assumptions about independence.
% 6. (I guess I could) show how evidence = quotiening by a filter, and how
% quotiening by an ideal works
% 7. (WMI) If all weights are 1, we extend Boolean variables to discrete
% variables, and continuous variables are arbitrary subsets (in the
% sigma-algebra), then the WMI process calculates the probability of any event
% and doesn't require going down to the level of models (assuming a suitable
% probability measure).
% 8. If the weight of a literal (and its negation) is 1, then we don't need to
% look at its models--the probability can be calculated at any level. This
% extends to any number of variables.
% 9. How WMI transforms to our algebras. The boring version and what's wrong
% with it.

Previous/related work:
\begin{itemize}
\item Hailperin's approach to probability logic
  \cite{DBLP:journals/ndjfl/Hailperin84}
\item Nilsson's (somewhat successful) probabilistic logic
  \cite{DBLP:journals/ai/Nilsson86}
\item Semiring programming \cite{DBLP:journals/corr/BelleR16}
\item WMC with functions \cite{DBLP:conf/uai/Belle17}
\item WMI \cite{DBLP:conf/ijcai/BellePB15}
\item Measures on Boolean algebras: overview articles (from most cited to least
  cited)
  \begin{itemize}
  \item Horn and Tarski \cite{horn1948measures}
  \item Concerning measures on Boolean algebras \cite{gaifman1964concerning}
  \item Jech -- Measures on Boolean algebras (arXiv) \cite{jech2017measures}
  \end{itemize}
\item Measures on Boolean algebras: more specific articles
  \begin{itemize}
  \item On possibility and probability measures in finite Boolean algebras
    \cite{DBLP:journals/soco/CastineiraCT02}
  \item Representation of conditional probability measures
    \cite{krauss1968representation}
  \end{itemize}
\end{itemize}

\section{Preliminaries}

\begin{definition} \label{def:ba}
  A \emph{Boolean algebra} (BA) is a tuple $(\mathbf{B}, \land, \lor, \neg, 0,
  1)$ consisting of a set $\mathbf{B}$ with binary operations \emph{meet}
  $\land$ and \emph{join} $\lor$, unary operation $\neg$ and elements $0, 1 \in
  \mathbf{B}$ such that the following axioms hold for all $a, b, \in
  \mathbf{B}$:
  \begin{itemize}
  \item both $\land$ and $\lor$ are associative and commutative;
  \item $a \lor (a \land b) = a$, and $a \land (a \lor b) = a$;
  \item $0$ is the identity of $\lor$, and $1$ is the identity of $\land$;
  \item $\lor$ distributes over $\land$ and vice versa;
  \item $a \lor \neg a = 1$, and $a \land \neg a = 0$.
  \end{itemize}
\end{definition}

For clarity and succinctness, we will occasionally use three other operations
that can be defined using the original three\footnote{We use $+$ to denote
  symmetric difference because it is the additive operation of a Boolean ring.}:
\begin{align*}
  a \to b &= \neg a \lor b, \\
  a \leftrightarrow b &= (a \to b) \land (b \to a), \\
  a + b &= (a \land \neg b) \lor (\neg a \land b).
\end{align*}
We can also define a partial order $\le$ on $\mathbf{B}$ as $a \le b$ if $a = b
\land a$ (or, equivalently, $a \lor b = b$) for $a, b \in \mathbf{B}$.
Furthermore, let $a < b$ denote $a \le b$ and $a \ne b$. For the rest of this
paper, let $\mathbf{B}$ refer to the BA $(\mathbf{B}, \land, \lor, \neg, 0, 1)$.
For any $S \subseteq \mathbf{B}$, we write $\bigvee S$ for $\bigvee_{x \in S} x$
and call it the \emph{supremum} of $S$. Similarly, $\bigwedge S = \bigwedge_{x
  \in S} x$ is the \emph{infimum}. By convention, $\bigwedge \emptyset = 1$ and
$\bigvee \emptyset = 0$.

\begin{definition}[\cite{DBLP:books/daglib/0090259,levasseur2012applied}]
  An element $a \ne 0$ of $\mathbf{B}$ is an \emph{atom} if, for all $x \in
  \mathbf{B}$, either $x \land a = a$ or $x \land a = 0$. Equivalently, $a \ne
  0$ is an atom if there is no $x \in \mathbf{B}$ such that $0 < x < a$. A BA
  $\mathbf{B}$ is \emph{atomic} if for every $a \in \mathbf{B} \setminus \{0
  \}$, there is an atom $x$ such that $x \le a$.
\end{definition}

\begin{lemma}[\cite{ganesh2006introduction}]
  For any two distinct atoms $a$, $b \in \mathbf{B}$, $a \land b = 0$.
\end{lemma}

\begin{lemma}[\cite{givant2008introduction}] \label{thm:representation}
  The following are equivalent:
  \begin{itemize}
  \item $\mathbf{B}$ is atomic.
  \item For any $x \in \mathbf{B}$,
    \[
      x = \bigvee_{\text{atoms } a \le x} a.
    \]
  \item $1$ is the supremum of all atoms.
  \end{itemize}
\end{lemma}

\begin{lemma}[\cite{givant2008introduction}] \label{lemma:atomic}
  All finite BAs are atomic.
\end{lemma}

\begin{definition}[\cite{gaifman1964concerning,DBLP:books/daglib/0090259}] \label{def:measure}
  A \emph{measure} on $\mathbf{B}$ is a function $m\colon
  \mathbf{B} \to \mathbb{R}_{\ge 0}$ such that:
  \begin{itemize}
  \item $m(0) = 0$;
  \item $m(x \lor y) = m(x) + m(y)$ for all $x, y \in \mathbf{B}$ whenever $x
    \land y = 0$.
  \end{itemize}
  If $m(1) = 1$, we call $m$ a \emph{probability measure}. Also, if $m(x) > 0$
  for all $x \ne 0$, then $m$ is \emph{strictly positive}.
\end{definition}

\section{WMC as a Measure}

\begin{definition} \label{def:algebra_from_logic}
  Let $\mathcal{L}$ be a propositional (or first-order) logic, and let
  $\Delta$ be a theory in $\mathcal{L}$. We can define an equivalence
  relation on formulas in $\mathcal{L}$ as
  \[
    \alpha \sim \beta \quad \text{if and only if} \quad \Delta \vdash \alpha
    \leftrightarrow \beta
  \]
  for all $\alpha, \beta \in \mathcal{L}$. Let $[\alpha]$ denote the equivalence
  class of $\alpha \in \mathcal{L}$ with respect to $\sim$. We can then let
  $B(\Delta) = \{ [\alpha] \mid \alpha \in \mathcal{L} \}$ and define the
  structure of a BA on $B(\Delta)$ as
  \begin{align*}
    [\alpha] \lor [\beta] &= [\alpha \lor \beta], \\
    [\alpha] \land [\beta] &= [\alpha \land \beta], \\
    \neg[\alpha] &= [\neg\alpha], \\
    1 &= [\alpha \to \alpha], \\
    0 &= [\alpha \land \neg\alpha]
  \end{align*}
  for all $\alpha, \beta \in \mathcal{L}$. Then $B(\Delta)$ is the
  \emph{Lindenbaum-Tarski algebra} of $\Delta$
  \cite{koppelberg1989handbook,tarski1983logic}.
\end{definition}

\begin{figure}
  \[
    \begin{tikzcd}
      & & & & & \colorbox{color3}{$1$} \ar[dlll,dash,gray] \ar[dl,dash,gray]
      \ar[dr,dash,gray] \ar[drrr,dash,gray] & & & \\
      & & \colorbox{color3}{$p \lor q$} & & \colorbox{color3}{$q \to p$} & &
      \colorbox{color2}{$p \to q$} & & \colorbox{color4}{$\neg p \lor \neg q$}
      \\
      \colorbox{color3}{$p$} \ar[urr,dash,gray] \ar[urrrr,dash,gray] & &
      \colorbox{color2}{$q$} \ar[u,dash,gray] \ar[urrrr,dash,gray] & &
      \colorbox{color2}{$p \leftrightarrow q$} \ar[u,dash,gray]
      \ar[urr,dash,gray] & & \colorbox{color4}{$p + q$} \ar[ullll,dash,gray]
      \ar[urr,dash,gray] & & \colorbox{color4}{$\neg q$} \ar[ullll,dash,gray]
      \ar[u,dash,gray] & & \colorbox{color1}{$\neg p$} \ar[ullll,dash,gray]
      \ar[ull,dash,gray] \\
      & & \fcolorbox{black}{color2}{$p \land q$} \ar[ull,dash,gray]
      \ar[u,dash,gray] \ar[urr,dash,gray] & & \fcolorbox{black}{color4}{$p \land
        \neg q$} \ar[ullll,dash,gray] \ar[urr,dash,gray] \ar[urrrr,dash,gray] &
      & \fcolorbox{black}{color1}{$\neg p \land q$} \ar[ullll,dash,gray]
      \ar[u,dash,gray] \ar[urrrr,dash,gray] & & \fcolorbox{black}{color1}{$\neg
        p \land \neg q$} \ar[ullll,dash,gray] \ar[u,dash,gray]
      \ar[urr,dash,gray] \\
      & & & & & \colorbox{color1}{$0$} \ar[ulll,dash,gray] \ar[ul,dash,gray]
      \ar[ur,dash,gray] \ar[urrr,dash,gray] & & &
    \end{tikzcd}
  \]
  \[
    \begin{tikzcd}
      & \colorbox{color3}{$\left[1\right]$} \ar[dl,dash,gray] \ar[dr,dash,gray]
      & \\
      \fcolorbox{black}{color2}{$\left[q\right]$} & &
      \fcolorbox{black}{color4}{$\left[\neg q\right]$} \\
      & \colorbox{color1}{$\left[0\right]$} \ar[ul,dash,gray] \ar[ur,dash,gray]
      &
    \end{tikzcd}
  \]
  \caption{Two BAs from \cref{example:construction}: $B(\mathcal{L})$ at the top
    and $B(\Delta)$ at the bottom. An edge between elements $a$ and $b$ (with
    $a$ positioned lower than $b$) means that $a < b$. Each element of
    $B(\Delta)$ is an equivalence class of elements of $B(\mathcal{L})$, and the
    colours show which elements of $B(\mathcal{L})$ belong to which class. In
    both algebras, atoms have borders around them.}
  \label{fig:example}
\end{figure}

\begin{example} \label{example:construction}
  Let $\mathcal{L}$ be a propositional logic with $p$ and $q$ as its only atoms.
  Then $L = \{ p, q, \neg p, \neg q \}$ is its set of literals. Let $w : L \to
  \mathbb{R}_{\ge 0}$ be the \emph{weight function} defined by
  \begin{align*}
    w(p) = 0.3, \\
    w(\neg p) = 0.7, \\
    w(q) = 0.2, \\
    w(\neg q) = 0.8.
  \end{align*}
  Let $\Delta$ be a theory in $\mathcal{L}$ with a sole axiom $p$. Then
  $\Delta$ has two models, i.e., $\{ p, q \}$ and $\{ p, \neg q \}$. The
  \emph{weighted model count} (WMC) \cite{DBLP:journals/ai/ChaviraD08} of $\Delta$ is
  then
  \[
    \sum_{\omega \models \Delta} \prod_{\omega \models l} w(l) =
    w(p)w(q) + w(p)w(\neg q) = 0.3.
  \]

  The corresponding BA $B(\Delta)$ can then be constructed using
  \cref{def:algebra_from_logic}. Alternatively, one can first construct the free
  BA generated by the set $\{ p, q \}$---this corresponds to $B(\mathcal{L})$ in
  \cref{fig:example}---and then take a quotient with respect to either the
  filter generated by $p$ or the ideal\footnote{More details on these concepts
    can be found in many books on BAs
    \cite{givant2008introduction,koppelberg1989handbook}.} generated by $\neg
  p$. In any case, the resulting BA is pictured at the bottom of
  \cref{fig:example}.

  Each element of $B(\mathcal{L})$ can also be seen as a subset of the set of
  all models of $\mathcal{L}$, with $0$ representing $\emptyset$, $1$
  representing the set of all (four) models, each atom representing a single
  model, and each edge going upward representing a subset relation. Thus,
  the Boolean-algebraic way of calculating the WMC of $\Delta$ consists of:
  \begin{enumerate} % TODO: why is step 1 always possible?
  \item Identifying an element $a \in B(\mathcal{L})$ that corresponds to
    $\Delta$.
  \item Finding all atoms of $B(\mathcal{L})$ that are `dominated' by $a$
    according to the partial order.
  \item Using $w$ to calculate the weight of each such atom.
  \item Adding the weights of these atoms.
  \end{enumerate}
  This motivates the following definition of WMC generalised to BAs.
\end{example}
% TODO: how to compute the number of elements in the algebra.
% TODO: clarify what B(L) means. And whether B(Delta) is even necessary.
% TODO: reference for the set/subset thing.

\begin{definition} \label{def:wmc}
  Let $\mathbf{B}$ be an atomic BA, and let $M \subset \mathbf{B}$ be its set of
  atoms. Let $L \subset \mathbf{B}$ be such that every atom $m \in M$ can be
  uniquely expressed as $m = \bigwedge L'$ for some $L' \subseteq L$, and let
  $w\colon L \to \mathbb{R}_{\ge 0}$ be arbitrary. The \emph{weighted model
    count} $\WMC_w\colon \mathbf{B} \to \mathbb{R}_{\ge 0}$ is defined as
  \[
    \WMC_w(x) = \begin{cases}
      0 & \text{if } x = 0 \\
      \prod_{l \in L'} w(l) & \text{if } M \ni x = \bigwedge L' \\
      \sum_{\text{atoms } a \le x} \WMC_w(a) & \text{otherwise}
    \end{cases}
  \]
  for any $x \in \mathbf{B}$. Furthermore, we define the \emph{normalised
    weighted model count} $\nWMC_w\colon \mathbf{B} \to [0, 1]$ as $\nWMC_w(x) =
  \frac{\WMC_w(x)}{\WMC_w(1)}$ for all $x \in \mathbf{B}$. For both $\WMC_w$ and
  $\nWMC_w$, we will drop the subscript when doing so results in no potential
  confusion.
\end{definition}

% TODO: Feedback: clarify the relationship between NWMC and the probability
% induced by a WMC formulation, i.e., Pr_{\Delta, w}(q) = \frac{WMC(\Delta \land
% q, w)}{WMC(\Delta, w)}.

\begin{proposition}
  $\WMC$ is a measure and $\nWMC$ is a probability measure.
\end{proposition}
\begin{proof}
  First, note that $\WMC$ is non-negative and $\WMC(0) = 0$ by definition. Next,
  let $x, y \in \mathbf{B}$ be such that $x \land y = 0$. We want to show that
  \begin{equation} \label{eq:additivity_proof}
    \WMC(x \lor y) = \WMC(x) + \WMC(y).
  \end{equation}
  If, say, $x = 0$, then \cref{eq:additivity_proof} becomes
  \[
    \WMC(y) = \WMC(0) + \WMC(y) = \WMC(y)
  \]
  (and likewise for $y = 0$). Thus we can assume that $x \ne 0 \ne y$ and use
  \cref{thm:representation} to write
  \[
    x = \bigvee_{i \in I} x_i \quad \text{and} \quad y = \bigvee_{j \in J} y_j
  \]
  for some sequences of atoms $(x_i)_{i \in I}$ and $(y_j)_{j \in J}$. If
  $x_{i'} = y_{j'}$ for some $i' \in I$ and $j' \in J$, then
  \[
    x \land y = \bigvee_{i \in I} \bigvee_{j \in J} x_i \land y_j = x_{i'} \land
    y_{j'} \ne 0,
  \]
  contradicting the assumption. This is enough to show that
  \begin{align*}
    \WMC(x \lor y) &= \WMC\left( \left( \bigvee_{i \in I} x_i \right) \lor \left(\bigvee_{j \in J} y_j \right) \right) = \sum_{i \in I} \WMC(x_i) + \sum_{j \in J} \WMC(y_j) \\
                   &= \WMC(x) + \WMC(y),
  \end{align*}
  finishing the proof that $\WMC$ is a measure. This immediately shows that
  $\nWMC$ is a probability measure since, by definition, $\nWMC(1) = 1$.
\end{proof}

% TODO: Feedback: Can you say something here about factorized vs non-factorized
% weight function definitions? That is, factorized is when w maps literals to
% R_>=0, non-factorized is when w maps models to R_>=0 and show:
% a) come up with nice example when non-factorized weights are intuitive
% b) what if weight functions are negative/complex?
% c) clarify that the factorized definition have is w.r.t. models, in case some
% one gets confused [It doesn't have to be, if the BA is not free -- P.]
% d) can you say something about WMI

\section{What Measures Are WMC-Computable?}

\subsection{WMC Requires Independent Literals}

% TODO: maybe I should gives this kind of a BA a name? A synonym of 'complete',
% perhaps.
% TODO: a special case for weight=0.

\begin{proposition}
  Let $\mathbf{B}$ be a finite measure algebra with measure $m\colon \mathbf{B} \to
  \mathbb{R}_{\ge 0}$. Let $L \subset \mathbf{B}$ be defined as
  \[
  L = \{ l_i \mid i \in [n] \} \cup \{ \neg l_i \mid i \in [n] \}
  \]
  for some $n \in \mathbb{N}$. Finally, assume that $\mathbf{B}$ has $2^n$
  atoms, where each atom $a \in \mathbf{B}$ is an infimum
  \[
    a = \bigwedge_{i=1}^n a_i
  \]
  such that $a_i \in \{ l_i, \neg l_i \}$ for $i \in [n]$. Then there exists a
  weight function $w\colon L \to \mathbb{R}_{\ge 0}$ that makes $m$ a WMC measure if
  and only if
  \begin{equation} \label{eq:wmccondition}
  m(l \land l') = m(l)m(l')
  \end{equation}
  for all distinct $l, l' \in L$ such that $l \ne \neg l'$.
\end{proposition}

\begin{remark}
  Note that if $n = 1$, then \cref{eq:wmccondition} is vacuously satisfied and
  so any valid measure can be expressed as WMC.
\end{remark}

\begin{proof}
  Let us begin with the `if' part of the statement. Let $w\colon L \to
  \mathbb{R}_{\ge 0}$ be defined by
  \begin{equation} \label{eq:assumption}
    w(l) = m(l)
  \end{equation}
  for all $l \in L$. We are going
  to show that $\nWMC = m$. First, note that $\nWMC(0) = 0 = m(0)$ by the
  definitions of both $\nWMC$ and $m$. Second, let
  \begin{equation} \label{eq:def_of_a}
    a = \bigwedge_{i=1}^n a_i
  \end{equation}
  be an atom in $\mathbf{B}$ such that $a_i \in \{ l_i, \neg l_i \}$ for all $i
  \in [n]$. Then
  \[
    \nWMC(a) = \frac{\WMC(a)}{\WMC(1)} = \frac{1}{\WMC(1)} \prod_{i=1}^n w(a_i)
    = \frac{1}{\WMC(1)} \prod_{i=1}^n m(a_i) = \frac{1}{\WMC(1)} m \left(
      \bigwedge_{i=1}^n a_i \right) = \frac{m(a)}{\WMC(1)}
  \]
  by \cref{def:wmc,eq:assumption,eq:wmccondition,eq:def_of_a}. Now we just need
  to show that $\WMC(1) = 1$. Indeed,
  \begin{align*}
    \WMC(1) &= \sum_{\text{atoms } a \in \mathbf{B}} \WMC(a) = \sum_{\text{atoms
      } a \in \mathbf{B}} \prod_{i=1}^n w(a_i) = \sum_{\text{atoms } a \in
      \mathbf{B}} \prod_{i=1}^n m(a_i) \\
    &= \sum_{\text{atoms } a \in
      \mathbf{B}} m \left( \bigwedge_{i=1}^n a_i \right) = \sum_{\text{atoms } a
      \in \mathbf{B}} m(a) = m \left( \bigvee_{\text{atoms } a \in \mathbf{B}}
    \right) = m(1) = 1.
  \end{align*}
  Finally, note that if $\nWMC$ and $m$ agree on all atoms, then they must also
  agree on all other non-zero elements of the Boolean algebra.

  For the other direction, we are given a weight function $w\colon L \to
  \mathbb{R}_{\ge 0}$ that induces a measure $m = \nWMC\colon \mathbf{B} \to
  \mathbb{R}_{\ge 0}$, and we want to show that \cref{eq:wmccondition} is
  satisfied. Let $k_i, k_j \in L$ be such that $k_i \in \{ l_i, \neg l_i \}$,
  $k_j \in \{ l_j, \neg l_j \}$, and $i \ne j$. We will first prove an auxiliary
  result that
  \begin{equation} \label{eq:to_prove}
    m(k_i \land k_j) = m(k_i)m(k_j)
  \end{equation}
  is equivalent to
  \begin{equation} \label{eq:to_prove2}
    m(k_i \land k_j) \cdot m(\neg k_i \land \neg k_j) = m(k_i \land \neg k_j)
    \cdot m(\neg k_i \land k_j).
  \end{equation}
  First, note that $k_i$ can be expressed as
  \[
    k_i = (k_i \land k_j) \lor (k_i \land \neg k_j)
  \]
  with the condition that
  \[
    (k_i \land k_j) \land (k_i \land \neg k_j) = 0,
  \]
  so, by properties of a measure,
  \begin{equation} \label{eq:temp}
    m(k_i) = m(k_i \land k_j) + m(k_i \land \neg k_j).
  \end{equation}
  Applying \cref{eq:temp} and the equivalent expression for $m(k_j)$ allows us
  to rewrite \cref{eq:to_prove} as
  \begin{align*}
    m(k_i \land k_j) &= [m(k_i \land k_j) + m(k_i \land \neg k_j)] \cdot [m(k_i \land k_j) + m(\neg k_i \land k_j)] \\
                     &= m(k_i \land k_j)^2 + m(k_i \land k_j)[m(k_i \land \neg k_j) + m(\neg k_i \land k_j)] + m(k_i \land \neg k_j)m(\neg k_i \land k_j)
  \end{align*}
  Dividing both sides by $m(k_i \land k_j)$ gives
  \begin{equation} \label{eq:temp2}
    1 = m(k_i \land k_j) + m(k_i \land \neg k_j) + m(\neg k_i \land k_j) +
    \frac{m(k_i \land \neg k_j)m(\neg k_i \land k_j)}{m(k_i \land k_j)}.
  \end{equation}
  Since $k_i \land k_j \land k_i \land \neg k_j = 0$, and
  \[
    (k_i \land k_j) \lor (k_i \land \neg k_j) = k_i \land (k_j \lor \neg k_j) =
    k_i \land 1 = k_i,
  \]
  we have that
  \[
    m(k_i \land k_j) + m(k_i \land \neg k_j) = m(k_i).
  \]
  Similarly, $k_i \land \neg k_i \land k_j = 0$, and
  \[
    k_i \lor (\neg k_i \land k_j) = (k_i \lor \neg k_i) \land (k_i \lor k_j) =
    k_i \lor k_j,
  \]
  so
  \[
    m(k_i) + m(\neg k_i \land k_j) = m(k_i \lor k_j).
  \]
  Finally, note that
  \[
    (k_i \lor k_j) \land \neg(k_i \lor k_j) = 0,
  \]
  and
  \[
    (k_i \lor k_j) \lor \neg(k_i \lor k_j) = 1,
  \]
  so
  \[
    m(k_i \lor k_j) + m(\neg(k_i \lor k_j)) = m(1) = 1.
  \]
  This allows us to rewrite \cref{eq:temp2} as
  \[
    \frac{m(k_i \land \neg k_j)m(\neg k_i \land k_j)}{m(k_i \land k_j)} = 1 -
    m(k_i \lor k_j) = m(\neg(k_i \lor k_j)) = m(\neg k_i \land \neg k_j)
  \]
  which immediately gives us \cref{eq:to_prove2}.

  Now recall that $m = \nWMC$ and note that \cref{eq:to_prove2} can be
  multiplied by $\WMC(1)^2$ to turn the equation into one for $\WMC$ instead of
  $\nWMC$. Then
  \begin{align*}
    \WMC(k_i \land k_j) &= \sum_{\text{atoms } a \le k_i \land k_j} \WMC(a) = \sum_{\text{atoms } a \le k_i \land k_j} \prod_{m \in [n]} w(a_m) \\
                        &= \sum_{\text{atoms } a \le k_i \land k_j} w(a_i)w(a_j) \prod_{m \in [n] \setminus \{ i, j \}} w(a_m) = \sum_{\text{atoms } a \le k_i \land k_j} w(k_i)w(k_j) \prod_{m \in [n] \setminus \{ i, j \}} w(a_m) \\
    &= w(k_i)w(k_j) \sum_{\text{atoms } a \le k_i \land k_j} \prod_{m \in [n] \setminus \{ i, j \}} w(a_m) = w(k_i)w(k_j)C,
  \end{align*}
  where $C$ denotes the part of $\WMC(k_i \land k_j)$ that will be the same for
  $\WMC(\neg k_i \land k_j)$, $\WMC(k_i \land \neg k_j)$, and $\WMC(\neg k_i
  \land \neg k_j)$ as well. But then \cref{eq:to_prove2} becomes
  \[
    w(k_i)w(k_j)w(\neg k_i)w(\neg k_j)C^2 = w(k_i)w(\neg k_j)w(\neg k_i)w(k_j)C^2
  \]
  which is trivially true. By showing that WMC satisfies \cref{eq:to_prove2}, we
  also showed that it satisfies \cref{eq:to_prove}, finishing the second part of
  the proof.
\end{proof}
% TODO: the auxiliary result should be a 'claim' in-between the theorem and the proof.

\subsection{Extending the Algebra}

A well-known way to overcome this limitation of independence is by adding more
literals \cite{DBLP:journals/ai/ChaviraD08}, i.e., extending the set $L$ covered
by the WMC weight function $w\colon L \to \mathbb{R}_{\ge 0}$. Let us translate this
idea to the language of Boolean algebras.

\begin{theorem} \label{thm:extension} % TODO: cite the fact about atoms
  Let $\mathbf{B}$ be a finite Boolean algebra freely generated by some set of
  `literals' $L$, and let $m\colon \mathbf{B} \to \mathbb{R}_{\ge 0}$ be an
  arbitrary measure. We know that $\mathbf{B}$ has $n = 2^{|L|}$ atoms. Let
  $(a_i)_{i=1}^n$ denote those atoms in some arbitrary order. Let $L' = L \cup
  \{ \phi_i \mid i \in [n] \} \cup \{ \neg \phi_i \mid i \in [n] \}$ be the set
  $L$ extended with $2n$ new literals. Let $\mathbf{B'}$ be the unique Boolean
  algebra with
  \[
    \{ \phi_i \land a_i \mid i \in [n] \} \cup \{ \neg \phi_i \land a_i \mid i
    \in [n] \}
  \]
  as its set of atoms. Let $\iota\colon \mathbf{B} \to \mathbf{B'}$ be the inclusion
  homomorphism (i.e., $\iota(a) = a$ for all $a \in \mathbf{B}$). Let $w\colon L'
  \to \mathbb{R}_{\ge 0}$ be defined by
  \[
    w(l) = \begin{cases}
      \frac{m(a_i)}{2} & \text{if } l = \phi_i \text{ or } l = \neg\phi_i \text{
        for some } i \in [n] \\
      1 & \text{otherwise}
    \end{cases}
  \]
  for all $l \in L'$, and note that this defines a WMC measure $m'\colon \mathbf{B'}
  \to \mathbb{R}_{\ge 0}$. Then
  \[
    m(a) = (m' \circ \iota)(a)
  \]
  for all $a \in \mathbf{B}$.
\end{theorem}

In simpler terms, any measure can be computed using WMC by extending the Boolean
algebra with more literals. More precisely, we are given the red part in
\[
  \begin{tikzcd}
    \textcolor{red}{\mathbb{R}_{\ge 0}} & & \\
    \textcolor{red}{\mathbf{B}} \arrow[red]{u}{m} \arrow{r}{\iota} &
    \mathbf{B'} \arrow{lu}[swap]{m'} & \\
    \textcolor{red}{L} \arrow[Subset,red]{u}{} \arrow[Subset]{r}{} & L'
    \arrow[Subset]{u}{} \arrow{r}{w} & \mathbb{R}_{\ge 0}
  \end{tikzcd}
\]
and construct the black part in such a way that the triangle commutes.

% TODO: make J depend on i
\begin{proof} % TODO: find a reference for this first claim
  Since $\mathbf{B}$ is freely generated by $L$, each atom $a_i \in \mathbf{B}$
  is an infimum of elements in $L$, i.e.,
  \[
    a_i = \bigwedge_{j \in J} a_{i,j}
  \]
  for some $\{ a_{i,j} \}_{j \in J} \subset L$. Moreover, each atom $b \in
  \mathbf{B'}$ can be represented as either
  \[
    b = \phi_i \land a_i \quad \text{or} \quad b = \neg\phi_i \land a_i
  \]
  for some atom $a_i \in \mathbf{B}$, also making it an infimum over a subset of
  $L'$. Then, for any $b \in \mathbf{B}$,
  \[
    (m' \circ \iota)(b) = \sum_{\substack{\text{atoms } a_i \in \mathbf{B}:\\
        \phi_i \land a_i \le \iota(b)}} (w(\phi_i) + w(\neg\phi_i)) \prod_{j \in
    J} w(a_{i,j}),
  \]
  recognising that, for any $\iota(b)$, any atom $a_i \in \mathbf{B}$ satisfies
  \[
    \phi_i \land a_i \le \iota(b)
  \]
  if and only if it satisfies
  \[
    \neg\phi_i \land a_i \le \iota(b).
  \]
  Then, according to the definition of $w$,
  \[
    (m' \circ \iota)(b) = \sum_{\substack{\text{atoms } a_i \in \mathbf{B}:\\
        \phi_i \land a_i \le \iota(b)}} (w(\phi_i) + w(\neg\phi_i)) =
    \sum_{\substack{\text{atoms } a_i \in \mathbf{B}:\\ \phi_i \land a_i \le
        \iota(b)}} m(a_i) = m(b),
  \]
  provided that
  \[
    \phi_i \land a_i \le \iota(b) \quad \text{if and only if} \quad a_i \le b,
  \]
  but this is equivalent to
  \[
    \phi_i \land a_i = \phi_i \land a_i \land b \quad \text{if and only if}
    \quad a_i = a_i \land b
  \]
  which is true because $\phi_i \not\in L$.
\end{proof}

Now we can show that the construction in \cref{thm:extension} is smallest
possible.

\begin{conjecture}
  Let $\mathbf{B}$ and $\mathbf{B'}$ be Boolean algebras, and $\iota\colon
  \mathbf{B} \to \mathbf{B'}$ be the inclusion map such that $\mathbf{B}$ is
  freely generated by $L$, all atoms of $\mathbf{B'}$ can be expressed as
  meets of elements of $L'$, and the following subset relations are satisfied:
  \[
    \begin{tikzcd}
      \mathbf{B} \arrow{r}{\iota} & \mathbf{B'} \\
      L \arrow[Subset]{u}{} \arrow[Subset]{r}{} & L' \arrow[Subset]{u}{}
    \end{tikzcd}
  \]
  If, for any measure $m\colon \mathbf{B} \to \mathbb{R}_{\ge 0}$, one can
  construct a weight function $w\colon L' \to \mathbb{R}_{\ge 0}$ such that the WMC
  measure $\WMC\colon \mathbf{B'} \to \mathbb{R}_{\ge 0}$ with respect to $w$
  satisfies
  \[
    m = \WMC \circ \iota,
  \]
  then $|L' \setminus L| \ge 2^{|L|+1}$.
\end{conjecture}
% \begin{proof}
%   % 1. An atom in B' must have more than just elements of L.
%   Let $a$ be an atom in $\mathbf{B}$, and let $b$ be an atom in $\mathbf{B'}$
%   such that $b \le a$. First, let us notice that as long as $|L| \ge
%   4$\footnote{Note that $|L|$ has to be an even number.}, $b \ne a$. Indeed, let
%   $p, r, \neg p, \neg r \in L$. Then
%   \begin{align*}
%     (\WMC \circ \iota)(p \land r) &= w(p)w(r), \\
%     (\WMC \circ \iota)(p \land \neg r) &= w(p)w(\neg r), \\
%     (\WMC \circ \iota)(\neg p \land r) &= w(\neg p)w(r), \\
%     (\WMC \circ \iota)(\neg p \land \neg r) &= w(\neg p)w(\neg r), \\
%   \end{align*}
%   But then we have that
%   \[
%     \frac{m(p \land r)}{m(\neg p \land r)} = \frac{w(p)}{w(\neg p)} =
%     \frac{m(p \land \neg r)}{m(\neg p \land \neg r)}.
%   \]
%   This places a condition on $m$, contradicting the assumption that the
%   construction works for an arbitrary $m$. Hence $b < a$.

%   Second, we can show that if $b = a \land \bigwedge_{i = 1}^k \phi_i$ for some
%   positive integer $k$, then there must also be $2^k - 1$ other atoms in
%   $\mathbf{B'}$ that correspond to every possible way to negate a subset of
%   $\phi_i$'s, i.e., ranging from

%   % 2. If we add phi, then we must also add -phi.
%   % 3. Extension to multiple literals: we must have all (2^n) combinations of
%   % added literals).
%   % 4. Profit
% \end{proof}

Let us note how our lower bound on the number of added literals compares to two
methods of translating a discrete probability distribution into a WMC problem
over a propositional knowledge base proposed by Darwiche
\cite{DBLP:conf/kr/Darwiche02} and Sang et al. \cite{sang2005solving}. Suppose
we have a discrete probability distribution with  $n$ variables, and the $i$-th
variable has $v_i$ values, for each $i \in [n]$. Interpreted as a logical
system, it has $\prod_{i=1}^n v_i$ models. My expansion would then use
\[
  \sum_{i=1}^n v_i + 2\prod_{i=1}^n v_i
\]
variables, i.e., a variable for each possible variable-value assignment, and two
additional variables for each model. Without making any independence
assumptions, the encoding by Darwiche \cite{DBLP:conf/kr/Darwiche02} would use
\[
  \sum_{i=1}^n v_i + \sum_{i=1}^n \prod_{j=1}^i v_j
\]
variables, while for the encoding by Sang et al. \cite{sang2005solving},
\[
  \sum_{i=1}^n v_i + \sum_{i=1}^n (v_i - 1) \prod_{j=1}^{i-1} v_j
\]
variables would suffice.

% TODO: a section on WMI: 3 results
% TODO: introducing evidence can be done by taking a quotient w.r.t. the ideal
% that contradicts the evidence (equivalently, the filter generated by the
% evidence).

\bibliographystyle{plain}
\bibliography{paper}

\end{document}
